\section{Электронная почта}

Письмо, посланное Младеном Константину в Корнелл.
Следует принимать во внимание то, что в письмах к своему другу Младен имеет
обыкновение называть Варанга шефом Константина, хотя тот не является его научным
руководителем.

\begin{quote}
From: mladen.limic@math.adv-studies.edu

To: constantine.dimitriakis@cornell.edu

Subject: Hi

\medskip
Привет,

Видел вчера твоего шефа и решил тебе написать.
За те пару недель, что он отсутствовал, важности у него прибавилось хоть
отбавляй.
Но при этом очень довольный "--- ходит и на всех посматривает с такой
снисходительно"=великодушной улыбкой.
Думаю, если ты собираешься написать ему о том, что снова хочешь приехать в
институт, сейчас "--- самое подходящее время.
Не упускай момент, пока у него такое доброе настроение!

Ещё до того, как он вернулся, во многих газетах написали о полученной им медали.
У нас в институте, около чайной комнаты, повесили вырезки из наиболее интересных
статей.
Только вот журналисты, ты знаешь, как всегда всё на свете умеют перепутать: в
одной статье вместо его фотографии по ошибке поместили фотографию его младшего
брата!
Но это ещё что, автор другой статьи вообще не знает, в какой стране находится
Пекин (или совсем не смотрит на то, что пишет):
\enquote{\ldots\ крупнейший математик нашей страны профессор Варанг, только что
получивший в Пекине престижную награду \ldots\ после завершения торжественной
церемонии и приуроченному к ней конгрессу возвращается завтра домой из Японии!}

Короче, приезжай, сам увидишь, если ещё не читал эту статью.

Как дела у вас там в Корнелле?

пока, Младен
\end{quote}

Приведённое выше письмо "--- стандартный образчик переписки между Константином и
Младеном, такого рода письмами они обменивались по крайней мере раз в неделю.
А вот, напротив, достаточно необычное письмо, посланное профессором Моррисом
профессору Мэйли.

\begin{quote}
From: phil.morris@math.adv-studies.edu

To: philippe.mayley@math.adv-studies.edu

Subject: Re: Lunch

\medskip
Дорогой Коллега,

Вот ещё список некоторых потенциальных визитёров и постдоков.
Привожу только те фамилии, которые мы не успели с Вами в прошлый раз обсудить.

Брюс Стивенсон, 45~лет, 95~кг примерно

Изабелль Фрост, 29~лет, $\approx$ 50~кг, (по некоторых признакам страдает
малокровием)

Петер Монк, 37~лет, 70~кг примерно

Стив Лайонс, 35~лет, вес не установлен

Мартин Ланг, 58~лет, 70~кг

Кин Хменг, 26~лет, 65~кг примерно

Мария Рождественская, 25~лет, 60~кг

Ян Дворецки, 25~года, 70~кг примерно

Роберто Паризи, 23~года, вес не установлен

Напишите, дорогой коллега, что Вы о них думаете.
Завтра после чая собираюсь быть в институте, так что можно будет также лично всё
обсудить.

С уважением, Филипп.
\end{quote}
