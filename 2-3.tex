\section{Конфуций, Гвенаэль и Альсбета}

Ситуация в городе тем временем становилась мрачнее.
Император R.~ле~Кин издал указ, запрещающий театральные представления.
Поводом для него послужил театр комедии, на сцене которого часто появлялись
Пьерро и Арлекин.
Правитель усмотрел в этом намёк на себя и своего нового премьер"=министра
Пьерро.

Потом с прилавков книжных магазинов стали исчезать книги.
Вроде бы никто их не запрещал, но многие издания стало совершенно невозможно
найти.

И ещё бездомную кошку теперь никогда уже не встретишь, но это не совсем уже
новость, кошек отловили давно, сразу после к приходу к власти R.~ле~Кина.

Все эти изменения происходили постепенно, о каждом из них поначалу много
говорили, но вскоре забывали.
Так что мало кто осознавал, что в итоге жизнь очень заметно отличалась от того,
что было раньше.

Омер Раджафер, погруженный в написание своего нового романа, и вовсе не был в
курсе многих последних событий.
Но проснувшись однажды утром он почувствовал, что что-то в этом городе
происходит не так, просто нестерпимо не так.

Раньше кое-что из жизни города можно было узнать из газет, но в последние время
газеты стали совсем скучными и бессодержательными.
К счастью был другой, более надёжный способ узнавать свежие новости.
Надо было дождаться среды, потому что именно среда была базарным днем, и пойти с
самого утра на городской рынок.

Во вторник вечером Омер завёл будильник, чего давно уже не делал в последнее
время, но это оказалось излишним "--- следующим утром он сам проснулся очень
рано, почти за час до намеченного, быстро собрался и пошёл на городской базар.

Для начала он купил три зелёных яблока и спросил при этом у продавщицы:

"--* Что нового в городе?
О чём говорят на рынке?

"--* Говорят, что к осени подорожает сахар, "--- ответила торговка яблоками.
"--- А ещё пойдите, спросите у моей свояченицы "--- вон там, видите, та, что
продаёт мороженую рябину "--- она обычно больше меня знает.

Омер протолкался через толпу, отделявшею его от соседнего прилавка, купил
маленькую баночку рябинового варенья и спросил:

"--* Что слышно, о чём говорят нынче на рынке?

"--* Не знаю, сынок, "--- ответила торговка.
"--- Я на прошлой неделе застудила левое ухо, и что-то плохо слышу теперь.
Думаю, тебе лучше спросить у продающего персидские ковры.

На этот раз Омеру пришлось пробираться в противоположный конец рынка.

Торговец коврами начал собрался было начать расхваливать свой товар, но поглядев
пристально на Омера понял, что у того вряд ли хватит денег даже на самый что ни
на есть маленький персидский коврик.
Недовольный этим открытием, он произнёс весьма сердитым голосом:

"--* Иди, найди премьер"=министра.
И поторопись, тебе стоит найти его до сегодняшнего полудня.

"--* Премьер"=министра? "--- очень удивился Омер.
"--- Не представляю, о чём бы я вообще мог разговаривать с министром
Пьерро \ldots

"--* Да нет же, "--- презрительно ответил торговец (было непонятно, относилось
ли его презрительность к непонятливости Омера, к министру Пьерро, или и к тому,
и к другому).
"--- Я говорю, конечно, про бывшего премьер"=министра Конфуция.
И давай, ступай, а то ты мне заслоняешь ковры от покупателей.

Уже почти у самого выхода Омер увидел продавца старыми книгами.
В обычных книжных магазинах последние время трудно было купить что-либо
интересное, но здесь, на рынке, всё кажется было по"=старому.
Внимание Омера быстро привлёк золочёный томик Макбета очень старого издания,
сборник пьес Евгения Шварца, тоненькая книжечка стихов Хармса и ещё какая-та
книга, называвшаяся \enquote{Урожаи и посевы}, незнакомая ему, но сразу же
бросавшаяся в глаза благодаря своей ярко"=голубой обложке.

Раджафер очень любил разглядывать старые книги, и в другое время провёл бы у
этого прилавка не менее часа, но сейчас надо было торопиться.
До полудня было ещё порядочно времени, но ведь неизвестно, сколько времени
придётся потратить на поиски премьер"=министра.
Правда, Омер надеялся, что найти его будет нетрудно "--- говорили, что после
ухода в отставку он большую часть дня проводит у себя дома.
И хотя Омер никогда раньше не был у него в гостях, он хорошо знал, где находится
его дом.
Министр жил на площади Александера, и до недавнего времени, когда кто"=нибудь
упоминал её, почти всегда добавляли: на площади Александера, вы знаете, это там,
где находится дом премьер"=министра.

Место это было довольно интересным.
В отличии от многих других площадей города, бывших шестиугольными, площадь
Александера была круглой, и по середине её стояла забавная скульптура.
Это было что-то вроде большого белого шара, из которого на встречу друг другу
выходили два толстых рога, на конце у каждого из них было два рога потоньше,
потом из каждого тонкого роге ещё по два тоненьких рога и так далее.
Каждый раз рога не дотягивались друг до друга, но в целом казалось, что всё это
сооружение из рогов в конце концов каким-то хитрым образом зацеплено.
Точно нельзя было разглядеть, потому что с какого-то момента рога становились
совсем маленькими, невооружённым глазом было не увидеть, что там именно с ними
происходило.
И ещё потому, что на скульптуре обычно сидели голуби.

В этот день их было видимо"=невидимо.
Некоторые из них суетливо взлетели, увидев проходящего мимо них Омера, но вскоре
уселись обратно на белые рога полюбившейся им скульптуры.

Раджафер не ошибся в своих ожиданиях, премьер"=министр оказался у себя дома.
Он сидел, склонившись над шахматной доской, и не то составлял, не то решал
какой-то шахматный этюд.

Когда Омер вошёл, Конфуций передвинул коня с G2 на F4, после чего спросил у
нежданного гостя, не хочет ли тот сыграть с ним партию.

"--* Нет, я не умею играть, "--- извинился Омер.
"--- Даже не знаю толком, как ходят фигуры.
В детстве я научился играть только в игры попроще, в шашки или в трик"=трак.

"--* Попроще?! "--- удивился Конфуций.
"--- Наверное, вы по старинке играли в шестидесяти четырёх клеточные шашки.
Потому что теперь, вы знаете, принято играть на стоклеточной доске, и это очень
сложная игра.
И так в шахматах никогда не знаешь, что делать с клетками E7 и E8, а в шашках
теперь есть ещё E9 и E10, это совсем уже трудные клетки, хорошо хотя бы то, что
E10 белая, и шашки по ней не ходят.

Омер не понял последнего замечания премьер"=министра, но в любом случае он
пришёл вовсе не для того, чтобы разговаривать о шахматах или шашках.
Он минутку помолчал, потому что ему было несколько неловко переводить разговор
на более серьёзную тему, и потом сказал:

"--* Я пришёл поговорить с вами о последних малоприятных событиях.
Надо что-то делать, я уверен, есть ещё способ всё это изменить.
И вы, я думаю, его знаете.

"--* Способ должен быть, но вы ошибаетесь, я его не знаю, "--- ответил
Конфуций.
"--- Обо всём этом пора подумать молодым.
Знаете, я ведь ушёл в отставку не только из-за смены правительства.
Я занимался этим городом больше сорока лет, и нет ничего, чем можно было бы
заниматься бесконечно.

"--* Бросьте, вы не хуже меня знаете нашу молодёжь, "--- возразил ему Омер.
"--- Одни "--- вроде бы неглупые "--- поддались на соблазны R.~ле~Кина и его
двора.
Есть другие "--- почестнее и попрямей душой, но без большого ума тоже далеко не
уйдёшь \ldots\
Так что если уж вы ничего не придумаете, у других тем более вряд ли получится.

"--* Ладно, я попробую, "--- устало и грустно ответил Конфуций.
"--- Но учтите, либо всё удастся успеть к началу следующей недели, либо вам не
придётся больше рассчитывать на мою помощь.
Через неделю мне исполняется 80~лет, и после этого в политических делах я уже
точно участвовать не буду.

"--* Мне надо будет покопаться в старых книгах, "--- добавил он после некоторого
размышления, кажется, что-то вспомнив и несколько оживившись.
"--- По-моему, однажды я видел одну книгу с заклинаниями, возможно, она могла бы
нам помочь \ldots\
Приходите снова завтра утром "--- часов в~7, это для вас не слишком рано?
Мы всё тогда толком обсудим.

Когда Омер собрался уже уходить, из соседней комнаты выскользнула пушистая
чёрная кошка и стала тереться об его ноги.
Омер наклонился её погладить, и она довольно замурлыкала, зажмуривая ненадолго
от удовольствия свои круглые жёлтые глаза.
Омер погладил её снова, почесал ей шею "--- кошка была чёрная"=чёрная, и только
на шее было небольшое белое пятнышко "--- и спросил у премьер"=министра, как её
зовут.

"--* Гафа, "--- ответил Конфуций.
"--- Она любит провожать моих гостей.
Иногда доходит с ними почти до самого дома, и только потом возвращается.
Но последнее время я боюсь отпускать её на улицу без присмотра.
Так что нет, Гафа, дай Омеру выйти, не загораживай дверь, ты останешься со мной,
не надо его провожать.

По дороге домой Омер решил пройтись вдоль по набережной.
Это был не самый короткий путь, но погода была хорошей, и ему хотелось
посмотреть, как плещутся в прозрачной воде Кортевеговки разноцветные рыбки.

Подходя к реке он увидел Альсбету и Гвенаэля.
Он хотел их окликнуть, но промолчал, увидев, что они о чём-то напряжённо спорят.
Он подошёл поближе, но они так и не замечали его: принцесса "--- потому что
стояла к нему спиной, Гвенаэль "--- потому что был слишком занят тем, что
говорила ему в этот момент принцесса.

"--* \ldots\ ты настолько изменился, всё, что ты говоришь теперь, так холодно
и бездушно!
Я бы и вовсе уже забыла, каким ты был раньше, если бы не хранила твои старые
письма.

Тут Альсбета достала из потайного кармашка своего синего платья сложенный лист
бумаги, исписанный мелким почерком Гвенаэля, и продолжила:

"--* Вот, это моё любимое письмо, ещё не так давно ты писал:
\enquote{\ldots мне так хочется, Альсбета, чтобы ты была счастлива, очень-очень-очень
счастлива \ldots}

И, несколько более тихим и смущённым голосом она прочла вслух ещё несколько
отдельных строчек письма:

"--* \enquote{\ldots\ голубые глаза твои, большие и яркие, ямочка на правой щеке,
когда ты смеёшься \ldots}
Мне даже трудно поверить, что тот, кто писал эти строчки, и ты сегодняшний,
такой недобрый и раздражительный, "--- это один и тот же человек!

Было, конечно, довольно невоспитанно слушать столь личный разговор, к тебе не
относящийся, и Омер был заметно взволнован услышанным, но уходить теперь было
поздно: Гвенаэль, наконец, заметил его присутствие.

"--* Ямочка на правой щеке, когда ты смеёшься "--- ты писал такие письма,
Гвенаэль? "--- спросил Омер, и принцесса вздрогнула, потому что не ожидала
услышать за своей спиной его голос.
Она робко улыбнулась ему, потом повернулась обратно к Гвенаэлю, думая, что тот
сейчас ответит, что нечего Омеру лезть в дела, совершенно его не касающиеся "---
это было бы очень на него похоже, к тому же в данном случае, такой ответ не был
бы безосновательным "--- но Гвенаэль ничего не сказал и повёл себя в высшей
степени странно.

Он вырвал из рук Альсбеты своё письмо, скомкал его, быстро перевёл свой взгляд с
принцессы на Омера, потом снова обратно на принцессу, открыл рот, как будто
собирался что-то ответить, но, издав непонятный гортанный звук, по тембру
похожий на заикание, замолчал, так ничего и не сказав.

"--* Что это значит? "--- спросила Альсбета, испуганная его странными
действиями.
"--- Почему ты отнял у меня письмо?

"--* Спроси у Омера, "--- ответил Гвенаэль каким-то неживым голосом.
"--- Я не писал этого письма, оно было частью его нового романа \ldots\
Мне уже всё равно теперь \ldots

"--* Да, да, можешь не смотреть на меня так удивлённо, "--- добавил он громче,
по"=странному воодушевляясь.
В его серых глазах, обычно довольно спокойных, танцевали злые и безумные
искорки.
"--- Да, я полный подонок, залез когда-то в комнату Омера, залез как грабитель,
через окно, с крыши дома напротив "--- это совсем нетрудно, Кошачий переулок
ведь совсем узкий, это ничего не стоит сделать!

Говорил Гвенаэль очень быстро, намного быстрее, чем обычно, казалось, что начав
говорить он не может остановиться пока всё не скажет:

"--* Да, как самый обычный вор, я залез через окно в комнату к Омеру и украл
несколько страниц из его рукописи, украл из чистого любопытства вначале, даже не
думал вначале красть, хотел просто посмотреть.
Но потом читать было темно, и глупо было ждать, пока кто"=нибудь обнаружит
моё присутствие.
И ещё потом оказалось, что на этих страницах письмо, возможно весь роман был из
писем, но скорее нет, это я так никогда и не узнал.
А ровно в тот вечер мне надо было написать письмо, которое никак не удавалось
написать, и я решил "--- раз уж всем так нравится, как пишет Омер, наверное, и
письма его должны нравиться больше чем обычные письма, так ведь?
И смешно сказать, что я не ошибся, как можете видеть, я не ошибся!

Закончив говорить, он и вправду засмеялся, вздрагивая всем телом и судорожно
моргая обоими глазами.

Принцесса прикусила краешек нижней губы и нахмурила лоб, пытаясь понять то, что
сказал ей Гвенаэль.
Она вопросительно посмотрела на Омера, тот молчал, она опустила глаза и
продолжала думать, и вдруг с ней что-то произошло, непонятно, как это можно
описать "--- у Омера было ощущение, что у неё внутри что-то сломалось, во всяком
случае и ему, и Гвенаэлю стало ясно, что она наконец поняла и поверила в
сказанное.
Глаза её наполнились слезами, она резко мотнула головой и побежала быстро по
набережной, прочь от Раджафера и Варанга.
Омер попытался было что-то сказать ей, но она обернулась и крикнула:

"--* Я не хочу вас видеть, вас обоих, не хочу больше видеть ни тебя, ни его!

Отчаяние в её глазах в этот момент было столь выразительным, что Омер замолчал и
стоял в оцепенении до тех пор, пока она не скрылась из виду.
Гвенаэль сделал неловкий шаг в сторону воды, и чуть не свалился в неё, в
последний момент сумев удержать равновесие и даже не набрать воды в ботинок.
Потом он быстрым шагом ушёл, кажется, в сторону противоположную той, куда
убежала принцесса.
Точно Омер не мог потом припомнить, потому что окончательно он пришёл в себя
только тогда, когда ни Альсбеты, ни Гвенаэля уже не было видно.

Он побрёл домой.
Красные, зелёные, и синие рыбки плескались в Кортевеговке и даже иногда
выпрыгивали из воды, но Омер шёл, абсолютно их не замечая.
Он не видел также, что некоторые из появившихся на набережной прохожих пытались
с ним поздороваться "--- он шёл задумавшись, ни на что не глядя, и ему хотелось
как можно скорее оказаться дома.
