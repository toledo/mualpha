\section{Дом, поддерживаемый пятками деревянного атланта}

Только вернувшись домой, Омер почувствовал, насколько он устал.
Эта была приятная усталость "--- когда ты долго пытался сделать что-то важное и
очень трудное, почти не веря, что оно получится, и, наконец, всё получилось и
закончилось, и у тебя совсем уже нет сил, даже нет сил радоваться успеху,
хочется лечь и лежать не двигаясь, не спать (Омер уже достаточно проспал за эти
последние дни), просто лежать и ни о чём не думать, во всяком случае не о
математике, там более что математика "--- это обычно для той, другой жизни по ту
сторону ото сна, так вот, не только не думать о математике, но и вообще
совсем"=совсем ни о чём\ldots

Но Омер не мог себе этого позволить.
Ведь сколько всего ни произошло за последний день, особенно за последние его
часы, как ни изменился город после исчезновения императора (Омер, кстати,
заметил уже из окна своей мансардной квартирки первого котёнка на крыше дома
напротив. Котёнок, казалось, размышлял, не запрыгнуть ли ему в комнату к Омеру,
но потом передумал, и просто перепрыгнул на соседнюю крышу), так вот, сколько
всего ни произошло, сколь быстро всё уже ни менялось к лучшему, оставалось одна
проблема, по-прежнему очень тревожившая Раджафера.

\enquote{Принцесса Альсбета.
Она с детства, во всяком случае после того, как пропали её родители, была очень
ранимой и впечатлительной девочкой}, "--- думал Омер, уже снова выходя из
главного подъезда дома номер~53 по Кошачьему переулку.
Ещё он думал о том, что она была очень влюблена в Гвенаэля, и каким должна была
быть для неё ударом эта история с украденным письмом, и о том, что он сам был
отчасти виноват во всём этом.
Его мучили угрызения совести, наверное, он не должен был тогда дать понять, что
письмо было похищено Гвенаэлем из рукописи его неоконченного романа.
Но, с другой стороны, тот сам об этом проговорился, и в любом случае, эта
история с письмом, даже если бы он тогда не оказался при их разговоре, была
слишком скверной.

Когда-то ему казалось, что принцесса и Гвенаэль так мило влюблены друг в друга,
что будут наверняка очень счастливы вместе, и он заранее за них радовался,
особенно за Альсбету.
Гвенаэль тогда ему не то что бы очень нравился, но всё же казался довольно
симпатичным и способным молодым человеком.
Но последнее время он так изменился, или Омер раньше просто не замечал,
насколько он заносчив и как-то по-болезненному тщеславен?

В общем, последнее время Гвенаэль уже вовсе не казался Раджаферу симпатичным, и
мысль о его предстоящей свадьбе с Альсбетой казалась ему всё менее и менее
удачной, тем более, что он замечал, что сама принцесса очень часто огорчается и
обижается на Гвенаэля.

Надо также сказать, что отношения между Раджафером и Варангом с самого начала
не были очень дружескими.
Омер списывал это на то, что Гвенаэль завидует его литературной популярности.
Он пытался убедить себя в том, что это ничего, по молодости лет не страшно,
главное, что Гвенаэль с Альсбетой хорошо ладят, так ли важно как он к нему,
Омеру, относится?
Но потом Гвенаэль стал меняться, Альсбета "--- грустнеть, потом наконец эта
история с украденным письмом.
После неё уже трудно было простить Гвенаэля, какая девушка простит своего
жениха, если он даже любовное письмо сам написать не может и крадёт его у
другого?
Так что Омер в конце концов не был виноват в том, что эта история выплыла наружу,
иначе ещё хуже всё могло бы оказаться.
Но почему-то, несмотря на эти рассуждения, его совесть по-прежнему была
неспокойна.
Не радовался ли он в тайне размолвке принцессы и Гвенаэля?
И ещё одна совсем уж нелепая мысль промелькнула у него в голове: украденное
письмо было черновиком и ни для чьих глаз не предназначалось, особенно для глаз
Альсбеты.
Жалко, что она прочитала его в этом черновом варианте.
Но какое всё это имело значение, когда главное было понять, где сейчас принцесса
и как она?
У неё ведь был такой безнадёжный и несчастный вид, когда она убежала тогда в
слезах, оставив его и Гвенаэля стоять на набережной\ldots

С этими мыслями Омер проделал недолгий путь до дома Альсбеты, подошёл к
двухэтажному зданию с прибитой над входом большой синей табличкой \enquote{№~28},
поднялся на крыльцо, на котором стоял на плечах деревянный человечек,
поддерживавший пятками крышу над крыльцом, и постучал в дверь.

Никто не ответил, но так бывало часто: если принцесса была наверху, она обычно
не слышала стука при входе.
Он толкнул дверь, которая почти никогда не запиралась, и вошёл в дом.
На вешалке при входе он увидел знакомое серое пальто в крупную клетку, и
обрадовался тому, что принцесса, должно быть, дома.

Он громко крикнул: \enquote{Альсбета}, но по-прежнему никто не ответил, и он
подумал с грустью: \enquote{Какой же я дурак, сейчас уже слишком тепло, чтобы
ходить в пальто, наверное, её всё-таки нет}.

Он зашёл на кухню и увидел небрежно разбросанную по столу посуду "--- чайник,
чашку, щипцы для сахара "--- и снова обрадовался.
Видимо, принцесса была здесь совсем недавно, и, если она и вышла из дома, то
наверняка скоро вернётся.
Это было совсем на неё не похоже, куда"=нибудь далеко уходить, даже не прибрав
со стола посуду.

Может быть, она даже дома, но не услышала его.
На всякий случай он снова крикнул: \enquote{Альсбета} и потом ещё раз, но тише:
\enquote{Альсбета} и в последний раз уже почти шёпотом произнёс:
\enquote{Альсбета}.
Он повторял её имя, не потому, что всё ещё надеялся, что она здесь, а потому,
что тишина в доме начала казаться ему пугающей, и ему хотелось хоть как-то её
нарушить.
Он пытался успокоиться и говорил сам себе:
\enquote{Вот чашка и чайник на столе, она, должно быть, только что пила чай, и
скоро вернётся}.

Но когда он подошёл к столу и разглядел получше стоявшую на нём чашку, то по
следу засохшего на её донышке чая понял, что ошибся: из этой чашки последний раз
пили пару дней тому назад, по крайней мере.

\enquote{Я даже не знаю, как долго я спал перед походом в императорский дворец,
надо было спросить об этом у Конфуция}, "--- сказал он сам себе с упрёком.

Он сел на диван, стоявший в углу кухни, и грустно задумался, спрятав лицо в
ладони.
В какой-то момент он его открыл и пробормотал:
\enquote{Нет, нет, только ничего страшного, ничего плохого!
Это моя ответственность, история принцессы не должна кончиться грустно, ещё с
самого её детства это было моей ответственностью}.
После чего он снова закрыл лицо руками и продолжал о чём-то думать.
