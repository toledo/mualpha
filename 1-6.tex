\section{Яблочные косточки и миндаль}

"--* Сегодня будем готовить порошок от кашля, "--- сказал аптекарь своему
помощнику, рыжему веснушчатому парню лет семнадцати или восемнадцати.
Тот нехотя почесал свой вихрастый затылок и спросил:

"--* Порошок от кашля?
Из тыквенных семечек?
Мне кажется, ещё с прошлого года должно было кое-что остаться.

"--* Из тыквенных семечек? "--- возмутился аптекарь.
"--- Где это видано чтобы порошок от кашля готовили из тыквенных семечек?
Я знаю несколько неплохих рецептов, но ни в одном из них тыквенные семечки не
участвуют.
Да, да, есть несколько неплохих рецептов, но один из них я люблю больше всего.
Достань"=ка банку с миндалём и банку с яблочными косточками, потом надо будет
всё это хорошенько истолочь в пропорции два к одному.
Толочь будешь в большой ступке, погода стоит холодная, и почти всё время идёт
дождь: что-то мне подсказывает, что порошок от кашля нам понадобится в немалых
количествах.

Помощник залез на приставную лесенку и стал искать нужные ему банки.
На банке с миндалём так и было написано \enquote{миндаль} (красивыми фиолетовыми
буквами по белой эмали), и он её довольно скоро нашёл, а вот на банке с
яблочными косточками было написано
\enquote{\foreignlanguage{latin}{semina mali}},
что означает на латыни \enquote{семена яблок}.
Латынь помощник аптекаря знал плохо, и по ошибке попытался вначале достать банку
с грушевыми косточками.

"--* Яблочные, яблочные косточки, а вовсе не грушевые! "--- немного сердито
повторял аптекарь.
"--- Грушевые косточки "--- неплохое средство от бессонницы, и от ревматизма
иногда тоже помогают, но к кашлю они никакого отношения не имеют!
Нам сегодня нужны яблочные косточки, именно яблочные "--- вон они там в той
зеленоватой фарфоровой банке справа от тебя, на третьей сверху полке.
Неужели не видишь?

Помощник аптекаря нашёл, наконец, нужную банку и удалился в прихожую, потому что
именно там стояла большая каменная ступка, в которой ему предстояло мелко
растолочь косточки и миндаль.
А аптекарь стал смотреть в окно на крупные капли дождя, падающие время от
времени в немного беспокойную все эти последние дни воду реки Кортевеговки.

\medskip
\begin{center}
*\quad*\quad*
\end{center}

\medskip
Обычно к пятичасовому чаю подавалось печенье или шоколадные кексы, но
сегодняшний день был почему-то особенным: сегодня к чаю были испечены два
огромных пирога, один с яблоками, другой с миндалём.

Константин съел уже два куска яблочного пирога и кусок миндального, и,
прислушиваясь к математическим и нематематическим разговорам в чайной комнате,
мучительно пытался решить для себя следующий вопрос: прилично ли было съесть ещё
один кусочек?

В углу комнаты висела доска, рядом с ней стояли два человека, Константину
незнакомых, и один из них что-то писал на доске, приговаривая при этом:

"--* Если мы поверим на мгновение в функториальность, то кручения здесь быть не
должно \ldots\
После редукции по модулю $p$ наша формула принимает вид \ldots

"--* Можно ли сравнить вклад Шекспира в исследование любви со вкладом Александра
Гротендика в алгебраическую геометрию? "--- спрашивал своего соседа один из
профессоров, запивая маленькими глотками чая кусочек яблочного пирога.

"--* Погружая полученное многообразие в $\mathbb{R}^5$ и рассматривая его
небольшую окрестность \ldots "--- говорил другой.

"--* Считаете ли вы, что в связи с последними событиями появилась надежда на
улучшение ситуации в Судане? "--- спрашивал кого-то третий.

\enquote{Интересно, придёт ли сегодня Варанг на это пятичасовое чаепитие}, "---
думал в это время Константин.
За последние дни ему удалось уже два раза достаточно основательно поговорить с
Варангом, основные имевшиеся у него вопросы он ему уже задал, но институтский
чай был всегда наиболее благоприятной возможностью для небольшой математической
беседы.
Поэтому почти все посетители института старались его не пропускать, даже те из
них, кто не особенно любили пить чай.

Но сегодня, несмотря на вкусные пироги, людей в чайной комнате было не так уж
много.

\enquote{Если бы все присутствующие в институте пришли на сегодняшний чай, то
каждому достался бы только один кусок яблочного пирога и один кусок миндального,
или, может быть вообще всего только один кусок на выбор: того или другого.
Но, принимая во внимание то обстоятельство, что время чаепития подходит к концу
и что людей в чайной комнате по-прежнему немного, пожалуй, я вполне имею право
на ещё один кусочек миндального пирога \ldots}

"--* Нормализуя, разделив на диаметр, и следя за кривизной при переходе к
пределу, "--- говорил кто-то неподалёку от Константина.

"--* Визовая политика Соединённых Штатов по отношению к странам восточной
Европы \ldots

"--* Оценивая $L^p$ норму обратного оператора и предполагая неотрицательность
второго собственного числа \ldots

\enquote{Нет, всё-таки пожалуй не очень удобно взять ещё один кусочек, "---
думал Константин.
"--- Учитывая то, что я и так уже съел один миндальный и целых два яблочных}.

Наверное, он весьма долго мог бы раздумывать на тему вышеупомянутого вопроса, но
в этот момент его размышления были внезапно прерваны.
Дверь чайной комнаты распахнулась, и весёлым и быстрым шагом в неё вошёл
профессор Франклин, вслед за которым вбежала его собака, огромный и очень
светлый, почти что белый, золотистый ретривер.

Франклин взял со стола кусок пирога, откусил от него немножко и кинул оставшуюся
часть куска своей собаке.
Та мигом его проглотила и довольно облизнулась.

"--* Понравилось? "--- спросил у неё Франклин.
"--- Мне тоже показалось, что довольно вкусно \ldots

После чего "--- хоп-хоп-хоп "--- он стал кидать своей собаке кусок за куском,
до тех пор пока на большом подносе для пирогов остались одни только аппетитные
крошки.

\medskip
\begin{center}
*\quad*\quad*
\end{center}

\medskip
Письмо, отправленное вчерашней почтой, должно было прийти сегодня утром.
Даже если почтальон несколько замешкался, теперь уже она его точно должна была
получить.
Прочесть письмо "--- всего пара минут, но на всякий случай можно добавить ещё
полчаса, вдруг принцесса была чем-то занята в тот момент, когда пришёл
почтальон, или, может быть, что-то могло её отвлечь, пока она читала письмо.

"--* Нет, теперь она наверняка его уже прочла, "--- сказал сам себе Гвенаэль,
глядя на большие часы на башне городской ратуши.
Он быстрым шагом направился в сторону Безымянной улицы, и через несколько минут
уже стоял перед домом №~28 "--- столь хорошо ему знакомым старым двухэтажным
зданием, крылечко которого подпирал своими пятками делающий стойку на плечах
деревянный человечек.

Но перед входом он внезапно остановился.

"--* Чёрт побери, совершенно дурацкая была затея посылать это письмо.
Можно подумать ты сразу не понимал, насколько она была дурацкой.
Надо было просто прийти "--- причём желательно вчера, а не сегодня "--- и
попросить у Альсбеты прощения, мы наверняка бы помирились.

Или даже после того, как я отправил письмо, можно было ведь прийти вчера вечером
и попросить её не читать его, когда она его получит, просто извиниться и
сказать, что я очень хочу помириться и больше никогда уже не ссориться.

Но теперь уже поздно об этом думать, теперь она не только получила, но и прочла
уже письмо, наверняка уже прочла!

Гвенаэль поднялся наконец на крылечко, поддерживаемое деревянным перевёрнутым
атлантом, и, робея, постучал во входную дверь.

Когда Альсбета была на втором этаже своего дома, она часто не слышала стука у
входа.
Но сегодня она не только была внизу, но, вероятно, уже ждала прихода Гвенаэля
"--- не успел он опомниться, как дверь распахнулась, принцесса выбежала на
крыльцо и бросилась к нему в объятья.

"--* Я так была тронута твоим письмом, Гвени!
Мне даже сразу стало ужасно совестно из-за того, что я когда"=либо могла на тебя
сердиться.
И знаешь, ты вот написал, что пишешь неловко, а мне казалось, что твоё письмо не
хуже написано, чем любая из твоих книг.
Даже, если честно, мне показалось, что оно написано гораздо лучше!
Я понимаю, это глупости, моё мнение по этому вопросу очень уж необъективно!
Всё дело в том, наверное, что я очень в тебя влюблена, и так приятно получать
такие письма от кого"=то, в кого ты очень влюблён.
В общем, ты как мило всё написал, особенно про ямочку на щеке, я сама про неё
толком не знала, но прочтя письмо, посмотрела в зеркало и заметила \ldots

Произнося эти фразы, Альсбета прижималась лицом к груди Гвенаэля и не могла
видеть выражения его лица.
Её наверняка удивило бы, если бы она узнала, что всё это время лицо её жениха
оставалось бледным и напряжённым.

Наконец он перестал хмуриться и улыбнулся: \enquote{Она простила меня за нашу
ссору, главное, что она меня простила!} "--- и нежно провёл рукой по её
белоснежным волосам.
Она подняла голову и сказала:

"--* Я всё болтаю без умолку, а ты всё время молчишь чего-то?
Ты какой-то очень серьёзный сегодня?
Ты о чём-то думаешь?
Мне не угадать по твоим глазам о чём \ldots

"--* Нет, нет, ни о чём я не думаю.
То есть если думаю, то о тебе, и ни о чём больше!

Последние слова Гвенаэль произнёс немного смущённо и тут же закашлялся, стараясь
скрыть свою неловкость.

"--* Ах, какая же я глупая! "--- спохватилась Альсбета.
"--- До сих пор заставляю тебя стоять на крыльце, а утро такое холодное сегодня!
Пойдём скорее в дом, там тепло.
И, знаешь, я вот тоже вчера вечером начала кашлять, и даже купила уже в аптеке
специальное средство от кашля.

Она увлекла его за собой на кухню, сняла с огня закипевший как раз в это
мгновение чайник и бросила в чайную чашку Гвенаэля щепотку какого"=то порошка,
пахнущего миндалём и яблочными косточками.
