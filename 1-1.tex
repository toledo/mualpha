\section{Константин Димитриакис}

Меня зовут Константин Димитриакис, я родился в Греции и закончил университет в
Афинах по специальности математика.
Многие, когда слышат слово математика, тут же говорят, что это безумно сложное
занятие, но мне так не казалось.
Я тратил не слишком много усилий на учёбу, и при этом считался одним из лучших
студентов своего курса.
Мой научный руководитель был очень доволен моей дипломной работой, и не
подозревал, что, занимаясь ею, я не меньше времени тратил на плаванье в море,
походы, вечеринки и игру в карты.

В аспирантуру я решил поехать в Америку: так мне советовал мой научный
руководитель, так хотели мои родители, к тому же мне самому было любопытно
немного пожить в какой"=нибудь малознакомой пока стране, не говоря уж о том что
аспирантские стипендии в Америке были большими и поэтому весьма привлекательными.

Начало лета я провёл на море, потом навестил родителей и съездил к бабушке и
дедушке, живущими на острове Итака.
В отличии от родителей бабушка с дедушкой плохо понимали, что такое Америка, для
них эта была неведомая страна, находящаяся за огромным океаном.
Но, как и все живущие на островах люди, они знали, что океан "--- это просто
большое"=большое море, так что Америка для них была просто большим незнакомым
островом, намного более далёким и большим чем Кефалония, но чем-то её, возможно,
напоминающим.
Их немного беспокоила моя предстоящая поездка, и они расстраивались, что теперь
будут видеть меня гораздо реже, чем пока я жил в Афинах. Но они сами понимали,
что, к примеру, на Кефалонии не бывает аспирантуры по математике, и мало"=помалу
согласились с тем, что раз уж я решил ехать в Соединённых Штаты, то так тому и
быть.

Покинув Итаку греческую, островную и скалистую, я переселился в Итаку
американскую, плоскую и лесистую, а более точно в находящийся в американской
Итаке город Корнелл, потому что отныне я был аспирантом Корнелльского
университета.

Первый год моего пребывания в Америке был очень весёлым.
Я немного скучал по морю и по оставшимся в Греции друзьям, но у меня появилось
много новых друзей, а студенческая жизнь в Корнелле была не менее бурной, чем в
Афинах.
Заниматься математикой приходилось немного больше, чем раньше, но я по"=прежнему
считал, что эти занятия очень неплохо можно совмещать с развлечениями,
вечеринками и партиями в бридж.

Но к концу учебного года что-то изменилось.
Я уже не помню точно, когда это произошло, то ли, когда я узнал об одной новой
задаче, то ли несколько позже, когда я уже получше в неё вдумался.
Постепенно математика начала вытеснять всё остальное: я начинал о ней думать
утром и не замечал, как наступал вечер.
Ночью иногда тоже, вместо того, чтобы спать, я думал о своей задаче.
Но всё-таки ночных размышлений я старался по возможности избегать, потому что
понимал, что несколько ночей без сна я не выдержу, так что даже для самих
занятий математикой будет намного лучше, если я ночью не буду заниматься, но
избежать этого соблазна было не так-то просто.

На лето я поехал в Афины.
Встречи со старыми друзьями, тёплое греческое море заставили меня немного
отвлечься от математики и вернули меня на время к старому жизненному ритму.
Но это не значит, что летом я забыл про свою задачу.
Я по"=прежнему о ней думал "--- иногда в гостях у своих родителей, иногда на
морском берегу, иногда ночью в постели перед тем как заснуть "--- но всё-таки
думал о ней менее напряжённо и сосредоточенно, чем раньше.

В конце августа я вернулся в Корнелл, и у меня снова появилось больше времени
для занятий математикой.
И к концу сентября мне впервые стало казаться, что задача наконец сдвинулась с
мёртвой точки.
Примерно в это же время я сделал несколько странное наблюдение "--- задумавшись
о математике, я иногда не мог сказать, как течёт время: думал ли я в течение
получаса или нескольких часов?
И, теряя счёт времени и только к вечеру отвлекаясь от своих размышлений, я
обычно не мог вспомнить, о чём именно я думал весь день, я только чувствовал,
что очень устал, и у меня было очень странное ощущение, будто бы я вовсе не
думал до этого о математике, а занимался тяжёлой физической работой.
Образ, который мне приходил в голову, был почему"=то сбором риса.
Стоило мне один раз подумать об этом сравнении, как потом мне даже казалось, что
после занятий у меня остаются несколько затуманенные воспоминания: как весь день,
стоя по колено в воде я собираю рис, с утра до самого вечера делаю однообразные
монотонные движения, и палящее, висящее над рисовым полем солнце делает эту
скучную и малоинтересную работу ещё более тяжёлой.

После таких дней я начинал сомневаться в правильности избранного мной пути.
И всё-таки у меня было ощущение, что мои размышления куда-то двигаются.
Может быть, я ошибался в том, что я сделал до этого некоторый прогресс?
Если ответ был не таким, как я на это надеялся, то от доказанного мной недавно
утверждения в конечном счёте могло не быть никакого толка.
Но я не сдавался, сам не знаю уже почему "--- надеялся ли я на победу, была ли
тема мне настолько интересной, что я был не в силах от неё оторваться, или,
может быть, просто из упрямства?

И, одним октябрьским вечером, борьба моя с моей задачей наконец завершилась:
ответ оказался именно таким, как я всё это время предполагал, и мне удалось это
доказать!

Ночью я с трудом заснул, меня уже начинали мучить сомнения "--- не допустил ли
я в своём аргументе какой"=либо ошибки?
В моём доказательстве было пару довольно тонких мест, где запросто можно было
серьёзно провраться.
Я понимал, что в этих двух сомнительных леммах нельзя быть до конца уверенным,
пока не запишешь их доказательства, а записывать доказательства я очень не любил.

У меня был не очень большой, правда, опыт с моей дипломной работой:
мне гораздо больше понравилось решать задачу, чем писать о ней статью.
Собственно, это даже не настоящая статья была, а коротенькая заметка для
\enquote{\foreignlanguage{french}{Comptes rendus}}
французской академии наук, но написание этих четырёх"--~пяти страничек маленькой
заметки отняло у меня немало времени и не доставило никакого удовольствия.
И всё же я решил: прямо завтра утром сяду записывать своё новое доказательство,
и после принятия этого решения я смог наконец заснуть.

Но утром, прежде чем садиться записывать моё доказательство, я решил его ещё раз
хорошенько обдумать.
Это обдумывание заняло в результате несколько дней, и только в начале следующей
недели я начал его записывать.
Я настолько беспокоился о возможной ошибке, что даже не хотел в начале говорить
о своём результате научному руководителю прежде, чем всё сумею записать.
Но в конце концов я всё-таки решил, что рассказать надо:
тем более что многие очень существенно используемые мной идеи я узнал именно от
него, было бы нехорошо после этого что-либо от него скрывать.

Мой шеф вначале несколько недоверчиво отнёсся к доказанной мной теореме, но
после того как я рассказал ему все основные этапы доказательства, кажется, стал
менее скептичным.

"--* И всё-таки, "--- сказал он, "--- чтобы до конца быть уверенным я хотел
бы увидеть всё в письменном виде, особенно доказательство леммы о локализации.

"--* И поторопитесь с написанием Вашего препринта, "--- добавил он.
"--- Я знаю, что у профессора Варанга есть студент, который занимается очень
близкой тематикой.
Как только статья будет готова, надо будет сразу же послать её Варангу.

Но меня не надо было поторапливать:
уже через пару недель предварительная версия моей статьи была закончена.
Правда, это была действительна очень предварительная версия, ещё очень далёкая
от той, которую можно было бы послать на рассмотрение в журнал.
Но мой шеф теперь уже полностью согласился с правильностью моего аргумента, и
даже сказал, что внеся в препринт несколько исправлений, его можно будет послать
нескольким специалистам, в том числе профессору Варангу.

Перед тем, как рассылать свою статью, я снова начал колебаться.
Нет ли подвоха в доказательстве самой первой леммы?
Я настолько был в ней с самого начала уверен, что последнее время о ней совсем
не думал, и доказательство записал довольно небрежно.
Но нет, кажется с первой леммой всё-таки особых проблем не было.
Потом я стал нервничать насчёт применения теоремы Мишры:
я никогда сам до конца не разбирался в её доказательстве, если она окажется
неверна, то весь мой аргумент полностью провалится.
Конечно, теорема Мишры была результатом, который стал уже почти что классическим,
но, самому не проверив, ни в чём ведь нельзя быть до конца уверенным!

Но, не без помощи моего научного руководителя, по поводу теоремы Мишры в конце
концов я тоже более или менее перестал беспокоиться, после чего, наконец,
разослал свой препринт.
Неплохо было бы также положить его на мою электронную страницу, но от этого я
пока что воздержался.

Больше всего меня интересовала возможная реакция профессора Варанга.
Первые дни он ничего не отвечал.
Это было довольно естественным, он был человеком занятым, наивно было бы ожидать,
что, получив мой препринт, он сразу бросится его читать.
И всё же, проверяя по несколько раз в день электронную почту, я с грустью
говорил себе, что он мог бы хотя подтвердить получение моей статьи.
\enquote{Дорогой Константин, спасибо за Вашу статью.
С уважением, \ldots} "--- и дальше подпись.
Казалось бы, даже у занятого человека такой ответ не отнял бы очень много
времени, а мне бы уже было веселее.
Я говорил себе, что возможно он никогда мне не ответит, такое часто случается,
и что очень глупо так вот каждый день надеяться на его возможный ответ.
Но всё-таки уже через неделю он мне написал, и это было не просто формальное
подтверждение о получении моего электронного письма.

Открыв его послание, я подумал, что у него уже есть комментарии или замечания по
поводу моей статьи "--- то, что он писал, было заметно длиннее, чем пара
строчек.
Но оказалось, что мой текст, как это и следовало ожидать, он ещё не начал читать,
только посмотрел на основную формулировку, и она его заинтересовала.
Зато он предлагал мне приехать в Институт высших научных исследований, в котором
он работал, и поговорить с ним лично.
На такое я до этого совершенно не мог рассчитывать!
И причём не когда"=нибудь приехать, а прямо сейчас!
\enquote{Лучше всего, "--- писал он, "--- если Вы сможете приехать до конца
декабря, потому что в этом году у нас в институте есть специальная программа для
молодых учёных, проще всего мне Вас пригласить в её рамках, но если не получится
приехать в декабре, приезжайте в январе или в феврале (потом, начиная с марта,
я буду в течение двух с половиной месяцев отсутствовать)}.

Конечно, я тут же с огромной радостью принял его приглашение.
Мне было ужасно интересно с ним лично встретиться, к тому же, я от многих слышал
о том, что институт высших научных исследований был очень интересным местом, а
я до этого никогда раньше в нём не был.
Я договорился о переносе своих занятий (в этом семестре я вёл практические
занятия по анализу для нескольких групп первокурсников), купил билет на самолёт,
но за несколько дней до отлёта заболел.
Откладывать поездку ужасно не хотелось, но с температурой под сорок я всё-таки
не в силах был куда-либо ехать.
К тому же, даже когда температуру удавалось сбить, я чувствовал необычайную
слабость, такую, что даже ходить по квартире я мог только с заметным трудом.

Я написал Варангу и секретарше, которая в институте высших научных исследований
занималась визитёрами, о том что из-за гриппа мне приходится приехать позже, и
перенёс свой билет на самолёт на неделю, но через неделю я по"=прежнему
чувствовал себя очень плохо.
Я понял, что придётся сходить к врачу, хотя вообще"=то ходить к врачу по поводу
простуды или гриппа мне казалось обычно довольно бессмысленным.
Но в данный момент мне чрезвычайно хотелось поскорее поправиться, к тому же
узнавшая о моей болезни мама ужасно волновалась, звонила мне каждый день и
совершенно негодовала по поводу того, что я до сих пор ещё не был у врача.

Врач назначил мне какие"=то анализы, а потом, глядя на их результаты, долго
задумчиво качал головой, прослушивал мне внимательно сердце, хотя никаких
проблем с сердцем у меня до этого не наблюдалось, и задавал немного странные
вопросы:

"--* Вы не пьёте регулярно и в больших количествах алкоголь?

"--* Чем вы обычно питаетесь "--- я понимаю, конечно, большинство студентов питается
не очень полноценно и чем придётся, но ведь хотя бы когда"=нибудь вы едите овощи,
фрукты, хлеб или мясо?

"--* Вы пьёте когда"=нибудь молоко?

"--* Едите ли Вы что-либо кроме очищенного риса?

Я очень удивился и немного встревожился.
Питался я не слишком регулярно, но ведь действительно почти все студенты так
делали.
Алкоголь последние пару месяцев я, кажется, вообще совсем не пил.
Раньше, когда я вёл значительно более весёлый образ жизни, бывали случаи, когда
я изрядно напивался на какой"=нибудь вечеринке, но, пожалуй, даже и про тот
прошедший период было бы преувеличением сказать, что алкоголь я употреблял
регулярно и в очень больших количествах.

"--* Значит, у меня не просто грипп? "--- спросил я у врача.
"--- Что-нибудь серьёзное?

"--* Сейчас у вас грипп, "--- ответил врач.
"--- Но не это меня беспокоит.
У вас довольно сильный авитаминоз, из-за этого, возможно, Вы довольно медленно
сейчас поправляетесь.
У Вас присутствуют симптомы Бери"--~Бери, и это меня достаточно удивляет.
Считается, что в Соединённых Штатах этой болезнью болеют только грудные дети,
матери которых страдают недостатком витамина B, или хронические алкоголики.
В Азии, кажется, ситуация иная, и до сих пор довольно много случаев Бери"--~Бери,
особенно среди очень бедных слоев населения, питающихся одним только очищенным
белым рисом \ldots

В результате врач назначил мне несколько уколов тиамина, после которых мне
следовало есть витамины уже в таблетках, посоветовал как можно более полноценно
питаться и не слишком переутомляться с занятиями математикой, и велел мне
приходить снова, если температура по"=прежнему будет оставаться высокой.
Не знаю, помогли ли мне уколы или просто мой грипп наконец закончился, но дней
через десять я был наконец совершенно здоров.
Был уже самый конец декабря, Новый год совсем скоро, но я и так уже несколько
раз откладывал свою поездку, больше ждать мне совсем совершенно уже не хотелось.
Поэтому получилось так, что вылетел я в итоге в самом конце декабря, и Новый год
мне таким образом предстояло встретить не дома, а в институте высших научных
исследований.

Перед отъездом я распечатал с интернет"=страницы Института высших научных
исследований указания о том, как до него добираться, но внимательно прочёл их
только в самолёте.
Прочитав, я забеспокоился, не оставил ли я часть инструкции дома.
Я не раз до этого слышал о том, что институт находится не в самом городе, а в
лесу неподалёку от него, указания же оканчивались словами: сесть на 25-ую линию
метро в направлении Юг"--~2 и доехать до станции ИВНИ.
Что делать после этого, не было сказано.
Трудно представить, что метро могло вести прямо в лес, значит наверняка после
этого надо пересесть на какой"=нибудь автобус?
Я не сразу догадался, что ИВНИ означает институт высших научных исследований,
но потом догадался наконец и решил, что возможно никакой автобус мне всё-таки не
нужен.
Ведь если бы станция метро находилась бы достаточно далеко от института, вряд ли
бы её назвали в его честь.

Прилетев, я довольно быстро получил багаж, вышел из аэропорта и сел в метро.

Поезд, на котором я ехал, на некоторое время вынырнул на поверхность, но вскоре
снова ушёл под землю, так что я не мог видеть, где именно он едет.
Я разглядывал названия проезжаемых станций, вначале они были мало о чём
говорящими, но потом стали попадаться такие как \enquote{Южная роща},
\enquote{Лесной ручей}, и это меня несколько ободрило.
Может, метро и правда довезёт меня до самого леса, в котором находился институт?

Мои ожидания оправдались, выйдя на станции ИВНИ и поднявшись по эскалатору я
оказался на дороге, по одну сторону от которой находилась небольшая деревенька,
а по другую сторону "--- хвойный лес, деревья которого я никогда раньше не
видел, какие"=то полусосны"--~полуёлки.
Неподалёку от меня начиналась довольно широкая тропинка, она уходила в лес, и
через некоторое время на ней виднелся указатель.
Одна из его стрелок была направлена в мою сторону, на ней был нарисована большая
буква M и написано в кавычках ИВНИ, что всё вместе наверняка должно было
означать: станция метро \enquote{ИВНИ}.
На другой, смотрящей в противоположную сторону стрелке было просто написано
ИВНИ, уже без буквы M и без кавычек.
Я пошёл по тропинке, и минут через пятнадцать ходьбы оказался перед институтом.

Он находился в большом парке, огороженном невысокой каменной оградой.
Деревья в парке росли не так густо, как в лесу, через который я только что
прошёл.
Часть из них была всё теми же полусоснами"--~полуёлками, но попадались также и
другие деревья, многие из которых тоже были хвойными.
Так что почти все они были несмотря на зиму зелёными, не считая нескольких рябин
и довольно большого, поднимающегося по склону холма вправо от моей дорожки
кустарника, не то это был малинник, не то кусты ежевики, точно я не мог сказать.
Насчёт рябин я тоже не был уверен, потому что до этого видел их только на
картинках, но на некоторых из них висели грозди красноватых маленьких ягод,
именно так по моим представлениям должны были выглядеть зимой эти деревья.

Через большие ворота, которые были настежь распахнуты, я вошёл на территорию
института и по петляющей между ёлок и сосен тропинке добрался до главного
научного здания.

В институте меня ждало некоторое разочарование.
Оказалось, что сейчас, перед Новым годом, людей в нём гораздо меньше, чем
обычно.
Мне удалось, правда, зарегистрироваться и получить ключ от моего кабинета, но
даже записаться в библиотеку уже было нельзя до начала января.
Что касается библиотеки, впрочем, мне объяснили, что она всё время открыта, так
что в неё всегда можно заходить и читать то, что ты хочешь.
Только вот книги из неё нельзя выносить, пока не записался.

Ещё мне сказали, что профессора Варанга сегодня нет, и что он появится теперь
уже только после Нового года.
И что многие другие профессора и визитёры тоже сегодня отсутствуют.
Вообще"=то, это я мог предвидеть заранее "--- 30~декабря у нас в университете
людей было ещё намного меньше, чем тут в институте.
Но узнав, что Варанга я смогу увидеть только через несколько дней, я довольно
сильно расстроился.

Правда, я уже заметил, где находится его кабинет "--- он был на первом этаже,
недалеко от входа.
Я вышел в административное здание, чтобы оформить какие"=то бумажки связанные с
моим приездом, и на обратном пути не удержался и заглянул в окно кабинета
Варанга.
Окно было большим и находилось не слишком высоко, поэтому кабинет можно было
неплохо разглядеть:
мне удалось увидеть несколько заставленных книгам и оттисками книжных полок,
нахлобученную на вешалку для пальто чёрную мужскую шляпу, и ещё повешенную над
письменным столом фотографию белой дикой лошади.
В остальном ничего особенно примечательного.

Я был довольно голодным и решил узнать, где здесь поблизости можно поесть.
Оказалось, что каждый день в институте бывает общественный обед.
Это объяснила мне одна из секретарш.
Она также показала мне столовую, это было небольшое одноэтажное здание,
находящееся в пятистах метрах от главного научного здания, и выглядевшее
заметно иначе, наверное, потому что было гораздо более старым.

За несколько минут до обеденного времени, я увидел как многие профессора и
посетители института стали двигаться по дорожке в сторону столовой, и тоже к ним
присоединился.
Я почти ни с кем ещё не успел познакомиться, но про некоторых людей догадывался,
кем они были.
Вот, например, тот высокий тощий человек, с немного развевающимися по ветру
растрёпанными волосами и блистающими огненными глазами, наверняка это знаменитый
профессор Фифтифулз.
Я никогда его раньше не видел, но зато не раз видел его фотографии, один раз про
него, кажется, даже была статью в \enquote{Нью-Йорк Таймс}.
Я прислушался к разговору, который он вёл со своими спутниками "--- да, да,
наверняка это был Фифтифулз.
Раздел математики, которым я занимался, был весьма далёк от области, которой
занимался, или, вернее, которую основал Фифтифулз, и всё же мне было очень
интересно послушать то, о чём он говорил.

Недалеко от входа в столовую я увидел двух человек, которые шли нам навстречу,
и услышал обрывок их разговора.
Один из них, помоложе, крепкого сложения и с несколько жёстким выражением лица,
говорил другому, невысокому и довольно толстому:

"--* Вы, я смотрю, как и я не ходите на общественный ланч?

"--* Да, не хожу, уж больно не по мне эта еда, "--- отвечал его собеседник.
"--- На мой взгляд, было бы неплохо сменить повара в столовой.
Хотя многим, я слышал, наоборот очень нравятся институтские обеды.
Я пытался иногда на них ходить, ведь Вы знаете, именно за ланчем часто
обсуждаются весьма важные планы и новости.
Но каждый раз, когда я здесь поем, у меня так после этого болит живот, что я
отказался от этих попыток.
Просто съедаю по утрам более плотный завтрак, чтобы потом без ланча как-нибудь
дотянуть до ужина.

Обед в институте очень сильно отличался от еды в обычной университетской
столовке.
Для начала, все сидели за длиннющим столом, покрытом клетчатой скатертью и на
колени полагалось класть клетчатую матерчатую салфетку "--- уже всё это в
академическом мире было необычной роскошью.
Меню тоже было гораздо интереснее, чем это бывает в университетах:
на закуску "--- салат из маринованной ежевики, потом "--- курица, запечённая с
яблоками и крупный тёмного цвета рис.
\enquote{Неочищенный}, "--- отметил я про себя.
Хотя история о том, как я возможно болел Бери"--~Бери, была довольно странной, хотя
в любом случае никогда я не питался одним только рисом и, наконец, хотя где-то
в интернете я нашёл недавнюю информацию о том, что Бери"--~Бери вообще, вопреки
стандартному мнению, не связана с авитаминозом, у меня всё-таки навсегда теперь
возникло сильное предубеждение против белого, лишённого витамина B риса.

Каким бы вкусным не был сегодняшний обед, я едва успевал его есть, потому что со
всех сторон от меня шли очень оживлённые математические беседы, я поворачивал
голову то налево, то направо и с вниманием их слушал, не решаясь пока что к
какой"=либо из них присоединиться.

Недалеко от меня сидел профессор Раджафер.
Его я вообще никогда раньше не видел, даже на фотографиях.
Тёмные глаза, высокий лоб, коротко постриженные густые чёрные волосы.
Мне трудно было сказать, таким ли себе его представлял до моего приезда в
институт.
Раджафер был весьма крупным специалистом в области, которой я занимался, не
таким прославившимся, как молодой профессор Варанг, к которому я сейчас приехал,
но всё же тоже весьма известным и серьёзным.
Меня достаточно сильно интересовало то, чем он занимается, хотя я точно не знал,
чем именно он занимался в самое последнее время.
Узнать это было непросто, потому что у него была не так уж часто встречающаяся в
математическом мире странность "--- он почти никогда не записывал свои
результаты.
Менее известные математики этого вообще, конечно, не могли себе позволить, по
крайней мере, пока не получат постоянную работу:
не будешь писать достаточного количества статей, работы тебе, понятное дело, не
видать.
Но даже и более крупные учёные редко так поступали.
Раджаферу отчасти везло, многие из его неопубликованных результатов потом
всё-таки записывались другими математиками, и "--- особое везение "--- даже
его авторство при этом часто не забывалось.
Но всё же, как это ни было обидно, думаю, некоторые из его теорем так и
оставались навсегда непонятыми и незаписанными.

Ещё до приезда в институт я был наслышан о том, что Раджафер был вообще большим
чудаком, но сейчас, глядя на него, мне показалось, что часть этих рассказов была
несколько преувеличена.

У него действительно была некоторая чудаковатость "--- он всё время был
несколько задумчивым и отрешённым, и ещё он говорил с заметным акцентом и у него
была медлительная и певучая речь "--- но, что касается этого последнего
свойства, оно просто было данью той стране, в которой он когда-то родился.

То, что он говорил, и про математику, и не про математику, часто казалось мне
достаточно разумным, это скорее окружающие реагировали так, будто бы то, что он
сказал, всё время было очень смешным, и от этого он действительно иногда казался
немного странным.

Например, после основной еды подали сыр, и не какой"=нибудь, а французский,
нескольких разных сортов, большинство из которых я никогда раньше не пробовал.
Один из сыров назывался ми-шевр.

"--* Что значит шевр? "--- как всегда немного задумчиво спросил Раджафер у своих
соседей по столу, и какой"=то французский молодой человек ответил:

"--* Шевр по"=французски означает козу.

"--* А правда, что сыр этот называется ми-шевр потому, "--- продолжил своим
низким, растягивающим каждое слово голосом Раджафер, "--- что делают его
наполовину из козьего, наполовину из коровьего молока?

Ещё прежде, чем он закончил свой вопрос, вокруг него раздался дружный громкий
смех.

"--* Раджафер тут спрашивает, "--- объяснил сидящим дальше и не слышавшим
предыдущего разговора людям профессор Франклин, "--- Раджафер тут спрашивает,
правда ли сыр ми-шевр так называется потому, что его делают из молока животного,
которое наполовину коза, наполовину корова?

И дальний конец стола тоже присоединился к всеобщему смеху.

К концу еды разговоры за столом стали менее математическими и более светскими.
Сидящий справа от меня профессор поинтересовался, из какой я страны и чем я
занимаюсь.

"--* Сейчас я приехал из Корнелла, но вообще я из Греции и учился до этого в
Афинах \ldots

"--* Из Афин? "--- воскликнула сидящая неподалёку от нас молодая женщина.
"--- Прошлым летом мы были в Греции, и мне там ужасно понравилось.
В Афинах мне очень понравился акрополь, и особенно кариатиды на храме Эрехтейон.

"--* Кариатиды "--- это как атланты, но только женщины? "--- уточнил сидящий
напротив неё молодой японский математик.

"--* Атланты "--- это ведь люди, держащие здание на своих плечах, "---
задумчиво подхватил его вопрос Раджафер.
"--- А как они называются, если, наоборот, они стоят на плечах, а свод здания
подпирают своими пятками?

Тут снова все засмеялись, и, как должен был я это признать, на этот раз не без
основания.

После сыра был кофе и пирожки с рябиновым вареньем.
Подливая себе в чашку с кофе немного молока из фарфорового кувшина, на котором
было нарисовано множество крошечных разноцветных бабочек, молчавший до этого
Алан Фифтифулз произнёс:

"--* Недавно я где-то слышал, что есть такая порода бабочек:
всю жизнь они летят от самого южного края Южной Америки до Аляски, и потом
обратно: через всю Северную Америку обратно в сторону Южной.
В течение пути в один конец сменяется три поколения бабочек.
Это значит, что у бабочек есть генетическая память, говорящая им, в какую
сторону надо лететь.
Удивительно, как можно это помнить:
до цели предстоит долететь не тебе, а только твоим детям или внукам, но ты точно
знаешь, в какую сторону надо лететь, а твои внуки, в свою очередь, долетев до
самого края Северной или Южной Америки, будут точно знать, что пора повернуть,
чтобы продолжить это бесконечное путешествие над обоими американскими
континентами \ldots

Через окна столовой было видно, как на улице падает снег.

Во второй половине дня, ближе к вечеру, я решил сходить в библиотеку.
То, что она всегда была открыта, оказалось правдой.
Дверь не только была не заперта, а просто распахнута настежь, и я поэтому без
труда попал в первый библиотечный зал.
В этом первом помещении находились журналы.
Я был несколько разочарован их выбором "--- некоторых неплохих журналов,
которые обычно бывали в университетских библиотеках, здесь не было.
В глубине зала я увидел ведущую вниз лестницу.
Возможно, некоторые журналы находятся в другом помещении?
Но нет, спустившись по лестнице я попал в зал, в котором находились уже не
журналы, а книги.

Так что бюллетень американского математического общества, в котором недавно был
опубликован некий небезынтересный для меня обзор, мне так и не удалось найти.
Я вспомнил, что секретарша объясняла мне, что я так же имею право записаться в
библиотеку университета, который находился не очень далеко от института.
\enquote{Можно будет сходить в неё после Нового года}, "--- сказал я сам себе и
принялся разглядывать книги.

Выбор книг тоже был небольшим, но среди них попадались некоторые интересные и
довольно редкие издания.
Я полистал несколько книжек на русском языке "--- читать по"=русски я не умел,
но меня всегда радовала похожесть русских букв на родные для меня греческие
буквы.
К тому же, рассматривая одну из русских книг по геометрии я был вознаграждён за
своё любопытство, потому что в конце неё оказалась подборка очень забавных
математических картинок.

Из помещения с книгами можно было ещё дальше вниз спуститься по лестнице, что я
и сделал.
Я оказался в третьем, возможно несколько подсобном библиотечном помещении.
Отчасти на полках, отчасти запакованные в коробки в нем находились очень старые
журналы:
те же наименования, что я уже видел в зале наверху, но только выпуски
пятидесятых или шестидесятых годов, а также некоторые книги.
В отличии от предыдущего книжного зала книги здесь были расставлены без всякого
видимого порядка: во всяком случае он не был ни алфавитным, ни тематическим, а
некоторые вообще лежали кое-как неровными стопками на небольших столиках или
ящиках с журналами.
Я заметил, что на многих из них отсутствовали карманчики для читательских
формуляров.
Но всё равно я ещё не был записан в библиотеку, так что книги домой брать не мог,
и поэтому для меня это не имело большого значения.

Я перебирал старые журналы, уже не из математического, а чисто исторического
интереса.
В одном из ящиков, содержащем очень старые выпуски журнала Крелля я наткнулся на
толстенный препринт.
Вначале я подумал, что это книжка, потому что он был слегка переплетён, иначе бы
его было трудно взять в руки, так как в нем было около тысячи страниц.
Но открыв его, я увидел, что это был препринт какой"=то статьи профессора
Раджафера.
Судя по состоянию бумаги он был достаточно старым, и напечатан был не на
компьютере, а на пишущей машинке.
Но насколько мне было известно эта статья, называвшаяся \enquote{О нулях
некоторых дзета"=функций} так никогда и не была опубликована.
Конечно, в печатной версии название могло бы измениться, но я был уверен, что у
Раджафера не было напечатанных статей такой большой длины.

\enquote{Странно, что я даже никогда не слышал об этой его работе}, "--- сказал
я сам себе, и принялся за чтение препринта. Вообще"=то мне не стоило
слишком долго задерживаться в библиотеке и уже наверное было пора
пойти искать институтский городок, я так и не был ещё сегодня в
квартире, в которой мне предстояло жить. Но препринт Раджафера был
действительно очень интересным, так что я решил полчасика его почитать,
и только потом, забрав из кабинета до сих пор хранящийся там багаж,
пойти искать место моего будущего обитания.
Статья Раджафера оказалась чрезвычайно содержательной, но
довольно сложной для понимания. Первые страницы я прочёл внимательно,
вникая в детали всего происходящего, но достаточно быстро смысл её начал
от меня ускользать, и вскоре я уже не читал, а просто листал страницу за
страницей, уже почти не глядя на доказательства, стараясь понять только
отдельные предложения или замечания.

Странно, что я даже никогда не слышал об этой его работе, "--- сказал я сам
себе, и принялся за чтение препринта.
Вообще"=то мне не стоило слишком долго задерживаться в библиотеке и уже наверное
было пора пойти искать институтский городок, я так и не был ещё сегодня в
квартире, в которой мне предстояло жить.
Но препринт Раджафера был действительно очень интересным, так что я решил
полчасика его почитать, и только потом, забрав из кабинета до сих пор хранящийся
там багаж, пойти искать место моего будущего обитания.

Статья Раджафера оказалась чрезвычайно содержательной, но довольно сложной для
понимания.
Первые страницы я прочёл внимательно, вникая в детали всего происходящего, но
достаточно быстро смысл её начал от меня ускользать, и вскоре я уже не читал, а
просто листал страницу за страницей, уже почти не глядя на доказательства,
стараясь понять только отдельные предложения или замечания.

Иногда я рассматривал формулы, и хотя уже плохо их понимал, у меня было ощущение,
что при взгляде на них у меня в голове что-то начинало шевелиться.
Потом, завтра или после Нового года, я постараюсь прочесть статью более
внимательно "--- хотя не так то просто прочесть хоть сколько"=нибудь внимательно
статью длинною в тысячу страниц!
Может, когда я уже более серьёзно буду читать эту статью, мне поможет
сегодняшнее, первое и ещё малоосмысленное от неё впечатление\ldots

Скользя глазами по формулам я заметил некоторую особенность обозначений,
выбранных Раджафером: он очень часто использовал греческие буквы.
Так как статья была про дзета"=функции, совершенно естественно, что в ней было
полно букв $\zeta$.
Но, к тому же, не только буква $\pi$, как всегда, обозначала половину длину
окружности единичного радиуса, не только $\epsilon$ часто использовалась для
обозначения различных малых величин, не только не редко встречающиеся в
математике $\alpha$ и $\beta$ тут и там употреблялись во всевозможных контекстах,
но в статье было полно и других, реже встречающихся в математических текстах
букв, казалось, весь греческий алфавит был в ней задействован!
Вот, например, в следующей формуле почти все символы были греческими:
\begin{equation*}
\prod_j\left(\alpha_j - \lambda\tau_o\right).
\end{equation*}

Я уже плохо помнил что именно было обозначено при помощи буквы $\lambda$, но
зато, выкинув из формулы инородную латинскую $j$, я обнаружил, что оставшиеся
символы складываются в написанное с заглавной буквы греческое слово
$\pi\alpha\lambda\tau o$.
\enquote{Смешно, "--- подумал я, "--- не получится ли что-нибудь подобное с
какой"=нибудь другой формулой?}
Во многих формулах было слишком много латинский символов, но вот страницей
раньше от встретившегося мне слова \enquote{пальто} я обнаружил ещё одну длинную
формулу, где все обозначения, кроме стандартного обозначения для функции
логарифма, были греческими:
\begin{equation*}
\log(\alpha - \pi_o) + \beta_\alpha(\theta - \rho_\alpha).
\end{equation*}
Такое длинное слово $\alpha\pi o\beta\alpha\theta\rho\alpha$ уже точно не могло
быть просто случайностью.
\enquote{Аповатра} по"=гречески означало \enquote{набережная}, откуда могло
взяться это слово в формуле этого препринта про нули дзета"=функций?
Я стал разглядывать другие формулы, во многих из них не все обозначения были
греческими, но я обнаружил, что если из формул выкинуть все негреческие символы,
то оставшиеся буквы не просто складываются в слова, а даже в слитный текст!
Вот что я сумел в итоге прочитать.
