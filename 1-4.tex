\section{Письмо}

Последний разговор с Альсбетой казался Гвенаэлю плохо запомнившимся дурным сном.
Он не мог сказать, о чём именно они говорили.
Кажется, он всё-таки не проболтался о своей встрече со старшим Филипоном, но,
нервничая из-за боязни проболтаться, наговорил кучу всякой ерунды.
Он помнил, что Альсбета под конец не на шутку рассердилась, а рассердить её было
не так уж просто.
Он пытался в ответ тоже обидеться и разозлиться, но у него это плохо получалось.
Выплывавшие из памяти обрывки их разговора всё больше и больше убеждали его в
том, что во всём виноват был именно он сам.

В общем, он почувствовал, что ужасно хочет помириться с принцессой.
Что для этого сделать?
Надо, конечно, перед ней извиниться, но он очень плохо умел это делать.
Характер у него был гордый и несдержанный, и, начав извиняться, он мог вспылить
и наговорить ещё больше нехороших глупостей.

Ему приходит в голову, что написать письмо проще, чем сказать лично всё то, что
он хочет сказать.
Он несколько раз садится за стол, начинает писать, но потом всё зачёркивает и
выбрасывает, ходит большими шагами взад и вперёд по комнате, и снова садится
писать.

В какой-то момент он даже думает "--- хотя обычно он никогда не разрешает себе
об этом думать "--- что, может быть, правы те, кто, критикуя его романы, говорят,
что в них много рассудительности и не хватает непосредственности.
И язык, всегда безупречный грамматически, иногда кажется не достаточно живым.

Чтобы отвлечься от этих мыслей, он выходит из дома.
Уже совсем темно, и на улице не видно прохожих.
Обычно в это время можно встретить прогуливающиеся влюблённые пары, но сейчас и
их не видать.
Последние месяцы, после прихода к власти императора R.~ле~Кина, город стал
мрачнее, и горожане скорее спешат после работы домой, и менее охотно выходят
пройтись по вечерам.

Небо довольно ясное, и на севере хорошо видно Белую гору.
Называется она так по тому, что животные и птицы, живущие на ней, белые или во
всяком случае очень светлые.
И даже когда смотришь издалека на пасущиеся на ней стада белых коз или стаи
светло"=золотистых огромных собак, то и сама гора кажется белой.
Глядя на гору, Гвенаэль думал о том, что где-то там, в одной из горных деревушек
живёт его троюродная тётушка.
И что бывший премьер"=министр Конфуций, ушедший в отставку при формировании
нового правительства, тоже оттуда родом.
Теперь его место в правительстве занимает министр Пьерро, который раньше был
министром образования.
Но он-то исконный горожанин и к Белой горе никакого отношения не имеет.

Самой тёмной и пустынной улицей из тех, по которым проходил Гвенаэль, была,
пожалуй, Большая Торговая улица.
На ней ведь не было почти жилых домов, поэтому даже светящегося окошка сейчас не
увидишь.
Даже нарядные витрины магазинов было не разглядеть: одни на ночь были закрыты
ставнями, другие просто были слишком плохо освещены.
Только вот старинные часы в окошке часовой лавки было неплохо видно, потому что
на них падал свет от стоящего на перекрёстке фонаря и ещё, наверное, потому, что
и цифры, расставленные кругом по желтоватому циферблату, и одинокая часовая
стрелка этих часов были покрыты какой"=то фосфоресцирующей краской.
I, II, III и IIII были зеленоватыми, и IIII, записанная почему-то как IIII, а не
как IV, занимала больше всего места и казалась от этого особенно яркой.
Потом шли более блеклые V и VI, причём краска на палочке у VI была полустёрта,
и от этого шестёрка была похожа на предшествующую ей пятёрку.
VII, IIX и IX (восемь часов опять же были обозначены как IIX, а не как VIII)
тоже были довольно блеклыми и сероватыми, а X, XI и XII, хотя и менее яркие, чем
I, II и III, снова отсвечивали чем-то зеленоватым.

Гвенаэль немного помедлил на углу Большой Торговой улицы, потом пошёл дальше и
свернул в один из отходящих неподалёку переулков.
Он шёл без видимой цели, и, как это часто с ним бывало во время прогулок, очень
скоро он вышел на Безымянную улицу.
Вот уже и знакомый дом с деревянным атлантом вверх тормашками.
Проходя мимо него, у Гвенаэлю сжалось сердце: может быть, бросить эту затею с
письмом, зайти и сказать самому всё то, что ему так хочется ей сказать?
Но свет в доме Альсбеты уже погашен, значит она спит и заходить уже поздно, и
Гвенаэль продолжает идти по тёмному городу.

Вот Кошачий переулок.
Здесь он тоже уже был сегодня утром, когда в нём было немало прохожих, но теперь
ночью никого не видно, и поэтому особенно грустно от того, что больше нет кошек.
Раньше они и по ночам прыгали через узкий переулок, с крыши на крышу.

В окне у Омера свет не горит, но здесь-то можно быть почти уверенным, он не
горит не потому, что Омер уже лёг спать, а, напротив, потому что он ещё не
вернулся домой.
Гвенаэль никогда не был в гостях у Омера, но он точно знает, что вот то окошко,
слева и под самой крышей, это окно его комнаты.
Ложится Омер всегда очень поздно, если бы он был у себя, то наверняка сейчас бы
писал.
И тогда в этом левом окне был бы виден его тёмный силуэт, склонившийся над
освещённым настольной лампой письменным столом.

Наверное, там, около погашенной сейчас лампы и сейчас лежит рукопись его нового
романа.
Было бы чертовски интересно узнать, о чём он сейчас пишет.
Ходили нелепые слухи, будто бы он изменил своему обычному стилю и начал недавно
сочинять шуточную пьесу в стихах, и что вообще отныне он собирается писать
исключительно силлабическим александрийским стихом, не то одиннадцати, не то
двенадцатисложником.
Это, наверняка, было полным вздором.
Омер всегда был в центре внимания, особенно женского, и про него без конца
сплетничали.
Ещё не далее, чем в прошлые выходные Гвенаэлю по огромному секрету рассказали о
том, что не исключено, что Омер "--- незаконнорождённый сын императора
R.~ле~Кина.
За неделю до этого он слышал версию о том, что, наоборот, у R.~ле~Кина есть
незаконнорождённая дочка, на которой Омер вот-вот собирается жениться.
Вариант про сына был особенно смешным, учитывая полное отсутствие какого"=либо
сходства между писателем и императором.
Рассказы про пьесу в стихах наверняка были настолько же верными, как и все
остальные, и всё же Гвенаэлю больше чем кому бы то ни было ещё хотелось узнать,
о чём будет новая книга Омера.
Дом напротив, на другой стороне кошачьего переулка, был ниже, чем дом Омера,
его крыша находилась примерно на уровне его окна.
Гвенаэлю приходит в голову забавная мысль, что кошки, гуляя по крыше этого дома,
могли бы заглядывать в окно к Омеру и подглядывать, о чём он пишет, сидя за своим
письменным столом.
В смысле, они раньше, возможно, могли подглядывать.
Теперь их и в правду нет совсем, даже здесь, где раньше их было видимо"=невидимо.

Вроде и кошек нет, и прохожих почти не видно, но от чего-то сегодняшняя ночь
полна непонятных звуков: где-то что-то стучит, где-то что-то шумит, скрипят
крыши, и что-то шелестит, и что-то бормочет, и ещё какие-то звуки, похожие на
ветер, и на вздох, и на неровное чьё-то дыхание\ldots

Гвенаэль возвращается к себе, и постепенно город становится тише.
Поднимаясь по лестнице, он старается идти осторожно, чтобы ступени не скрипели
под ногами, ведь все его соседи давно уже спят.

На следующий день с утренней почтой он отправит принцессе письмо следующего
содержания.

\medskip
\begin{center}
\textbf{Письмо Гвенаэля Альсбете}
\end{center}

\medskip
Когда я пытаюсь написать тебе или о тебе, Альсбета, я понимаю, что не умею
писать.
После того, как напишешь несколько книг, испишешь сотни, а может даже и тысячи
страниц, кажется, что ты чему-то научился.
Но в какой-то момент оказывается, что нет, это всё бесполезно, и ты так же
беспомощен перед листом чистой бумаги, как ребёнок, первый раз взявший в руку
перо.
И я уверен, что никто из читателей моих книг не узнал бы автора неловких и
беспомощных строчек этого письма.

Мне бы хотелось написать тебе, Альсбета, о том, как ты красива, но я не нахожу
нужных слов.
Может быть, это оттого, что в твоей внешности есть, кажется, только две черты,
соответствующие обычному представлению о женской красоте, и есть так много черт,
этим обыденным представлениям не соответствующих!

Две черты, которые я упомянул, это твои белоснежные волосы и голубые глаза,
такие большие и яркие, что их взгляд замечаешь, даже когда видишь тебя издалека
"--- об этом просто написать.

Но когда я пытаюсь подробнее описать твою внешность, слова становятся настолько
бессмысленными, что, если им поверить, можно подумать, что ты не так уж хороша
собой "--- можно написать, например, что у тебя вытянутое лицо, крупные губы и
большие белые уши "--- и получается чёрт-те знает что, судя по этим словам можно
подумать, что ты дурнушка, в то время как любому мужчине достаточно одного
мимолётного взгляда на тебя, чтобы заметить, как ты ослепительно красива!

А может быть оттого сложно описать твою красоту, что самое пленительное в твоём
лице, это не застывшие его черты, а его выражение, порою быстро меняющееся и
наполняющее очарованием не только глаза, не только твой рот, но и щёки, и лоб,
и нос "--- да, даже ноздри твоего носа, которые всегда еле заметно вздрагивают,
когда ты говоришь.
Есть ещё сотни чёрточек твоего лица, которые я давно знаю наизусть "--- ямочка,
которая появляется на твоей правой щеке, когда ты смеёшься, или то, как ты
прикусываешь нижнюю губу, около левого уголка твоего рта, когда ты волнуешься
или задумываешься "--- и уж конечно, я умею безошибочно узнавать твою походку.
Иногда, даже сквозь городской шум, из-за поворота улицы, я слышу лёгкие шаги
"--- стук, стук, стук "--- и я точно знаю, что это каблучки твоих туфель
касаются неровной мостовой\ldots

Я до сих пор не знаю, догадываешься ли ты сама о том, как ты красива: иногда,
когда я смотрю, с какой естественной непринуждённостью ты отвечаешь на улыбки
встречных молодых людей, я думаю, что да, ты об этом знаешь.
Но когда иной раз я вижу, как ты с застенчивой завистью смотришь на проходящих
мимо придворных красоток, или когда ты поправляешь причёску, стараясь прикрыть
волосами твои несколько большие, но, как я уже писал, очаровательные уши, когда,
наконец, в дни праздников ты неуверенно разглаживаешь складки редко надеваемого
нарядного платья "--- я снова думаю, что нет, ты не знаешь, насколько ты хороша
собой, и мне хочется объяснить тебе, что ты так красива, что никакое платье это
обстоятельство изменить не может уже, ни улучшить, ни ухудшить.

Да, мне всегда ужасно хотелось объяснить тебе, как ты красива, потому что мне
кажется, что молодых девушек сознание собственной красоты обычно делает
счастливыми, а мне так хочется, Альсбета, чтобы ты была счастлива,
очень"=очень"=очень счастлива\ldots
