\section[Приём в императорском дворце]
{Приём в императорском дворце. Посвящение Гвенаэля в придворные писатели}

Два месяца перед церемонией посвящения в придворные писатели пролетели очень
быстро, быстрее, чем ожидал Гвенаэль.
Это было славное, очень"=очень славное время.

Гвенаэлю не удавалось скрывать радость, его наполнявшую, он ходил по улицам
улыбаясь.
Люди, близкие к придворным или литературным кругам знали, в чём была причина
этого хорошего настроения (хотя официально назначение придворным писателем
должно было оставаться в тайне вплоть до самого дня церемонии посвящения, слухи
распространялись быстро и многие уже давно были в курсе).
Впрочем, люди далёкие и от литературы, и от политики не подозревали, чем была
вызвана постоянная улыбка на лице Гвенаэля, и просто так улыбались ему в ответ,
если им приходилось с ним где"=нибудь столкнуться.

Даже Альсбета, вопреки опасениям Гвенаэля, сначала приняла новость гораздо лучше,
чем он думал.

"--* В конце концов то, что это называется придворный писатель "--- это ведь
просто формальность?
На самом деле это что-то вроде приза, которого вручают самому лучшему писателю,
правда?
Знаешь, я очень за тебя рада и ужасно тобой горжусь!

Но по мере того, как церемония приближалась, настроение принцессы ухудшалось.

"--* Дело не в самом титуле, который ты получишь, "--- говорила она.
"--- И даже не в том, что для его получения тебе придётся присутствовать в
императорском дворце.
Но ведь эту разноцветную ленточку с золотой надписью, её ведь будет вручать сам
император, и после этого тебе придётся поцеловать его руку?
Мне кажется, это всё-таки нехорошо и унизительно "--- целовать руку R.~ле~Кину,
ты так не считаешь?

"--* Раньше ты говорила, что рада за меня! "--- взрывался в ответ Гвенаэль.
"--- У тебя настроение способно так быстро меняться, что никакой возможности нет
за ним уследить!
Тем более что меняется оно почти всегда без малейшего на то повода!

"--* Почему же без повода, "--- возражала Альсбета.
"--- Я же тебе сказала, раньше я не знала о том, как именно проходит церемония
посвящения, и про поцелуй императорской руки тоже не знала\ldots

"--* Кто тебе, кстати, всё это рассказал?
У меня нет никаких в сомнений в том, что это был Омер, так ведь?
Он просто обижен, что звание придворного императора присудили не ему, и, вообще,
я давно уже заметил, что он всё время пытается настроить тебя против меня.

"--* Ты сам не знаешь, что ты говоришь, "--- перебивала его Альсбета, тоже
совсем уже выходя из себя.
"--- Про церемонию я прочла в газете, а Омера я не видела уже целый месяц по
крайней мере.
И никогда он не настраивал меня против тебя!

Вот так они начинали ссориться.
Правда, каждый раз они потом спохватывались и мирились, но всё-таки
чувствовалось, что что-то постоянно стоит между ними.
Свадьбу, которая должна была состояться летом, передвинули на осень.
Впрочем, на это как будто бы были какие-то практические причины.

По-настоящему отношения между Гвенаэлем и Альсбетой испортились тогда, когда она
узнала, что среди почётных гостей на церемонии посвящения будет присутствовать
Филипон Старший.
Собственно, Гвенаэль сам об этом проболтался, и потом ужасно корил себя за это.

\enquote{Как будто бы мне самому приятно, что этот болтливый развязный толстяк
будет присутствовать на церемонии во дворце, "--- оправдываясь, говорил сам себе
Гвенаэль.
"--- Но, с другой стороны, невозможно ведь поверить, что он когда-то мог съесть
родителей Альсбеты.
Бред, какой это всё бред!
Но говорить о нём Альсбете не стоило, вот уж действительно не стоило.
Кто меня только за язык тянул!}

Впрочем, даже все эти мысли не могли полностью омрачить его радость.
Последние дни перед церемонией пролетели так быстро, что о них ни осталось ни
малейшего воспоминания в его голове.

А вот сама церемония посвящения запомнилась ему очень хорошо.

По случаю праздника вокруг дворца были расставлены вазы с цветами, а по всему
холму, на котором стоял дворец "--- кадки с пальмами и другими южными деревцами.
В здешнем климате пальмам должно было быть довольно зябко.
Но не важно, всего ведь на один день всё это, за один день не успеют замёрзнуть!

Чёрные свинки, обычно спящие перед всеми входами и выходами, в этот вечер были
куда-то убраны.
Так что все гости могли смело входить во дворец, не рискуя заразиться чёрной
свиной лихорадкой.

Все залы во дворце сверкали, это было видно даже прежде, чем в него войдёшь:
из всех окон лился свет тысячи зажжённых свечей.

Но всё же только войдя вовнутрь Гвенаэль понял, насколько ярко и празднично во
дворце.
Свет свечей отражался в позолоте стен и потолка, в золотых тяжёлых перилах
парадной лестницы, и всюду стоял какой-то странный гул, даже не понятно было,
что это "--- голоса гостей, негромкая музыка откуда-то из глубины дворца или ещё
что-то.

Гвенаэль поднялся по устеленной красной дорожкой парадной лестнице.
Все смеялись и улыбались ему в ответ, искренне так и весело смеялись несмотря на
торжественность случая, и всё же всегда немного отступали перед ним с какой-то
смущённой почтительностью.

Гвенаэлю казалось, что ему даже не нужно было переставлять ноги, какая-то сила
несла его, он как бы плыл по этой парадной лестнице, по радостным разукрашенным
залам, рассекая, как корабль, волны улыбок, шёпота и праздничного гула.

Вот Старый Императорский зал, который всегда казался намного больше, чем он был
на самом деле потому, что все стены его были из зеркал.
Вот Малый Весенний зал, который казался намного выше, чем он был на самом деле,
так как верхняя часть стен и потолок в нём были светло"=голубыми и похожими на
небо.

И вот наконец "--- Большой Тронный зал.
Он никогда в нём раньше не был, но на рисунке в одной книжке по истории искусств
однажды видел изображение уникального паркета этого зала.
Он был из наборного дерева с вкраплениями золотых прожилок, с такими странными
красивыми фигурами: что-то округлое, к нему прилипли круги поменьше, потом ещё
поменьше, и всё же это не совсем круги, иначе фигуры выглядели бы слишком
ровными, а так "--- даже не знаешь, как их описать, живыми они не были и всё же
ужасно будоражили воображение, больше, чем рисунок инея на окне в морозную ночь,
больше, чем розовые кораллы в тихой воде южных прозрачных морей.

Он сразу вспомнил это своё детское впечатление, когда он впервые увидел на
книжной странице рисунок этого дворцового паркета "--- стоило ему сейчас увидеть
только совсем маленький кусочек этой загадочной фрактальной фигуры.
Но разглядеть рисунок толком никакой возможности не было: в тронном зале было
столько гостей, он был просто битком набит. От Гвенаэля гости по-прежнему
немного отшатывались с торжественной почтительностью, но несмотря на это
образовывавшееся вокруг него свободное пространство было очень маленьким.
И только очень небольшие кусочки паркета можно было поэтому увидеть.

Но Гвенаэль вскоре забыл про странные фигуры на наборном паркете "--- было не до
того "--- сверкали свечи, в ответ им светились золотые стены, праздничный гул
становился всё громче и сладостней \ldots

И вдруг он моментально затих.
Это на королевском троне "--- большущем золотом троне с подлокотниками в виде
голов семиязычных драконов "--- внезапно появился император R.~ле~Кин.
Гвенаэль не понял, как именно это произошло, да и, пожалуй, никто в зале этого
тоже не понял.
Но вскоре это обстоятельство тоже перестало быть важным.
Потому что R.~ле~Кин начал произносить свою торжественную речь, и все
присутствовавшие замерли, внимая его словам.
Гвенаэль тоже замер, со вниманием слушая императора, но вскоре он заметил очень
странную вещь: как он ни старался, он никак не мог понять или уж во всяком
случае никак не мог запомнить то, что говорил R.~ле~Кин.
Он резко мотнул головой и зажмурился, пытаясь стряхнуть с себя оцепенение,
которое, как он думал, мешало ему сосредоточиться на столь важной и почётной для
него речи, потом быстро открыл глаза и увидел вокруг себя следующее.
Или ему только померещилось, что он всё это видит?

Зал вокруг него на мгновение стал ещё намного больше, чем он был до этого, и
выглядел как-то иначе: золотые стены стали скорее красными, толком не разберёшь,
потому что находились они теперь очень далеко от него, свечи куда-то исчезли, но
по-прежнему со всех сторон лился яркий, почти что слепящий глаза свет.
Трон отодвинулся в глубину зала "--- или теперь это был не трон? "--- пожалуй,
это был не трон, а какое-то другое сооружение, лишь немного его напоминающее
"--- и там не то на, не то за этим странным сооружением стоял человек в тёмной
одежде.
Похож ли он был на императора или нет, с такого расстояния не было никакой
возможности разглядеть, но было слышно, что он тоже произносил речь, и в этой
речи нельзя было уже просто ни слова понять, хотя звучала она громко, пугающе
громко.
Гвенаэль снова невольно зажмурился и, открыв глаза, с облегчением увидел
прежний, нормального размера тронный зал.

Речь императора была, как и раньше, малопонятной, но сейчас Гвенаэль понимал
большинство отдельных слов, которые произносил R.~ле~Кин, просто смысл как-то
ускользал от него.
Впрочем, с остальными гостями всё обстояло по всей видимости иначе.
Время от времени в зале поднимался шум, это был не тот суматошный шум, который
стоял здесь до появления на троне императора, а более слаженный, одобрительный
гул, явно означавший, что гостям очень нравится только что произнесённая фраза.
В один из моментов, когда гул стал особенно громким, Гвенаэлю показалось, что
стены тронного зала раздвигаются, становится красноватыми\ldots

Но нет, он взял в себя в руки, и спокойно обведя вокруг себя взглядом убедился в
том, что зал прежний, золотостенный, освещённый свечами и ровно такого размера,
каким он был когда Гвенаэль в него вошёл.

Император тем временем в своей речи стал говорить о Гвенаэле.
Он это понял по поворачивавшимся то и дело в его сторону головам и,
прислушавшись, стал замечать, что R.~ле~Кин неоднократно произносит его имя.
Тут ему, конечно, стало особенно любопытно узнать, что же именно он говорил,
стараясь изо всех сил слушать внимательно он уловил несколько очень лестных
эпитетов, но общий смысл фраз по"=прежнему оставался непонятным, так вплоть до
самого конца торжественной речи.

Когда смолкли последние слова императора, Гвенаэль понял, что ему следует
приблизиться к трону.
Более того, ему показалось, что что-то схватило его и с огромной силой швырнуло
к подножию императорского трона.

Впервые в жизни он увидел R.~ле~Кина с такого близкого расстояния.
Император улыбался, но при этом выражение лица у него было напряжённым, или,
может, быть просто очень официальным.
Гвенаэлю было интересно его получше разглядеть, но ему было не оторвать глаз от
прекрасной разноцветной ленточки в руках у императора.
И дело не только в том, что ему эта ленточка сулила необычайную честь "--- она
действительно была чрезвычайно красива, переливалась всеми цветами радуги, и
золото нанесённой на неё надписи тоже было необычным и потрясающе красивым,
каким-то очень светлым и глубоким.

В этот момент свет свечей огромной люстры, висящей за императорским троном,
отразился в одной из золотых букв на разноцветной ленточке, отразившись, ударил
в глаза Гвенаэлю, и после этого ему показалось, что он ослеп.

Он просто жутко испугался, разом забыл обо всём на свете "--- о церемонии,
императоре, титуле придворного писателя "--- единственной его мыслью было:
только бы не навсегда, только бы не навсегда эта слепота, только бы снова
научиться видеть.

Но всё это произошло за одну секунду, его страх был совершенно безосновательным.
Через пару мгновений он уже снова мог видеть, правда, какое"=то время всё вокруг
было мутным, нерезким, и перед глазами плыли большие чёрные пятна.
Он по"=прежнему смотрел на ленточку, но она уже не казалась ему такой
прекрасной.
Вначале он почти не разбирал цвета, наверное, всё дело было в этом.

Но вот чёрные пятна стали исчезать, всё вокруг стало приобретать снова свои
цвета, и Гвенаэль с изумлением увидел, что ленточка не была разноцветной, она
была самого что ни на есть обычного красного цвета.

Но у него не было времени чтобы хорошенько обдумать это обстоятельство, потому
что чьи-то руки обернули эту ленточку вокруг его шеи, что-то круглое стукнулось
об его грудь, и где-то в хаосе его совсем уже перепутавшихся мыслей всплыло
воспоминание о том, что сейчас, наверное, полагается поцеловать руку императору.
В этот момент кто-то протянул ему руку, но не так, как её протягивают для
поцелуя, а так, как её протягивают для рукопожатия "--- это был какой-то
незнакомый человек в тёмной одежде пожалуй, всё-таки довольно похожий на
императора.
Гвенаэль мгновенно протянул ему в ответ свою руку и почувствовал очень сильное
пожатие, настолько сильное, что у него от боли перехватило дух.
Он пришёл в себя и ничего не понимающим взглядом уставился на императора.
В зале стояла какая-то неловкая тишина, казалось, что все чего-то ждут от
Гвенаэля.

\enquote{Ах да, "--- снова пронеслось у него в голове, "--- поцелуй императорской
руки! "--- и, упав на колени, он быстро коснулся губами протянутых ему бледных
и сухих пальцев}.

Тут заиграла музыка.
Гвенаэлю показалось, что эта самая замечательная мелодия из всех, которые ему
приходилось когда-либо слышать.
Он медленно поднялся с колен и обвёл глазами зал.

Отсюда, со стороны трона, зал казался особенно красивым.
Только сейчас Гвенаэль увидел все украшения, развешанные по стенам в честь
сегодняшнего праздника.
К тому же он смог наконец разглядеть перед собой редкий рисунок наборного
паркета: вот одна округлая линия, к ней, изгибаясь, приклеиваются округлые линии
размером поменьше "--- да, это был точно тот рисунок, который он видел когда-то
в книжке.
Паркет был собран из редкого дерева, но вот эти тонкие линии, разделяющие
кусочки дерева разных пород, были золотыми.
Всё же остальное "--- стены, потолок "--- были просто из золота.
Оказалось, что свечи по залу расставлены так, что их свет, отражаясь от золотых
стен, собирался сюда, к трону.
От этого зал, если смотреть на неё с того места, где стоял Гвенаэль, казался
особенно блистательным.
Он продолжал рассматривать всё это сверкающее великолепие, но в какой-то момент,
быстро скосив глаза, успел оглядеть лацкан своего пиджака.
На нём, переливаясь всеми цветами радуги, красовалось теперь ленточка с
золочёной надписью: \enquote{Придворный писатель}.

\enquote{Надо же, как я волновался эти последние дни, из-за этих волнений мне
даже непонятно что мерещилось во время торжественной церемонии, "--- сказал сам
себе Гвенаэль.
"--- Но теперь всё, свершилось!
Больше не о чем переживать!}

Он медленно шёл по залу.
Всё ещё играла музыка, наверное, в одном из соседних помещений находился
оркестр.
Многие гости принялись тем временем танцевать, но при приближении Гвенаэля
каждая пара "--- одна за другой "--- выходила из танца, поздравляла Гвенаэля, и
снова вливалась в танцующую толпу.

Особенно лестными были поздравления писателей и литературных критиков.
Не все из них присутствовали во дворце "--- как не трудно догадаться, Омера,
например, не было среди гостей "--- но всё же Гвенаэль с огромным удовольствием
отметил, что среди поздравляющих было много людей неглупых, достаточно тонко
понимающих литературу, многие из них очень неплохо были знакомы с творчеством
Гвенаэля, и ему удалось услышать немало в высшей степени небанальных замечаний
по поводу своих последних романов.

Но особенно он был растроган беседой с одной молоденькой девушкой, которая, как
оказалась, не только прочитала все его книжки от корки до корки, но некоторые
перечитывала даже по несколько раз.

Он покраснел от радости и смущения, и тут же подумал: как жалко, что Альсбета не
пришла со мной на этот торжественный вечер!

Перед церемонией он не раз думал об отказе принцессы идти с ним во дворец в том
свете, что это может вызвать неловкость и дурацкие разговоры: почему это невеста
будущего придворного писателя не сопровождает его на приёме у императора?
Но сейчас ему стало ужасно стыдно за все эти свои прошлые мысли.
Ведь ему действительно просто очень, очень"=очень хотелось, чтобы принцесса была
здесь с ним, видела бы этот прекрасный блистающий зал, нарядную весёлую толпу,
слышала бы эту замечательную музыку "--- он был уверен, что музыка ей особенно
бы понравилась, сменявшие друг друга мелодии были одна другой удивительней.

"--* Как жалко, что Альсбеты нет здесь со мной.
Надо будет ей хотя бы рассказать хорошенько об этом вечере!

И, продолжая отвечать на улыбки, рукопожатия и поздравления, он, сам себе не
отдавая в этом отчёта, стал проталкиваться в сторону выхода.

В соседних залах, менее торжественных, чем Большой Тронный, тоже теперь было
полным полно гостей, большинство из которых танцевали.
Глаза Гвенаэля немного устали от яркого света, губы устали улыбаться, рука
устала от рукопожатий, и радость в его сердце стала менее пронзительной, более
мягкой и спокойной.
Ему удалось наконец пройти через танцующую толпу к парадной лестнице, и,
спускаясь по красной ковровой дорожке, он уже полусонно думал об Альсбете, о
том, как он ей обо всём здесь расскажет, вот об этой лестнице тоже, хотя она,
наверное, её видела, по крайней мере, когда была ещё маленькой.
Зато чудесные мелодии непонятно где спрятавшегося оркестра она скорее всего
никогда не слышала "--- только вот как об этом рассказать?
Мысли путались в его усталой голове, он почти что видел перед собой лицо
принцессы: голубые огромные глаза, белоснежные волосы, небрежно закинутые за
уши, немного удивлённый рот \ldots\
Он задумчиво перешагивал через ступеньки парадной лестницы, как вдруг кто-то
подошёл к нему со спины и сильно ударил его по плечу.

"--* Хо-хо-хо, поздравляю! "--- услышал он у себя за спиной низкий и немного
булькающий голос, и резко обернувшись, увидел Филипона.
Старшего, разумеется "--- Филипон Второй младший на церемонии не присутствовал,
должно быть, ждал где-то в карете неподалёку от дворца своего хозяина.

"--* Поздравляю, поздравляю, "--- довольно улыбаясь, повторял бородатый толстяк,
и тут Гвенаэлю вдруг мгновенно стало очень нехорошо.
Он даже не успел подумать о том, как расстроилась бы принцесса, узнай она о его
встрече с Филипоном, не успел подумать о том, полная или неполная глупость
разговоры о том, что Филипон мог когда-до съесть родителей Альсбеты "--- просто
у него моментально, с необычайной ясностью, всплыла перед глазами сцена их
давнего обеда в харчевне \enquote{Жирная похлёбка}, он увидел, как наяву,
огромную миску тёмного мутного супа на столе перед Филипоном, и к его горлу
подступила сильная тошнота.
Он оттолкнул снова пытавшегося похлопать его по плечу толстяка, в два прыжка
преодолел остаток парадной лестницы и выбежал из дворца.

Снаружи стояла ночь, холодная и очень тёмная, и от свежего ночного воздуха ему
стало заметно легче.
Он тяжело дышал, прижимал руку к груди и пытался понять, окончательно ли он
пришёл в себя.

Небо над ним было чёрным, и по нему было разбросано несколько светлых и далёких
звёзд.
Холм, на котором стоял дворец, был совершенно пустынным, и кадки с пальмами,
расставленные на нём сегодня утром, сейчас, в темноте, выглядели загадочно и
странно.
Из-за одной такой кадки выглядывали два небольших чёрных глаза и пристально
смотрели на Гвенаэля.
Взгляд казался внимательным, пронизывающим, но при этом не совсем было понятно,
могут ли вообще эти глаза что-либо видеть.
В другое время Гвенаэль тут же испуганно бы отпрянул: чёрный поросёнок,
опасность заразиться чёрной свиной лихорадкой!
Но события этого вечера настолько его утомили, что он только похожим, мало что
понимающим взглядом смотрел в ответ в эти чёрные, глубокие и в то же время как
бы непроницаемые глаза, ни в силах пошевелиться или хотя бы моргнуть.

Наконец, свинке надоела эта игра в гляделки, она выскочила из-за кадки с
пальмой, и, к счастью так и не приблизившись к Гвенаэлю, скрылась за холмом, на
котором стоял императорский дворец.

Гвенаэль понял, что больше уже не хочет рассказывать Альсбете о сегодняшней
церемонии.
Во всяком случае в этот вечер он к ней не пойдёт, совершенно неподходящее у него
было настроение для того, чтобы к ней идти.
Да и в любом случае час уже был для этого слишком поздний.
