\section{Пробуждение}

Он проснулся и сразу понял, что ужасно проспал.
Было очень тепло и душно, потому что он забыл открыть окно, ложась спать "---
его комнатка была крохотной и находилась под самой крышей, обычно летом он
держал окошко открытым и днем и ночью, во всяком случае тогда, когда был дома.

Потянулся рукой к часам "--- 8~часов, не понятно только, утра или вечера.
Обычно это сразу ясно по звукам с улицы, утренний шум не спутаешь с вечерним
гулом, но сегодня было необычайно тихо.
Будильник стоял на шести часах.
\enquote{Неужели он не прозвонил?} "--- с грустным упрёком подумал Омер, но тут
же вспомнил про звяканье во время разговора про теорию струн и понял, что нечего
было винить будильник.
Он звенел вовремя, это он, Омер, не проснулся, несмотря на звон.

У него было ощущение, что спал он безумно долго, может быть больше, чем целые
сутки.
Но даже если он проспал всего одну только ночь, всё равно он опоздал на свидание
с бывшим премьер"=министром Конфуцием.
Одно это уже было достаточно скверным.
Но ещё мучительнее, чем мысль об этом опоздании, было для него воспоминание об
Альсбете.
Он ведь так и не поговорил с ней с тех пор, как она убежала в слезах после
разговора об украденном Гвенаэлем письме.

Но всё-таки прежде всего следовало встретиться с бывшим премьер"=министром и
узнать, как обстоят дела в городе и нашёл ли он средство противостоять власти
императора R.~ле~Кина.

Он позавтракал ломтиком яблока с рябиновым вареньем (он был голоден, но на
большее не было времени), вышел из своей мансардной квартирки, спустился
опрометью по лестнице, выбежал из Кошачьего переулка и быстрым шагом направился
в сторону площади Александера.

Он так торопился, что не сразу заметил, что улицы вокруг него выглядят очень
странно.
По солнцу, неторопливо встававшему за Белой горой, он понял, что сейчас утро,
где-то начало девятого, должно быть.

Обычно как раз в это время открываются булочные и съестные лавки, и улицы
наполняются гулом голосов, шумом шагов, шелестом газет и звоном молочных бидонов.

Но этим утром всюду стояла абсолютная тишина, и на всём пути от Кошачьего
переулка до площади Александера ему не встретился ни один прохожий.

Он со страхом подумал: я всё проспал, и тем временем император сделал с городом
что-то ужасное.
Но всё-таки как могло получиться, что все люди куда-то исчезли?
Он ускорил шаги, и почти бегом добрался до площади, на которой жил бывший
премьер"=министр.

Площадь тоже была пустынной.
Скульптура рогатой сферы выглядела необычно и немного устрашающе: днем ведь
почти всегда на ней сидит огромное количество голубей, но сегодня их не было.
Омер в этот момент понял, что по дороге он не видел не только ни одного человека,
но и ни одной птицы тоже.

Он столкнулся с премьер"=министром в дверях его дома и, даже не поздоровавшись,
спросил:

"--* Что случилось?
Куда подевались все люди и птицы?
Я спал и не мог проснуться, даже не знаю, сколько дней я проспал "--- всё так
изменилось за время моего сна.
Можно ли ещё что-то сделать?
Можно ли ещё как-то всё это поправить?
Ведь люди "--- они не на совсем исчезли, правда?

У Конфуция вид был чем-то озабоченный и немного грустный, но услышав
встревоженные вопросы Омера он не удержался от улыбки.

"--* Никто не исчез, просто все спят, "--- ответил он.
"--- Все будут спать ещё почти целых два часа.
Хорошо, что вы наконец пришли, Омер, я уже думал, что всё придётся делать
самому, без вашей помощи.
Кстати, я заходил к вам сегодня рано утром и долго стучал в вашу дверь, но никто
не ответил, и я думал, что вас не было дома.
Но пойдёмте, нельзя терять время.

Только теперь Омер заметил, что на подоконнике углового окна дома Конфуция
лежала кошка Гафа.
Она спала, свернувшись комочком, так что ни больших жёлтых глаз, ни белого
пятнышка не шее не было видно, просто круглый комок чёрной пушистой шерсти.
В другое время Омер подумал бы, что небезопасно её так оставлять из-за приказа
R.~ле~Кина об отлове кошек, но сейчас он был настолько ошарашен всем
происходящим, что не успел об этом подумать, и просто послушно последовал за
бывшим премьер"=министром.

Первые минуты он шёл молча, смущённый торжественной серьёзностью на лице у
Конфуция, но потом не выдержал и снова стал его расспрашивать:

"--* Но что мы теперь должны делать?
R.~ле~Кин и его стража тоже уснули, или нет?
Как вам удалось всего этого добиться?
Вы нашли в итоге ту книгу с заклинаниями, о которой вы мне рассказывали раньше?

"--* Это не совсем заклинание, "--- ответил Конфуций.
"--- Но нужную книгу я, действительно, нашёл и прочёл в ней то, что следовало
сделать.
В дворцовую библиотеку попасть было невозможно, но, к счастью, книга оказалась
также и в научной библиотеке.

"--* Но ведь научная библиотека была закрыта недавним приказом императора?

"--* Да, она теперь закрыта, но мне помог министр Пьерро, раньше он ведь был
министром образования, и у него сохранился ключ от неё.

"--* Министр Пьерро? "--- не поверил своим ушам Омер.
"--- Новый премьер"=министр в правительстве R.~ле~Кина?
Оказывается, он тоже против императора?

"--* Вы не понимаете, он не против императора и не за императора.
Просто он рождён быть министром, и никогда не откажется от этого поста, какие бы
ни были времена, какие бы ни формировались правительства.
Но он серьёзный человек, всегда уважавший науку "--- в конце концов, бывшая его
должность к этому обязывала.
Поэтому когда я сказал, что ищу книгу по зеркальной симметрии, он сразу
вспомнил, что она была в научной библиотеке, и сам мне предложил ключ от этой
библиотеки.

"--* Зеркальная симметрия, "--- задумчиво произнёс Омер.
"--- Когда-то меня немного интересовала эта наука.
И что же вы прочли в найденной книге?

"--* То, что надо было сделать, оказалось довольно несложным.
В зеркальном магазине на Старой Торговой улице я купил круглое зеркало 9{,}8~дюймов
в диаметре, потом пошёл в часовой магазин и, дождавшись полудня, поднёс
зеркало к циферблату часов в витрине, помните, там всегда в витрине стоят
старинные часы с круглым циферблатом и без минутной стрелки.
Число XII, отразившись в зеркале, превратилось в IIX, и мгновенно все остальные
часы тоже перепрыгнули на 8~утра, люди попрятались по своим домам, мыши "--- по
своим норам, птицы "--- по своим гнёздам, и всё уснуло.
И у нас теперь есть 4~часа, пока время не вернётся обратно к 12-ти, и за эти два
часа нам надо попасть во дворец и увидеть императора.

"--* Но мы идём вовсе не в сторону дворца? "--- удивлённо заметил Омер, в то
время как они вышли на набережную и собирались перейти реку Кортевеговку по
Горбатому мосту.
"--- Ведь времени у нас немного, если стражи и придворные тоже спят, что нам
мешает просто пойти туда и спокойно войти во дворец?

"--* Стражи, наверное, спят.
Но вы забыли про чёрных свиней вокруг дворца, "--- возразил ему министр.
"--- Я не знаю, спят ли сейчас чёрные свиньи, но даже если они и спят, чёрная
свиная лихорадка может быть заразной даже во время их сна.

В этот момент Омер увидел, что они подошли к зданию аптеки.
\enquote{Вряд ли мы сможем найти здесь лекарство от свиной лихорадки, "---
подумал он.
Все эти листья ландыша, корни боярышника и тёртые мумии помогают в лучшем случае
от простуды или зубной боли, но при серьёзных болезнях от них мало толка.
Если бы существовало лекарство от чёрной свиной лихорадки, император, наверное,
уже давно не был бы таким могущественным}.

Но всё же Омер поднялся вслед за Конфуцием по лестнице "--- две ступеньки,
3~ступеньки побольше, поворот, 5~ступенек, ещё поворот, 7~ступенек, крылечко с
выложенным лазурным песчаником правильным 17-ти угольником "--- и вошёл в аптеку.

Он всегда раньше считал, что из маленькой прихожей есть только одна дверь "---
та, что вела в основной аптечный зал.
Но премьер"=министр отодвинул большую каменную ступку, в который помощник
аптекаря смешивает обычно порошки для приготовления лекарств, и оказалось, что
под ней в полу есть тоже что-то вроде дверцы.
Надо было потянуть за железную заржавевшую ручку, и тогда открывался вход в
подвал аптеки.

Они спустились по шаткой приставной лестнице, и Омер увидел, что они находятся
посреди аптечного склада.
Так же как и наверху в аптечном зале стены были покрыты полками с фарфоровыми
банками, только многие из них были менее яркими, чем в зале наверху, и тоже на
каждой банке цветными выпуклыми буквами было написано название содержащегося в
ней снадобья.

В помещение было темно, свет падал только через дверцу в потолке, через которую
они только что вошли.
К счастью, на одной из полок лежала свечка и спички, Конфуций зажёг её и повёл
Омера дальше "--- из этой подвальной комнаты начиналась другая лестница, уже не
приставная, а обычная.
Они спускались всё ниже и ниже, и на каждом этаже были тёмные комнаты и полки,
заставленные лекарствами.

Но вид помещений менялся.
Деревянные полки с какого-то момента стали меньше, но теперь они были аккуратно
покрашены, и на них вместо больших фарфоровым банок стояли коробочки и пузырьки.

Они спустились ещё ниже, и пузырьки тоже исчезли, а на коробочках теперь были
наклеены ярлычки с названиям лекарств.
\enquote{Аспирин} "--- прочёл на одной из них Омер слово, знакомое ему только по
жизни по другую сторону сна.

Потом ещё одним этажом ниже коробочки стали цветными, названия на них были
теперь всегда напечатаны, а не написаны от руки.
Одна из полок была полностью занята коробочками с надписью \enquote{Пенициллин},
Омер хотел рассмотреть её получше, но премьер"=министр кивком головы дал ему
понять, что надо продолжать спускаться.

Пройдя ещё один лестничный пролёт они оказались в помещении, которое по размеру
было больше, чем все предыдущие, и различных лекарств в нём было столько, что
Омер даже не попытался запоминать их названия.
Эта комната было последней, лестница здесь кончалась, зато в одном из её углов
находилась дверь лифта.

Конфуций задул свечку, она больше не была нужна, так как всё было ярко освещено
находящимися под потолком синими продолговатыми лампами.

Лифт находился уже прямо на этом этаже, они вошли в его кабину, и Омер удивился,
как много в ней было разных кнопок: были, как обычно, \enquote{1}, \enquote{2},
\enquote{3}, \ldots, но были также и \enquote{$-1$}, \enquote{$-2$},
\enquote{$-3$}, \ldots, были кнопки, помеченные буквами, или буквами с цифрами:
\enquote{G2}, \enquote{E7}, \enquote{E8}.
Были также кнопка \enquote{P} и кнопка \enquote{NP}.
Омер подумал, что для того чтобы узнать, равны ли P и NP, надо просто по очереди
нажать эти две кнопки и посмотреть, приедет ли лифт на один и тот же этаж или
нет.
Но в этом момент бывший премьер"=министр нажал кнопку \enquote{$-273$}, и лифт
начал плавно двигаться вниз.

Омер подумал: ладно, меня никогда всерьёз не интересовала задача про NP.
Лифт в это время набирал скорость, он спускался всё быстрее и быстрее, у Омера
захватило дыхание и ему стало казаться, что они не едут, а падают.
Но к счастью в конце концов лифт стал замедляться, и остановился, как ему и
полагалось, на минус двести семьдесят третьем этаже.

Выйдя из лифта, они оказались в маленьком коридорчике, перед дверью с надписью:
\enquote{Лаборатория. Посторонним вход воспрещён}.
Бывшего премьер"=министра нисколько не смутила эта надпись, он уверенно постучал,
и приятный голос ответил из-за двери:
\enquote{Да, да, входите, я вас уже ждал}.

Они вошли, и оказалось, что голос принадлежал молодому человеку в белом халате,
который сидел, склонившись над микроскопом.
Комната, в которой они находились, была довольно маленькой, и вся она была
заставлена какими-то колбочками и множеством сложных приборов, по большей части
Омеру неизвестных.

"--* Вот и вы, наконец, "--- воскликнул молодой человек, увидев Кон~Фу~дзе и
Раджафера.
"--- Можете меня поздравить, полчаса назад я выявил вирус, отвечающий за чёрную
свиную лихорадку!

"--* То есть последнее препятствие для входа во дворец преодолено? "--- спросил
бывший премьер"=министр.

"--* Ну, этого я бы не сказал.
Вирус я выявил, смотрите, вот как он выглядит, "--- сказал биолог, протягивая
Конфуцию и Омеру лист с довольно сложной картинкой.
"--- Но одно дело, выявить вирус, а другое дело, найти против него лекарство,
если оно, вообще, существует.

"--* Можно ведь ещё попытаться придумать против него вакцину? "--- неуверенно
спросил премьер"=министр.

"--* Да, я думаю, это реально, но её изготовление займёт в лучшем случае
несколько месяцев.
У меня есть некоторая идея \ldots

"--* Несколько месяцев, "--- перебил его министр.
"--- В нашем распоряжении есть только несколько часов.
И, вынув из кармана часы на золотой цепочке, он прибавил: 3~часа, 14~минут и
16~секунд, если быть более точным.

В это время Омер, разглядывая лист со схемой вируса, пробормотал:
\enquote{Где-то я уже видел эту картинку}.

"--* Что-то математическое? "--- спросил его Кон~Фу~дзе.
"--- Постарайтесь вспомнить!
Вы ведь почти единственный, кто и здесь, по эту сторону, может помнить про
математику.

Омер несколько минут очень внимательно смотрел на картинку, но вспомнить ему не
удавалось.

"--* У меня что-то последние дни не то с головой, "--- пожаловался он.
"--- Мне кажется, я вообще сейчас про математику не могу думать.

"--* Но это очень важно, Омер, попробуйте ещё, "--- настаивал бывший
премьер"=министр.

Тогда Омер снова посмотрел на схему вируса и задумчиво пробормотал:
\enquote{\foreignlanguage{french}{Dessin d'enfant}}.

"--* Детский рисунок? "--- удивлённо переспросил биолог.
Сложная аккуратная схема на листе бумаги совсем не была похожа на что-либо,
нарисованное ребёнком.

"--* Есть, кажется, такое математическое понятие, "--- объяснил ему Конфуций.
"--- Но это совсем не по моей специальности, так что кроме самого термина я
ничего про это не знаю.
Омер, постарайтесь, вы должны вспомнить и объяснить нам, в чём смысл этого
детского рисунка.

"--* Кажется, я просто видел его в каком-то учебнике как пример рисунка с
маленьким полем определения, "--- ответил Омер, задумался, но потом мотнул
головой и добавил: \enquote{Нет, я ничего не помню}.

"--* Подумайте ещё Омер, я ведь действительно про это ничего не знаю, "---
сказал бывший премьер"=министр.
Единственная надежда на вас.
Что такое поле определения?

"--* Я не помню, "--- сказал Омер.
"--- Поле определения \ldots\
Он закрыл глаза и изо всех сил постарался сосредоточиться.
"--- Поле определения, пробормотал он, не открывая глаз.
Поле \ldots\
Кажется, я что-то вспоминаю.

Биолог и бывший премьер"=министр посмотрели на него с надеждой.

"--* Что вы вспоминаете? "--- спросил Конфуций.

Казалось, что Омер находится в некоторой прострации.
"--- Поле, "--- произнёс он очень задумчивым голосом.
"--- Поле \ldots\
Это, кажется, когда на плоской местности растёт много травы или колосьев \ldots

"--* Это мы и сами знаем, "--- в голосе Конфуция, обычно неизменно спокойном,
прозвучала некоторая досада.
"--- Время идёт, надо что-то делать, осталось немногим больше трёх часов до того,
как стрелка часов заново дойдёт до XII и всё проснётся.

"--* Если бы я мог заснуть, полистать книжки в моем кабинете, а потом снова
вернуться сюда, то я бы вам объяснил, что означает этот рисунок, "--- сказал
Омер.

"--* В течении этих четырёх несуществующих часов всё спит, но тот, кто не спит,
уже не может заснуть, "--- возразил бывший премьер"=министр.

"--* Глупости, "--- не слишком почтительно перебил его биолог.
"--- С хорошим снотворным кто угодно заснёт, даже сейчас.
Давайте, я сделаю Омеру инъекцию, он поспит немного, а потом нам всё расскажет.

Бывшему премьер"=министру эта мысль не очень понравилась, но после короткого
спора он всё-таки согласился.

Биолог открыл небольшую дверцу в стене и оказалось, что за ней находится
холодильник.
Он достал оттуда какую-то ампулу и сделал Омеру укол, после чего тот моментально
уснул.

Пока он спал, молодой человек снова сел что-то рассматривать под своим
микроскопом, а Конфуций сидел молча и о чём-то думал, стараясь не мешать его
работе.

Прошло чуть меньше часа, когда они решили, что Омера пора разбудить.

"--* Подождите, я сварю кофе, а то ему трудно будет проснуться после
снотворного и такого короткого сна, "--- сказал биолог, зажёг небольшую
газовую горелку, на которой обычно кипятят в пробирках химические вещества,
достал из стенного шкафа маленькую кофейную джезвочку и поставил её на огонь.

"--* Это не противоречит правилам безопасности, готовить еду, или, вернее, питьё,
прямо в лаборатории? "--- удивлённо спросил Конфуций, но биолог в ответ только
устало махнул рукой и сказал:

"--* Будите Вашего друга, кофе сейчас будет готов.

Омер, едва открыв глаза, поспешил сообщить им то, что он узнал.

"--* Ничего особенного про этот рисунок мне выяснить не удалось.
Он, действительно, просто приведён как пример в одном учебнике по
\enquote{\foreignlanguage{french}{dessins d'enfants}}.
Его орбита под действием группы Галуа $\bar{\mathbb{Q}}$ над $\mathbb{Q}$
состоит из трёх элементов, два других элемента орбиты тоже были нарисованы в
книге, которую я смотрел.

И, сделав несколько глотков кофе, Омер взял чистый лист бумаги и нарисовал на
нём два рисунка, сопряжённых с изначальным под действием группы Галуа.

"--* Это невероятно! "--- воскликнул биолог.
"--- То, что вы нарисовали, это тоже схемы вирусов, первый "--- оспы, второй
"--- обычной свинки!
У меня и раньше был некоторые мысли о возможной связи вируса с оспой, но вот
про свинку я не подозревал!
Теперь я почти уверен, что смогу изготовить вакцину против лихорадки "--- вернее,
даже ничего по настоящему нового изготовлять не надо "--- учитывая наше знание
механизмов вакцины против свинки и вакцины против оспы.

"--* Даже названия сходятся, "--- пробормотал себе под нос бывший
премьер"=министр.
"--- Чёрная свиная лихорадка: свинка и чёрная оспа!

Биолог тем временем снова открыл маленький холодильник в стене, и Омеру на этот
раз лучше удалось рассмотреть его содержимое.
На трёх небольших полках теснилось огромное количество баночек, ампул и
коробочек с какими-то химическими веществами, но Омеру показалось также, что в
дверце холодильника лежало нечто очень похожее на небрежно завёрнутый в бумагу
кусок полузасохшего сыра.
Молодой человек достал две ампулы, одну с совершенно прозрачной, другую с
зеленоватой жидкостью, потом закрыл холодильник и взял с полки над входом
какой-то порошок.
Он вернулся к своему столу, повернул небольшую ручку на стене слева от себя,
после чего казавшаяся до этого частью стены пластинка откинулась вниз, и они
увидели встроенный в стену прибор, который, судя по его виду, был самым новым и
современным из всех находившихся в комнате приборов.
Биолог нажал одну за другой три кнопки на его верхней панели, после чего их
стены выскочили три ящичка, два их которых явно были предназначены для ампул.
Молодой человек засыпал в первый из ящичков взятый им до этого порошок,
вставил ампулы в два других, после чего задвинул их обратно в стену.
(Омер отметил с удивлением, что до этого он даже не открыл казавшиеся запаянными
ампулы).
Биолог же склонился над находящейся на нижней панели прибора клавиатурой,
быстрым движением нажал довольно длинную комбинацию клавиш, и сообщил,
обернувшись обратно к Омеру и Конфуцию, что через несколько минут новая вакцина
будет готова.

Действительно, уже через несколько минут из прибора выскочила пробирка,
наполненная какой-то жидкостью, на этот раз не зеленоватой, но и не прозрачной,
светлой, но немного мутной.

"--* Вам, кстати, очень повезло, что я недавно открыл способ делать вакцины,
которые действуют очень быстро, "--- объяснил биолог Омеру и Конфуцию.
"--- Сейчас я сделаю вам обоим по прививке, и уже меньше, чем через час, вы
будете иммунны к чёрной свиной лихорадке.

После прививки премьер"=министр и Омер поблагодарили молодого биолога и
отправились в путь.
Они сели в лифт, который вверх шёл ещё быстрее, чем спускался, потом поднялись
по узкой лестнице, проходя все те же комнаты с лекарствами, которые они уже
видели при входе, и через аптечную дверь вышли на набережную.

На улице по"=прежнему было необычайно тихо, только изредка в воздухе раздавался
еле слышный звук, напоминающий звон фарфора.
Всю дорогу они шли молча, было как-то неловко нарушать столь совершенную тишину,
и только уже подойдя к дворцовому холму Омер произнёс:

"--* Оказывается, свиньи тоже спят.

Действительно, склоны холма и площадь перед дворцом были заняты чёрными
свиньями, которые лежали и спали, некоторые тихонько похрюкивая во сне.

Конфуций и Омер подошли к главным воротам, осторожно выбирая при каждом шаге,
куда поставить ногу "--- тревожить свиней, даже спящих, им было как-то не по
себе "--- и вошли во дворец.

Обычно площадка перед парадной лестницей охранялась большим количеством до зубов
вооружённых охранников, но сейчас никого из них не было видно.

Они поднялись по лестнице, прошли по парадному коридору, столь же пустынному,
как и вход во дворец, и вошли в Старый Императорский зал (который до прихода
R.~ле~Кина к власти назывался Старым Королевским или попросту Зеркальным залом).
Потом из него прошли в Малый Весенний зал, тот самый зал, который всегда казался
очень высоким потому, что голубой потолок его был очень похож на небо.
Только тут Омеру пришёл в голову вопрос, о котором он мог бы подумать и раньше.

"--* А что, собственно, мы собираемся делать с R.~ле~Кином?
Он, конечно, совершил много нехороших вещей "--- отловил кошек, запретил
театральные представления "--- но всё-таки несмотря на это я не готов к тому,
чтобы совершить какое-либо серьёзное насилие по отношению к нему.

"--* Насилие? "--- переспросил бывший премьер"=министр, и если бы не его
всегдашний спокойной голос, можно было бы подумать, что он несколько шокирован.
"--- Насилие?
Конечно, нет, Омер, следуйте за мной.

Раджафер послушно двинулся вслед за Конфуцием, но подходя к дверям, ведущим в
Большой Тронный зал, тот внезапно сам остановился.

"--* Подождите, Омер, я хочу кое-что ещё обдумать, перед тем как входить.

Он подумал вначале, что премьер"=министр забыл предусмотреть то обстоятельство,
что Большой Тронный зал может оказаться заперт.
Обычно проход между ним и Малым Весенним залом был открыт "--- по крайней
мере, так было во время тех немногочисленных королевских приёмов, на которых
Омеру когда-то давно доводилось присутствовать "--- но сегодня обе створки
бело"=голубых высоких дверей, разделяющих Весенний и Тронный зал, были плотно
замкнуты.
Впрочем, разглядывая их внимательнее, Омер пришёл к заключению, что дверь просто
закрыта, а не заперта.
По крайней мере, с того места, где он стоял, казалось, что между двумя створками
дверей остаётся небольшая щель, через которую в Малый Весенний зал просачивается
полоска золотистого света.
На двери или за дверью не было видно ни замка, ни задвинутого засова.
Для того, чтобы в этом окончательно убедиться надо было подойти к ней поближе,
но Омер не решался это сделать без указания Конфуция.
Он посмотрел нерешительно в его сторону, и его несколько удивило задумчивое и,
как ему показалось, довольно мрачное выражение лица бывшего премьер"=министра.
Было ясно, что думал тот о чём-то значительно более существенном, чем возможный
засов на входе в Большой Тронный зал.
И в этот момент Омер впервые за последние несколько часов наконец с полной
ясностью осознал серьёзность того, что они собираются сделать, или, вернее, уже
отчасти сделали.

До этого, конечно, ещё по дороге во дворец, ему становилось время от времени
немного не по себе, но всё было столь необычным и совершалось в такой спешке,
что у него не было практически ни одной свободной минуты для того, чтобы
спокойно подумать и отдать себе отчёт в происходящем.
Но вот теперь, за те недолгие мгновения, когда Конфуций почему-то нерешительно
остановился перед входом в Большой Тронный зал, мысли в голове у Омера
прояснились и он с ужасом подумал:
Неужели мы действительно уже во дворце и собираемся вот-вот совершить нечто
вроде антиимператорского переворота?!
И значит, стоит допустить малейшую ошибку "--- и не надо большого воображения
для того, чтобы представить, какая участь может им грозить, ему и бывшему
премьер"=министру.
Неясно, конечно, сделают ли тогда их возможное наказание всем известным, в
назидание остальным, или же, напротив, им достанется примерно столько же
внимания, как отловленным в своё время по всему городу кошкам \ldots\
Такие мысли надо было гнать сейчас от себя прочь со всей силой.
Омер никогда не был трусом, и сейчас с его стороны не было лицемерием считать,
что все эти его размышления "--- это только в небольшой части физическим страх
за его собственную жизнь.
Тем не менее, он чувствовал как его пронизывает ужас.

Ужас был похож на холод: он замораживал пальцы рук, пробегал мурашками по спине,
едва касался оцепеневших лодыжек и спускался в и без того уже замёрзшие ступни
ног.
Мраморный пол в Весеннем зале был очень холодным, Омер это отчётливо ощущал
сквозь тонкую подошву ботинок, в которые он был обут.
Переминаясь с ноги на на ногу, он продолжал думать о том, чем, в случае неудачи,
может обернуться их затея.

Действительно, дело ведь было не только в них самих, Конфуций был чуть ли не
единственным оставшимся в городе благоразумным политиком.
Даже не смотря на то, что он не участвовал в новом правительстве R.~ле~Кина,
его присутствие в городе безусловно имело смысл и, наверняка, не одному только
Раджаферу внушало существенную надежду.
Но теперь, если их план провалится, ни одного хорошего и сильного политика не
останется, и тогда страшно подумать о том, что может начать происходить и в без
того не слишком преуспевающем последнее время городе.
Не было ли всё-таки с их стороны ошибкой замышлять что-либо против R.~ле~Кина?
Они знают уже, что бывший генерал R.~ле~Кин "--- личность не из приятных, но
знают ли они на что он может быть способен, если его разозлить или хорошенько
напугать?

Размышляя, Омер оглядывал зал, в котором они находились.
В какой-то момент ему показалось, что дверь, ведущая в Большой Тронный зал
тихонько скрипнула, и одна из её створок немного отворилась, после чего полоска
золотистого света, падающая через дверную щель в Малый Весенний зал, стала
несколько шире, чем раньше.
Омер скользнул глазами по этой тонкой световой дорожке, потом дальше по
белому"=белому мраморному полу, по бело"=голубым стенам и небесно"=голубому
потолку.
Он был очень красив, этот Малый Весенний зал, красив и полон какой-то лёгкой,
действительно как будто бы весенней радостью.
Было очень странно в нём находиться сейчас, когда настроение было отнюдь не
весёлым и не лёгким.
Надо сказать, что он даже сам не мог понять, какое именно у него настроение.
До этого он привык почти всегда иметь очень чёткую картину того, что происходит
вокруг, а сейчас он чувствовал тревогу и полную растерянность.
Его уверенность в себе и даже его взрослость куда-то подевались, он чувствовал
себя намного младше, чем он был.
Этому, кроме наполнявшей его тревоги, было, возможно, ещё одно объяснение "---
рядом с Конфуцием почти кто угодно почувствует себя мальчишкой.

\enquote{Интересно, куда всё-таки делись императорские стражи, до сих пор мы ни
одного из них не видели.
Может быть, они как раз все собрались в Большом Тронным зале, вокруг R.~ле~Кина?}

Омер прислушался, но дворец был тих, абсолютно тих, по-прежнему ни единого звука
и ни единого шороха.
Если бы все стражи были бы в тронном зале, даже если бы все они заснули,
наверняка хотя бы один из них захрапел или пошевелился во сне, и это можно было
бы услышать.
Свиньи ведь похрюкивали тихонько во сне, когда они мимо них пробирались.

\enquote{Даже про вакцину у нас ведь нету никакой уверенности}, "--- внезапно
подумал Омер, хотя эта мысль и казалась ему довольно ничтожной по сравнению со
всеми остальными возможно грозящими им опасностями.
Тем не менее, и в правду, даже насчёт вакцины у них не могло быть никакой
уверенности, ведь они были первыми, на ком её испытали, и поэтому нельзя было
исключить возможность того, что они всё-таки заразились чёрной свиной лихорадкой
проходя, перед тем как войти во дворец, мимо дремлющих чёрных свиней.

\enquote{Но об этом уж точно сейчас без думать толку, "--- заключил свои размышления
Раджафер, "--- если мы и заразились чёрной свиной лихорадкой, то это, скорее
всего, мы узнаем не раньше, чем через пару дней}.

"--* Пойдёмте, Омер, "--- негромко сказал в этот момент бывший премьер"=министр.
"--- Простите, я тут несколько задумался, но нам не стоит больше мешкать.

"--* Подождите, "--- возразил ему Омер, и, несколько смущённым, но в то же время
твёрдым голосом добавил: Я хотел бы узнать, о чём вы сейчас думали.

Конфуций приподнял брови "--- едва-едва, так что по его лицу почти нельзя было
догадаться об его удивлении, потом немного улыбнулся и сказал:

"--* Это было нечто совершенно личного порядка, к нашему делу никак не
относящееся.
Нелепо, наверное, что я об этом вспомнил именно сейчас.

Сказав последнюю фразу, премьер"=министр сделал шаг в сторону входа в Большой
Тронный зал, но увидев некоторое разочарование и недоверие на лице у Омера, он
ещё раз улыбнулся краешком рта и добавил:

"--* Впрочем, если вы настаиваете, могу вам сказать, о чём я сейчас думал.
Тут уж, во всяком случае, нет ничего секретного.
Я думал о прошедшем годе.
У меня есть такая привычка, каждый год, к моему дню рождению подводить небольшой
мысленный итог того, что я сделал за этот год.
В молодости это было более увлекательным занятием, каждый год было множество
интересных новых городов, которые ты посетил, множество очень важных "--- или во
всяком случае казавшихся очень важными "--- новых законов, принятых
правительством.
Последние же годы это всё большей частью воспоминания о встречах со старыми
друзьями, которых долго до этого не видел, или, ещё вот бывает, о каком"=нибудь
особенно красивом или просто почему-то запомнившемся шахматном этюде.

Раньше о прошедшем годе я любил вспоминать в сам день рождения, но потом понял,
что лучше всего это делать за пару дней до него.
Потому что потом обычно начинается изрядная суматоха, съезжаются все мои
родственники "--- вы знаете, весь на одной только Белой горе у меня имеется
несколько дюжин кузин, племянников, племянниц, внучатых племянников и племянниц.
А в этом году, я вам, кажется, уже говорил, у меня юбилей, так что шума,
наверное, будет ещё больше, чем обычно \ldots

Омер слушал бывшего премьер"=министра и не мог поверить своим ушам.
В своём ли тот уме?
Перед лицом страшной опасности, в момент свершения антиправительственного
переворота, уже находясь в императорском дворце, он внезапно останавливается на
несколько минут чтобы подумать о прошедшем годе, своём дне рождения, внучатых
племянниках и даже особенно красивых шахматных этюдах?!

Или, может быть, он просто не хочет говорить Омеру то, о чём он на самом деле
думал?
Если так, не то чтобы это очень обижало Омера, но всё-таки казалось ему
достаточно нечестным.
В эту минуту, когда они вдвоём принялись за такое непростое и опасное дело, ему
казалось, что было достаточно естественно рассчитывать на полную откровенность
со стороны Конфуция.

И потом, наконец, ему пришло в голову объяснение, показавшееся ему более
убедительным, чем предыдущие: может быть, это всё-таки правда, и вовсе на такая
уж беспечность со стороны Конфуция, может быть, он именно сейчас решил подумать
о близких ему людях, родных, или даже просто о разных мелочах ушедшего года,
потому через мгновение они войдут в Большой Тронный зал, и после этого им может
понадобиться подвести итог не только ушедшему году, но, возможно, и всей их
прошедшей жизни.

Всё равно слова премьер"=министра казались ему очень странным, сам Омер
чувствовал себя как будто загипнотизированным, и не был в состоянии думать о
чём-либо ином, чем то, что их ждёт через несколько минут.

Все эти мысли пронеслись у него в голове со страшной скоростью, времени на
долгие размышления у него снова не было, потому что Конфуций толкнул дверь,
ведущую в тронный зал и вошёл в неё, явно ожидая, что Омер тотчас же за ним
последует.

Омер подошёл к приоткрытой до этого Конфуцием двери, боясь в неё заглянуть.
Глядя себе под ноги, он перешагнул через дверной порог и при этом почувствовал,
что у него замирает сердце: что-то подсказывало ему, что R.~ле~Кин должен
находиться именно в Большом Тронном зале.

Но, войдя, он наконец поднял голову и вздохнул с облегчением и некоторым
разочарованием: огромный парадный зал, в котором они стояли теперь, тоже был
пуст, или, во всяком случае, казался пустым.

В первый момент у Омера немного заболели глаза от непривычного блеска "---
почти всё в Большом Тронном зале было золотым: стены, потолок, огромные люстры,
и даже наборный паркет на полу, изображавший что-то, вроде множеств
Мандельброта, помимо различных видов редкого дерева содержал также и золотые
вкрапления.

Бывший премьер"=министр подошёл тем временем к трону и обернулся к Омеру,
ожидая, что тот тоже туда подойдёт.

Омер приблизился к огромному креслу с золотыми подлокотниками в виде голов
семиязычных драконов и только теперь заметил, что трон не пуст.
Посередине него, на большой подушке из тёмного бархата, стояла крохотная
фарфоровая фигурка.

Омер взял её в руки и увидел, что она изображает императора.
Он удивлённо посмотрел в сторону Конфуция, надеясь получить от него объяснение,
но тот вместо ответа только молча кивнул головой в сторону подножия трона, и
Омер увидел, что на золотых ступеньках перед тронным креслом стоит ещё целое
множество фарфоровых фигурок.
Одни из них изображали вельмож, другие "--- придворных дам, но больше всего
среди них было фарфоровых стражей и воинов, вооружённых фарфоровыми мечами и
фарфоровыми секирами.

"--* И это всё, что от них осталось? "--- недоверчиво спросил Омер, крутя в
руках фарфоровое изображение R.~ле~Кина.
И должны ли мы ещё что-либо сделать с императором?
Не смогут ли они все в будущем снова вернуться к своему прежнему облику?

"--* Есть один магазин на Старой Торговой улице, в котором я ещё не был на этой
неделе, "--- ответил бывший премьер"=министр на один из заданных вопросов.
"--- Надо все фигурки собрать и туда отнести.
Если мы вынесем их из дворца раньше, чем время снова вернётся к полудню, то вряд
ли они когда-нибудь смогут вернуться к своему прежнему облику.

У них не было с собой никакой сумки или котомки, поэтому Омеру пришлось снять с
себя пиджак и сложить в него всех придворных, воинов и стражников.
После чего, звеня своей фарфоровой ношей, они направились к выходу.

Свиньи на площади перед дворцом ещё спали, но, как показалось Омеру, уже менее
крепко "--- многие из них ворочались во сне, и их сонное похрюкивание
становилось всё более громким.

Омер и Конфуций спустились в город, и по-началу улицы, по которым они шли, были
пустынными, но той абсолютной тишины, которая была раньше, уже не было, город
понемногу оживал.
Через некоторое время им стали встречаться отдельные прохожие, которых
становилось всё больше и больше по мере того, как они подходили к центру города.

Когда они вышли на Старую Торговую улицу, она уже выглядела совсем обычно, на
ней было много людей, входивших или выходивших из разных лавок, просто гуляющих
или куда-то спешащих.

Проходя мимо часового магазина Омер заметил, что старинные часы в его витрине
показывают начало первого.
Сколько именно минут прошло после полудня сказать было нельзя, потому что, как
известно, минутная стрелка на них отсутствовала, а все остальные выставленные на
витрине часы были маленькими, и Омер не успел их толком разглядеть.

Они прошли мимо магазина зеркал и зашли в фарфоровую лавку.
Кроме них в ней не было ни одного посетителя, был только сам хозяин, который в
этот момент распаковывал большую картонную коробку и доставал из неё чайные
чашки, белые, фарфоровые "--- как и всё, находившееся в этом магазине "--- чашки,
покрытые крошечными разноцветными бабочками.
Конфуций махнул рукой, и, правильно угадав значение его жеста, Омер высыпал все
фигурки из своего пиджака на прилавок, после чего бывший премьер"=министр стал о
чём-то негромко говорить с хозяином лавки.

Омер не слушал их разговор.
Он прислонился спиной к дверному косяку у входа в магазин и наблюдал за
проходящими мимо людьми.

\enquote{Неужели это всё, неужели мы победили, и всё так просто? "--- думал он.
"--- И неужели все эти идущие по улице люди, смеющиеся и болтающие между собой,
ничего не знают, ничего не подозревают о четырёх неположенных часах, прошедших
между повторными восьми утра и сегодняшним полуднем?}
