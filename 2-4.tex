\section{О том, к чему может привести необычайно жаркая погода}

Несмотря на все перипетии этого дня, вечером Омер вспомнил, что ему пора
показаться в институте.
Его весьма беспокоили разные вещи, происходящие там в последнее время, особенно
людоедские планы двух Филипонов по отношению к визитёрам и постдокам.
На сегодняшнем собрании профессоров должны были как раз обсуждаться будущие
визитёры, так что стоило там показаться и постараться воспрепятствовать
осуществлению этих планов.

С другой стороны здесь тоже неприятностей и дел хватало, он упрекал себя за то,
что он не сумел поговорить с Альсбетой после того, как она узнала об украденном
Гвенаэлем письме.
Надо было бы как-то её утешить, может быть и сейчас ещё не поздно найти её и
как-то объяснить ей эту скверную историю?
С другой стороны, это скорее Гвенаэль должен в этой ситуации объясниться с ней,
ему-то что ей теперь сказать?
Но, вспоминая слезы в глазах принцессы, ему становилось настолько не по себе,
что он бы наверняка бросился её искать, если бы не боялся, что после
происшедшего ей может быть неприятно с ним встречаться.

Потом ещё была завтрашняя встреча с премьер"=министром, наверное, стоило бы к
ней подготовиться.
Но в конце концов он решил, что оставлять без присмотра двух Филипонов тоже
слишком опасно, поставил будильник, чтобы не проспать утром встречу с Конфуцием,
и с нелёгким сердцем заснул.

Проснулся он не в институте и не у себя дома, как это бывало обычно, а в
каком-то пригороде.
Сначала он даже не понял, где он находится "--- он стоял посреди негустого
хвойного леса, ничем, на первый взгляд, не примечательного.
Но вскоре он заметил огромные камни, раскиданные между деревьев, и после этого
сразу узнал это место, хотя ему нечасто приходилось здесь бывать.
Он вышел на большую дорогу, идущую через лес, на ней оказалось много гуляющих
людей, но ему уже не надо было у них спрашивать, как пройти дальше "--- он знал,
всего несколько минут ходьбы, и он окажется на железнодорожной станции.

День был очень жарким, и от этого люди на платформе выглядели усталыми, и
маленькие дети капризничали больше обычного.
Омер почувствовал, что в голове от жары возникает неприятная тяжесть.
Стоило, наверное, купить бутылку минеральной воды, но он не догадался это
сделать.
Здесь, по эту сторону было много довольно обычных вещей, которые он никогда не
делал "--- так, покупка минеральной воды в станционных автоматах совершенно не
входила в его привычки.

К счастью, электричка подошла довольно быстро.
Он сел в неё и начал раздумывать о том, как бы поскорее со всем здесь
разобраться и вернуться обратно.
Если ему повезёт, то он успеет ещё попасть в институт, быстро со всем разделаться
и вернуться назад, надо ни в коем случае не опоздать на свидание с Конфуцием,
всё ещё получится, "--- сказал он сам себе, но на этой его мысли электричка по
непонятной причине остановилась посередине перегона между станциями.
Водитель по репродуктору несколько раз объявил, что через пару минут она поедет
дальше, но прошло не меньше четверти часа, а поезд по"=прежнему не двигался.

Омер понял, что теперь ему точно не успеть обернуться быстро, к тому же он всё
равно уже скорее всего опоздает на сегодняшняя собрание в институте.
Он подумал, бог с ним с институтом, всё равно это не моя вина, что я туда не
успеваю, надо ни с чем возвращаться обратно, потому что я обещал
премьер"=министру, и ещё потому что там Альсбета, что мне делать с Альсбетой?

Он попытался задуматься о математике, чтобы вернуться, но из-за духоты в вагоне
было трудно сосредоточиться.
В конце концов ему удалось вспомнить задачу, которой он занимался последний
месяц.
Всерьёз о ней думать он ещё не начал, только размышлял для разминки о том, нужно
ли в её постановке требовать, что $\phi$ не равняется нулю, когда внезапно за
его спиной зазвучало громкое пение.

Он оглянулся и увидел, что пела немолодая негритянка, полная невысокая женщина с
большими и немного сумасшедшими глазами.
Низким красивым голосом она несколько раз пропела что-то вроде:

\foreignlanguage{english}{\ldots\ and then he approached me and touched me\ldots}

Омер следил за недовольными и смущёнными лицами пассажиров и спрашивал себя, что
их больше всего смущает, сам факт, что женщина поёт, или чувственность её песни?

Но в этот момент женщина ещё громче запела: \enquote{Аллилуйя! Аллилуйя!} и Омер
понял, что неправильно интерпретировал до этого содержание песни, которая, по
всей видимости, была каким-то известным христианским гимном.

В общем, всерьёз подумать о математике в электричке так и не удалось.
Когда Омер приехал в институт, был уже вечер.
Собрание давно должно было закончится, но он надеялся увидеть директора и
предупредить хотя бы его о людоедских замыслах двух Филипонов.
Но оказалось, что директор "--- как впрочем и все остальные "--- уже давно ушёл
домой.

С одной стороны, надо было теперь ждать до завтра, с другой стороны, события
там, по ту сторону, очень его беспокоили, было слишком рискованно оставаться
здесь ещё целую ночь.
Поэтому он снова решил не ждать, а вернуться, но понял, что очень голоден и не
может сосредоточиться на математике пока не поест.
Он жил недалеко от института, так что заход домой не представлял большого труда
и не должен был его слишком сильно задержать.

Придя домой, он обнаружил, что холодильник был почти что пуст.
В нём стояла только бутылка молока, да и оно оказалось совершенно прокисшим,
хотя дата на бутылке ещё не была просрочена.
Он только в этот момент понял, насколько сильной была жара, даже холодильник
из-за неё плохо работал.

В буфете Омер нашёл банку растворимого супа марки
\enquote{\foreignlanguage{english}{Brownian Loop}},
того самого, который Младен когда-то не советовал есть Константину.
Обычно Омер был очень непривередлив в еде и его почти всё что угодно устраивало,
но всё же в такую жару есть консервированный суп у него не было сил.

У него начинала болеть голова, как бывало с ним почти всегда, если долго не
поесть.
Поэтому он всё-таки залил кипятком стаканчик растворимого супа, надеясь, что
ужин поможет от головной боли.
Но, размешивая ложкой комок из петелек плохо размокшей лапши, он понял, что не в
состоянии её есть.

Мигрень продолжала усиливаться, так что думать о математике стало совсем
невозможно, он с грустью подумал сквозь головную боль: принцесса, встреча с
премьер"=министром "--- и рухнул, не раздеваясь, на постель.
К счастью, мигрень на этот раз оказалась той разновидностью мигрени, которая
проходит, если лежать не шевелясь и ни о чём не думая.
Он заснул, и проспал всю ночь, и потом ещё всё утро, до самого полудня.

Он спал глубоким беспробудным сном без сновидений и не слышал шагов соседей
этажом выше, которые вставали несколько раз за эту ночь, чтобы достать из
холодильника холодные простыни "--- кто-то подсказал им этот действенный, но
не долгосрочный способ бороться с жарой, без него им вообще не удавалось уснуть
"--- он не слышал утром крики, смех, и лай во дворе, когда дети сторожа поливали
водой из пожарного крана лохматую дворняжку, он не слышал даже как соседка за
стеной его спальни очень громко включила телевизор, и там рассказывали о том
тоже, как поливают из пожарных шлангов, правда не собак, а стены атомных
электростанций.

Когда он, наконец, проснулся, он с радостью отметил, что мигрень совершенно
прошла.
Но вместо мигреневой тяжести в голове была теперь какая-то необычная горячая
пустота.
Он не стал завтракать, только попил воды и пошёл в институт, по дороге встретил
директора, мельком с ним поздоровался, и даже не вспомнил о том, что у него к
нему есть важное дело.
Директор хотел спросить Омера, почему тот пропустил вчерашнее собрание, возможно
упоминание о нём навело бы Омера на мысли о людоедстве, двух Филипонах и всём
остальном, но было жарко, у директора было много важных дел, останавливаться и
разговаривать ему не хотелось, поэтому он в итоге ничего не спросил, только
кивнул Омеру в знак приветствия и ушёл в сторону административного здания.

Омер же зашёл в научное здание.
В его кабинете тоже оказалось жарко, но всё же прохладнее, чем на улице, и мысли
потихоньку стали возвращаться.
Он с беспокойством подумал: Альсбета, бывший премьер"=министр, но ещё не успел
толком вспомнить, что именно ему предстояло сделать с премьер"=министром и в чём
именно была проблема с Альсбетой, как в дверь его кабинета громко постучали.
Он ответил своим обычным задумчиво"=рассеянным голосом: \enquote{Да, войдите},
после чего в кабинет вошёл ему незнакомый молодой человек и заявил, что хочет
рассказать Раджаферу о своей работе по теории струн.
Омер попытался возразить, что ничего не понимает в теории струн, почему бы ему
не поговорить лучше с профессором Франклином или профессором Белинским?
Но молодой человек уверенно отклонил эти возражения, сказал, что хочет
поговорить именно с ним, Раджафером, и начал увлечённо излагать свои последние
результаты.
Омер пытался вначале понять то, что он рассказывал, но это ему плохо удавалось,
к тому же в какой-то момент, заглушая слова рассказчика, в голове стало
раздаваться странное звяканье, \enquote{дзинь"=дзинь, трынь"=трынь"=трынь}.
Ему казалось, что этот звук ему очень хорошо знаком, но он не мог вспомнить
откуда, что-то продолжало звенеть не то жалобно, не то требовательно
\enquote{дзинь"=дзинь"=дзинь"=дзинь}.
У Раджафера начинала снова болеть голова, у него не было сил вспоминать, что
означает это позвякивание, и он лениво подумал: это просто звенят струны в
формулах на доске.
Действительно, молодой человек исписал тем временем своими формулами почти всю
доску, и Омер совсем уже перестал понимать то, что он писал.
Он хотел снова сказать молодому человеку, что ему всё-таки стоит поговорить с
профессором Франклином, но ему не удавалось вставить и слово в быструю и
увлечённую речь рассказчика.
Даже просто спросить, что обозначают символы на доске, за которыми он давно
перестал следить, не было никакой надежды.

Его спас приход системного администратора, который искал профессора Кон~Фу~дзе
и зашёл спросить, не видел ли его Раджафер сегодня утром в институте.
Кон~Фу~дзе Омер не видел, но он воспользовался этим разговором для того, чтобы
попрощаться с молодым физиком.
После чего облегчённо вздохнул и пошёл в библиотеку посмотреть на новые
поступления.

Как всегда по утрам в библиотеке было довольно много посетителей.
Только вот выглядели они как-то не так, как обычно.
Даже Раджафер, который по эту сторону не имел обыкновения обращать слишком много
внимания на окружающих его людей, заметил эту перемену.
Почти все, сидевшие в читальном зале, были какими-то подозрительно толстенькими,
некоторые даже почти кругленькими, и у многих на лицах был яркий розовый
румянец.
Нигде не было видно, столь часто встречающихся среди математиков, отрешённых
бледных лиц со впалыми сверкающими глазами.

\enquote{Я совсем потерял голову с этой жарой, "--- с внезапным ужасом подумал
Раджафер.
"--- Моррис и Мэйли!
Им таки удалось пригласить в институт наиболее упитанных и аппетитных визитёров!
Я думал, что решающее собрание было только вчера, но видимо, я что-то перепутал.
Или может быть два Филиппа, Мэйли и Моррис, сплели свои интриги так ловко, что им
удалось позвать этих людей в институт минуя общественное собрание.
Что же теперь делать?
Предупредить этих визитёров?}
Но нелепо подходить к незнакомым людям и говорить им: \enquote{Держитесь подальше
от профессора Мэйли и профессора Морриса}.
Тем более что они точно не станут его слушать, если это именно Мэйли и Моррис
их сюда пригласили, не говоря уж о том что он, Раджафер, имеет в институте
репутацию довольно чудаковатого типа.
Что-то всё-таки надо сделать.
Поговорить с директором?
Но разговоры с ним всегда требовали достаточной дипломатичности, и ему не
удавалось сейчас придумать, как следовало бы правильно изложить столь деликатную
ситуацию.
Вариант прийти и сказать: \enquote{Мэйли и Моррис могут съесть наших визитёров}
явно не подходил, а ничего другого ему не приходило в голову.

Вообще-то, с другой стороны, это была какая-то ерунда.
Ведь \emph{здесь} оба Филиппа вовсе не должны были быть людоедами!
\enquote{Здесь} "--- что значит \enquote{здесь}?
Мысли в голове у Омера путались.
У него было ощущение, что большая часть происходящих в последнее время событий
как-то связана с сегодняшней жарой и его головной болью, но каким образом "---
он не мог себе объяснить.
Как бы то ни было, с находящимися в опасности визитёрами надо было быстро что-то
делать.

Впервые за всю свою жизнь Омер жалел об ограниченности своего влияния в ИВНИ.
Он даже пробормотал себе под нос что-то вроде: \enquote{Вероятно, стоило всё-таки
тогда, 15~лет тому назад, опубликовать мой препринт о нулях дзета"=функций!}
Как бы то ни было, теперь об этом было поздно сокрушаться.
И, хотя Омер и так пользовался в институте немалым уважением, он чувствовал,
что сейчас этого недостаточно и что одному ему не справиться со сложившейся
ситуацией.

Может, поговорить с кем"=нибудь из наиболее надёжных профессоров?
Пару лет назад в институте было несколько человек, которым он мог доверять, но
большинство из них недавно уехали.
Оставался, правда, профессор Кон~Фу~дзе, но, во-первых, он был на пенсии, а,
во-вторых, он физик, и ему трудно вмешиваться в дела, связанные с чисто
математическими визитёрами.

И всё-таки он очень обрадовался, когда увидел через несколько минут входящего в
библиотеку Кон~Фу~дзе.

"--* Вас искал мистер ван~дер~Палс, опять не работают почти все компьютеры,
"--- сказал Раджафер, вскочив ему навстречу.
"--- И, к тому же, есть нечто гораздо более важное, чем компьютеры\ldots

У профессора Кон~Фу~дзе вид был такой, как будто он куда-то очень торопился.

"--* Я в курсе, и почти всё уже исправил, "--- ответил он Раджаферу, не дав
последнему закончить свою фразу.

"--* Но дело не только в компьютерах\ldots

"--* Я знаю про все проблемы, некомпьютерные тоже.
Не беспокойтесь, Омер, "--- тут Кон~Фу~дзе улыбнулся и, несмотря на свой
озабоченный вид, с совершенно несвойственной ему непосредственной и дружеской
интонацией добавил: "--- Так что вы можете идти и спокойно заниматься
математикой, Омер.

Последняя фраза звучала в его устах довольно странно, тем более что он произнёс
её с некоторым нажимом. Можно было подумать, что он зашёл в библиотеку только
для того, чтобы её сказать, потому что произнеся слова \enquote{Вы можете идти и
спокойно заниматься математикой, Омер}, Кон~Фу~дзе, не глядя ни на книжные, ни
на журнальные полки, стремительным шагом вышел из читального зала и скрылся за
поворотом коридора, ведущего к выходу из института.

\enquote{Что-то сегодня со мной не так, "--- в который раз подумал Раджафер.
"--- От жары совсем путаются мысли}.

Но он послушно вернулся в свой кабинет, сел за письменный стол, задумался "---
и на этот раз даже быстрее, чем обычно "--- щёлк, что-то вспыхнуло в его глазах,
и после этого его мысли были уже очень далеко и от жары, и от института, и от
всего, с ними связанного.
