\section{Последний день пребывания в институте}

Константин собирал разбросанные по комнате вещи: одежду, тетрадки, статьи и
книжки, и пытался засунуть их в свой рюкзак, что было непросто.

Последние дни погода стояла тёплая, поэтому его зимняя куртка уже была засунута
на самое дно рюкзака, и после этого остальные вещи решительно отказывались туда
помещаться.
Надевать куртку в такую теплынь, даже просто на время дороги в аэропорт,
совершенно не хотелось.
Но не нести же её в руках!
Может, удастся привязать её снаружи к рюкзаку?
Он вытащил куртку, положил на её место несколько книжек "--- они были самыми
тяжёлыми из его вещей "--- потом стал запихивать в рюкзак свои рубашки, свитер,
какие-то бумажки и статьи, и при этом обнаружил среди них копию своего препринта.
Везти его с собой никакого смысла не имело, ведь он всегда может заново его
распечатать.
Тем не менее он зачем-то стал перелистывать свою статью, и мысли его от сборов в
дорогу перешли на математику, а также на то обстоятельство, что через несколько
дней он собирался отправить свою работу на рассмотрение в научный журнал.

Статья была уже вполне для этого готова, но, стоило ему подумать, что вернувшись
в Корнелл он должен будет её отослать, как его охватывал страх, что что"=нибудь
в ней неверно.
Примерно такой же, как когда он только что доказал свою теорему.
Конечно, теперь уже всё было тщательно написано, и он уже рассказал нескольким
людям своё доказательство, но всё же: кто знает?
Может быть в его аргумент закралась какая"=нибудь тонкая ошибка, и тогда
наверняка, если уж он сам её не заметил, другим ещё труднее её обнаружить.
Он отпихнул ногой не собранный ещё до конца рюкзак и сел к столу, чтобы
повнимательней прочитать заключительный этап своего доказательства, потому что
именно он у него только что вызвал внезапные подозрения.
В этот момент раздался звонок.

\enquote{Ах да, это Младен, который должен был сегодня зайти со мной попрощаться!}

И Константин, едва не споткнувшись о лежащий на полу рюкзак, бросился открывать
своему другу дверь.

Это действительно был Младен.
Он вошёл в комнату и широко улыбнулся.
Вид у комнаты и в самом деле был довольно забавный: валяющиеся на стульях и на
диване не собранные вещи, брошенная на стол дутая зимняя куртка, какие"=то
листочки, разбросанные по полу, и, наконец, посередине всего этого беспорядка
Константин, с растерянным видом теребящий зажатую в руке распечатку своей статьи.

"--* Тут совсем уже надо надо собираться, но мне только что пришло в голову, что
у меня в доказательстве может быть ошибка, поэтому я сел это проверять, "---
несколько извиняющимся тоном сказал Константин.
"--- Вроде всё правильно, даже не знаю, отчего я вдруг так разволновался!
Но ничего, после того, как я доказал результат этой статьи, меня поначалу ещё
худшие сомнения мучили.
С тобой такого не бывает?

"--* Обычно нет, "--- сказал Младен, немного подумав перед тем как отвечать.
"--- Но у меня бывает другое: стоит что-либо доказать, как я начинаю бояться,
что это результат не нов, или что его кто"=нибудь одновременно со мной докажет.
Причём бывает очень смешно: докажу что"=нибудь совершенно ничтожное, и при этом
мне начинает казаться, что все вокруг обязательно тоже думают над той же задачей.
И ужасно обидно, что все тоже могут вот-вот догадаться до решения!
Головой понимаю, что на самом деле "--- дай бог чтобы она вообще хоть кого-то
заинтересовала, но, как обычно, разумные доводы в таких ситуациях не очень
помогают.
Хорошо ещё, что пока думаешь о математике, ничего такого не бывает.
Вообще, меня всегда интересовало, что с тобой происходит, когда занимаешься
математикой "--- ты никогда об этом не размышлял?
Я замечал, что пока думаешь о ней "--- не только все обычные сомнения и
переживания из повседневной жизни куда-то исчезают, но и вообще, как будто ты
становишься другим человеком при этом, или во всяком случае как будто бы при
этом живёшь по каким"=то совершенно другим законам, к обычной жизни никакого
отношения не имеющим \ldots

"--* Ага, я тоже это замечал, "--- ответил Константин.
"--- Например, часто очень трудно оценить время, которое проходит, когда думаешь
над какой"=нибудь задачей.
И вообще иногда кажется, что находился при этом в каком"=нибудь другом месте.
Я тебе не рассказывал историю про уборку риса?
Странно, мне казалось, что я её уже рассказывал.
Этой осенью, когда я думал над своей задачей "--- ещё задолго до того, как мне
удалось найти её решение "--- я часто потом не мог вспомнить то, о чём я думал
весь день.
А если я пытался вспомнить, то воспоминания выплывали очень странные: мне
мерещилось, что я весь день, по колено в воде, работал на рисовом поле.
Наверное, это было от того, что я очень уставал, но всё равно странно.
Ведь рисовых полей я, понятное дело, никогда на самом деле не видел, да и вообще
в рисе почти ничего не понимаю.
И, знаешь что?
Очень забавно, мне только сейчас это пришло в голову!
Перед отъездом в институт я заболел, и врач мне сказал, что у меня симптомы
болезни, которая бывает, если питаешься одним только очищенным рисом "--- может
быть в этом есть какая"=нибудь связь с тем рисом, который мне мерещился во время
моих занятий математикой?
То есть, я, конечно, шучу, но с другой стороны я где-то читал, что если долго
думать о чём"=нибудь, связанном с болезнью, то могут появляться симптомы этой
болезни, даже если самой болезни при этом нет.
Но с рисом это в любом случае не то же самое, всё-таки \ldots\
Знаешь, вот я сейчас тебе рассказал, и при это вспомнил ещё ясней, чем до этого:
огромное такое рисовое поле, солнце печёт над головой, и мысль, что работать
придётся ещё до самого вечера.
Пожалуй, всё это мне мерещилось в те дни, когда мои математические размышления
были наименее продуктивными \ldots

"--* Интересно, таких воспоминаний, как у тебе о уборке риса, у меня никогда не
было.
Вообще, мне, наверное, ещё хуже, чем тебе, удаётся вспомнить что-либо,
происходившее в то время, пока я думал о математике.
Даже на уровне ощущений.
Я пытался проследить, как это происходит, но мне удалось запомнить только
следующее.
Иногда думаешь, как будто бы не всерьёз, или не удаётся толком
сконцентрироваться, или просто настроение какое"=то не достаточно
математическое.
О таких размышлениях ещё остаются какие-то воспоминания.
Но бывает, начинаешь думать, и потом в голове как будто бы какой-то щелчок
раздаётся, или иногда не щелчок, а скорее что-то вроде мгновенной вспышки, и
после этого уже полностью уходишь в какой"=то математический мир, о котором
почти не остаётся никаких воспоминаний.

"--* Да, а потом думаешь, думаешь, но в какой-то момент как будто бы
просыпаешься, и тогда снова оказываешься в обычном, нематематическом мире!
"--- поддержал Младена Константин.

"--* Ты что, и вправду помнишь ощущения, которые бывают перед тем как, до этого
думая о математике, внезапно перестаёшь о ней думать? "--- не поверил ему Младен.

"--* Да нет, "--- несколько растерянно сказал Константин.
"--- Я сам не знаю, почему я это только что сказал.
Как-то само слетело с языка \ldots

Они ещё долго болтали с Младеном, потом, наконец, спохватились, и Младен помог
Константину кое-как запихать его вещи в рюкзак.
Только после этого Константин посмотрел на часы и понял, что уже здорово
опаздывает.
С рюкзаком за плечами (и зимней курткой в руках!) он бегом побежал в институт.
Перед входом на его территорию он нос к носу столкнулся с какой"=то девушкой,
которая посмотрела на него и очень отчётливо улыбнулась.

"--* Вот черт, ведь не часто так бывает, чтобы на улице тебе так улыбались
такие хорошенькие девушки, "--- подумал Константин.
"--- И как назло, происходит это тогда, когда надо ужасно торопиться!

Девушка тем временем несколько преградила ему дорогу и продолжала улыбаться.
Только теперь Константин вспомнил, что он был с ней знаком: он несколько раз
видел её в институте, кажется, она занималась прикладной математикой, и ещё один
раз во время обеда он даже разговаривал с ней о Греции и о том, как ей нравятся
Афины.

\enquote{Неужели я становлюсь таким как Раджафер?
Скоро совсем знакомых перестану узнавать, вот до чего доводят занятия
математикой}.

Девушка, не переставая улыбаться, сказала:

"--* Привет, я хотела тебе нечто предложить.
Языковой обмен.
Ты бы не мог учить меня греческому?
Я, кроме английского, знаю ещё испанский и португальский, может, тебя это
интересует?
Правда, вообще-то я больше бы хотела учить древнегреческий.
Но я хотела тебе спросить: новогреческий и древнегреческий действительно сильно
отличаются?

"--* Ну, \enquote{бета} читается как \enquote{вэ}, а не как \enquote{бэ}, и есть
ещё несколько небольших отличий, "--- сказал Константин, пятясь спиной по
направлению к воротам на территорию института.
Он помнил, что ему надо очень торопиться, но при этом лицо его всё ещё было
повёрнуто в сторону девушки "--- в конце концов было бы невоспитанно убежать,
ничего ей не ответив.

Та немного обиделась и сказала:

"--* Я ведь серьёзно говорю \ldots

"--* Я тоже серьёзно, "--- ответил Константин.
"--- Но только с обменом к сожалению не получится, потому что я сегодня уезжаю.

И он побежал в сторону института "--- на страшной скорости.
Но всё-таки итоге он изрядно опоздал на встречу с профессором Варангом, и тот
был явно этим очень не доволен.
Они стали говорить, но во время разговора он слушал Константина не слишком
внимательно и всё время поглядывал на свои наручные часы.
Константин, конечно, был очень расстроен.
Во-первых, он надеялся сегодня получить от Варанга напоследок ещё несколько
полезных советов по поводу своей задачи.
Во-вторых, ему бы хотелось, чтобы у Варанга осталось о нем хорошее впечатление.
И тогда "--- кто знает? "--- тот может быть пригласил бы его ещё раз приехать в
институт.

Но Варанг хмурился, листал, вместо того чтобы слушать Константина, свою
последнюю статью, и в конце концов молодой человек понял, что беседа не удалась
и что пора уходить.

Он вежливо попрощался "--- Варанг в последний момент отвлёкся от своей статьи,
сказал в напутствие Константину нечто довольно приветливое, закрыл за ним дверь
своего кабинета и облегчённо улыбнулся.

Он ещё раз глянул на часы, хотя и знал, что это совершенно бесполезно "--- время
по ту и по эту сторону текло совсем по"=разному.
Он подошёл к доске, написал какую"=то формулу, задумался, рассеянно обвёл
взглядом свой кабинет "--- полки с книгами, письменный стол, фотографию белой
лошадки над столом "--- снова подошёл к доске, попытался исправить написанную
им ранее формулу, но что-то не сходилось.
Он вспомнил, что не совсем правильно домножил левую часть на группу
Вейля"--~Делиня, надо было добавить поправочный фактор, он исправил ошибку,
написал ещё несколько похожих выражений, провёл стрелочки так, что всё вместе
превратилось в довольно внушительную коммутативную диаграмму.
В этот момент в глазах его что-то вспыхнуло, и он зажмурил их на мгновение,
стараясь получше обдумать то, что он только что сам написал.

Он открыл глаза, но утреннее солнце было очень ярким, он закрыл их снова,
пытаясь понять, почему он проснулся так рано.
В этот момент раздался стук в дверь, на этот раз совсем громко, ему пришлось
выскочить из постели, одеться на ходу и поскорее открыть дверь нежданному
посетителю.

Оказалось, что это был почтальон.
Обычно он опускал письма в почтовый ящик при входе в дом, но сейчас письмо было
заказным "--- и к тому же отправлено из канцелярии R.~ле~Кина "--- поэтому
почтальон хотел, чтобы Гвенаэль получил его лично и расписался о получении.

Как с ним часто это бывало после того, как внезапно проснёшься, у Гвенаэля всё
немного путалось в голове, и первые минуты после сна он иногда не чётко помнил,
где именно он находится.
Поэтому вначале вместо своей подписи он попытался изобразить в толстой тетради
почтальона нечто вроде группы Вейля"--~Делиня от $GL(n)$ над функциональным
полем.
Но он вовремя спохватился, старательно зачирикал группу Вейля и расписался как
должно, потом проводил почтальона и только после этого открыл большой блестящий
конверт со штампом императорской канцелярии в правом верхнем углу.

Листая письмо, он уже полностью забыл про функциональные поля, про группы
Вейля"--~Делиня и про всё остальное, что происходило или могло происходить по
другую сторону сна.
Потому что письмо сообщало о событии, которое было крайне важным и почётным для
Гвенаэля, и хотя Филипон уже намекал ему о нем во время их совместного обеда в
\enquote{Жирной похлёбке}, всё же в некотором смысле это событие было для него
неожиданностью.

В письме говорилось, что Гвенаэль назначается главным придворным писателем, и
что по этому случаю ближайшим летом в императорском дворце состоится большой
торжественный приём.

"--* Чтобы там ни говорили всякие простаки, но приходит час, когда настоящая
литература получает должное признание, а развлекательная "--- она так навсегда
и остаётся развлекательной! "--- торжествовал Гвенаэль.
До этого момента в глубине души он был почти что уверен, что главным придворным
писателем назначат Омера, книги которого, и до этого значительно более читаемые,
чем книги Гвенаэля, последнее время приобретали всё большую и большую
популярность.

Но теперь, вот оно, перед ним, письмо, написанное золочёными буквами и даже с
подписью самого императора на последней странице! "--- и не оставалось никакого
сомнения в том, что главным писателем будет отныне именно он, Гвенаэль.
Он даже на всякий случай внимательно осмотрел конверт, нет ли тут какого"=нибудь
подвоха, но нет, письмо действительно отправлено из императорской канцелярии, и
адрес на конверте стоит именно его: Гвенаэль Варанг, улица Нелокальных
Возмущений, дом~45.
Да, именно улица Нелокальных Возмущений, а не какой"=нибудь, как Вы могли
подумать, Кошачий переулок "--- сколько бы в этом Кошачьем переулке не жило
творцов легковесных и развлекательных бестселлеров!

Где-то в самой глубине его подсознания шевелились гораздо менее приятные мысли о
том, что дело могло быть вовсе не в том, что книги у Омера слишком
развлекательные, а просто в том, что он не пользовался достаточной
благосклонностью новой власти.
И ещё одна мысль о том, что Альсбете, с которой он только недавно помирился и
уже почти договорился наконец после примирения о дате их предстоящей свадьбы,
вряд ли очень сильно понравится его будущий титул.

Но этим мыслям Гвенаэль не давал вылезти дальше самого глубокого уровня
подсознания, а все другие уровни и сознания, и подсознания были заняты одним и
тем же, довольно простым содержанием:

\enquote{Быть главным придворным писателем "--- это огромная честь, быть главным
придворным писателем "--- это очень"=очень здорово, поскорее бы состоялся приём
в императорском дворце, чтобы все поскорее узнали о необычайном успехе Гвенаэля!
Впрочем, новость эта наверняка быстро распространится ещё и до официальной
церемонии во дворце, потому что новость эта важная и почётная, потому что быть
главным придворным писателем важно и очень"=очень здорово, и ничто не помешает
его радости в этот так хорошо начавшийся для него день!}
