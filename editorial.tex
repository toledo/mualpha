\phantomsection
\chapter*{От редакции выпуска}
\addcontentsline{toc}{chapter}{От редакции выпуска}

\begin{quotation}
Математика "--- это наука, имеющая применения в различных областях, таких как
химия, физика, экономика и социология.
За последние два с половиной тысячелетия математика пережила стремительное
развитие.
В этой книге обсуждаются наиболее старые её результаты, такие как теорема
Пифагора, более современными, но всё же весьма элементарные (построение
правильного 17"=угольника), но также и уже введённые в двадцатом веке объекты и
понятия (сфера Александера, множество Мандельброта).
Вкратце затрагиваются активно развивающиеся в наше время области (теория струн,
конформная теория поля, программа Ленглендса).

Книга предназначена для школьников, студентов, исследователей и преподавателей
математики, а также для всех интересующихся этой наукой.
\end{quotation}

Закончив писать последнюю фразу, заведующий научно"=популярной серии оторвался
от своего компьютера и встал из-за стола, приветствуя входящего в его кабинет
технического редактора.

"--* Как, Вы уже написали аннотацию для новой книги? "--- спросил технический
редактор.

Заведующий научно"=популярной серии в ответ недовольно махнул рукой.

"--* Конечно, вот только что кончил писать.
Это дело недолгое.
Всегда примерно одна и та же дребедень.
Несколько строчек, которые все прочтут, потому что они будут напечатаны на
обложке жирным шрифтом, но из которых вряд ли много что можно узнать.

"--* А что бы Вам действительно хотелось сказать будущим читателем этой книги?
"--- спросил технический редактор.
"--- Представьте, что Вам не обязательно следовать стандартному стилю, что бы Вы
тогда написали?

"--* То есть совсем"=совсем забыв про стандартный стиль? "--- спросил, заметно
оживившись, заведующий научно"=популярной серии.
"--- Если бы действительно совсем отменили обязательный стандарт для аннотаций,
я бы написал, наверное, что-нибудь в таком роде.

\medskip
Эта книга "--- попытка описания математического мира.
Этот мир "--- огромный и удивительный, самой толстой на свете книги не хватило
бы, чтобы его полностью описать, так что не удивляйтесь, что в эту книгу попали
только крохотные его кусочки.
Чтобы узнать о нем побольше, надо самому в него попасть, а это совсем не просто.
Если вы никогда в нем не были, особенно если вы ещё молоды, я очень вам советую
попытаться это сделать!

Вы никогда раньше не слышали об этом мире и о вообще о математике почти ничего
не знаете?

Или, может быть, о математике Вы кое-что уже знаете, но полученные вами знания
приносят скуку и усталость, и Вы сильно сомневаетесь в том, что думая о
математике, можно попасть в совершенно другой прекрасный и увлекательный мир?

Тогда у меня для вас есть один совет.
Я сам отлично знаю, иногда так бывает: сидишь на лекции, профессор или докладчик
пишет на доске какие"=то запутанные формулы, смысл которых от Вас совершенно
ускользает.
А за окном светит солнце, так тепло и ярко, и хочется выскочить опрометью из
аудитории, и пойти гулять по нашему обычному миру, освещённому этим ласковым
весенним солнцем, такой бессмысленно скучной и блеклой кажется по сравнению с
ним непонятная и непонятно для чего нужная лекция \ldots

Так вот, именно на такой случай у меня для вас есть совет.
Возьмите зеркальце, обычное небольшое зеркальце, поймайте в него солнечного
зайчика, и, когда лектор отвернётся от слушателей, пустите прыгать вашего
зайчика по доске.
Поторопитесь, пока лектор этого не замечает, постарайтесь направить зайчика на
самую сложную и запутанную формулу.

Вас, наверное, удивит, что, вспыхнув на мгновение в сплетение каких-то
непонятных обозначений с ещё менее понятными двойными индексами, ваш зайчик
куда-то неожиданно пропал?

Знайте, что он не исчез, просто он скачет теперь где-то в лесу под Белой горой,
по крыше императорского дворца, по лестнице, ведущей к старой аптеке или просто
по набережной реки Кортевеговки "--- да, Ваш солнечный зайчик уже попал в тот
другой, математический, пока, возможно, ещё не доступный для Вас мир, и может
быть Вам когда"=нибудь тоже удастся в него проникнуть.
