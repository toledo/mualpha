\section{Запись из тетрадки профессора Варанга}

Среди людей, ожидающих на перекрёстке двух больших улиц зелёного сигнала
светофора, стоял невысокого роста молодой человек, светловолосый, довольно худой,
подчёркнуто скромно одетый: серые брюки, короткое тёмное пальто, очки в простой
коричневой оправе, которая вышла из моды по крайней мере лет десять тому назад.
Этот молодой человек был ни кто иной, как профессор Варанг.

Красный свет сегодня как-то особенно долго не хотел переключаться на зелёный.
Варанг бросил взгляд на свои наручные часы, пытаясь понять, успевает ли он на
ближайший поезд метро в направлении Юг--2.
Видимо, он ещё успевал, в любом случае, машины неслись с такой скоростью, что
переходить этот перекрёсток не по переходу в это время дня не было ни малейшей
возможности.

Наконец, поток машин остановился, красный человечек в светофоре напротив погас,
уступая место зелёному, и Варанг почти что бегом пересёк широкую улицу, обогнул
стоящий на углу кинотеатр, миновал мексиканское кафе, детскую площадку, магазин
спортивной обуви, и после этого немного замедлил шаг, считая, что теперь он уже
точно не может опоздать.

Подходя к метро, он понял, что что-то сегодня было не так.
У входа на станцию и на ближайшей автобусной остановке стояла огромная толпа.
Часть людей пыталась втиснуться в подошедший автобус, но он и так уже был забит
почти что до отказа.

Что случилось?
Не работает метро?
Забастовка?
Но тогда, скорее всего, автобусы тоже бы не ходили.
К тому же о забастовках обычно объявляют заранее, а он, вроде бы, ничего про это
не слышал.
И потом сейчас, в один из первых дней наступившего нового года время для них
было не самое подходящее.

"--* Что случилось, метро не работает? "--- спросил Варанг у выходящих со
станции людей.
Была такая толкучка, что ему не сразу удалось привлечь к себе внимание, но в
конце концов одна женщина ему ответила:

"--* Несчастный случай, поезда не ходят.
Уже по крайней мере полчаса\ldots

"--* Совсем не ходят?
По всем веткам?

"--* Да, да, всё движение совсем остановлено\ldots

Тут ему пришлось посторониться, чтобы вываливающиеся из метро люди его не
раздавили.

\enquote{Обидно, если сегодня не удастся попасть в институт, "--- думал Варанг.
"--- Ничего особенно важного на сегодняшний день не намечено, но я уже и так
почти целую неделю там не был.
И потом я обещал этому греческому аспиранту, Димитриакису, что я сегодня с ним
поговорю.
Он, кажется, уже несколько дней находится в институте по моему приглашению}.

Он решил узнать, нет ли надежды на то, что в ближайшее время движение
возобновится.
Правда, когда случалось что-либо непредвиденное, такого рода информацию бывало
непросто получить.
Он с огромным трудом протолкался через толпу у входа на станцию, и оказалось,
что сделал он это не зря.

"--* Несчастный случай на 16-ой ветке, "--- громко объявил репродуктор.
"--- Всё движение полностью остановлено на неопределённый срок.
9-ая ветка: поезда ходят, но с заметным опозданием.
Движение по 25-ой ветке практически нормальное.

Так что с идущей в институт 25-ой веткой никаких проблем не предвиделось.
Только вот пройти на нужную платформу было не очень просто: повсюду была
изрядная сумятица, пассажиры нервно ходили туда"=сюда по станции и внимательно
прислушивались к голосу репродуктора, говорил он громко, но довольно
неразборчиво:

"--* Несчастный случай на 16-ой ветке, всё движение по 16-ой ветке полностью
остановлено\ldots

Варанг, как и любой другой городской житель, отлично понимал, что
\enquote{несчастный случай} на этом метрошном языке означает самоубийство.

В любом большом городе происходят они довольно часто, в среднем не реже, чем раз
в день.
Большинство из них остаются совершенно незамеченными, но вот такие, в метро,
вызывают пару часов толкотни и неразберихи, прикрываемой тактичным эвфемизмом
\enquote{несчастный случай}.
Будет немало людей, опоздавших на работу или на более или менее важные свидания,
может быть, какой"=нибудь ребёнок потеряет в суматохе свою любимую игрушку:
куклу или плюшевого мишку, но через несколько часов поезда пойдут снова, и уже к
вечеру почти всё забудется.
Почти всё, кроме, разве что, потерянной любимой игрушки.

Размышляя о плюшевых мишках, несостоявшихся свиданиях и статистике самоубийств в
больших городах, Варанг проталкивался сквозь недовольную хаотично двигающуюся
толпу, и в конце концов ему удалось пройти на нужную ему двадцать пятую ветку.
Оглядев платформу, он убедился в том, что народу на ней ничуть не больше чем
обычно.
Это было не особенно удивительно, так как пассажирам неработающей шестнадцатой
ветки и плохо работающей девятой эта двадцать пятая ветка мало чем могла помочь.
Во всяком случае поезда этого её направления, через несколько остановок
выходящие из города и уходящие в лесопарк, в котором находился институт высших
научных исследований.

Так что было неудивительно, что платформа, как обычно бывает в это время дня,
была полупустой.
Несколько человек, едущих на работу, несколько студентов, небольшая группа
подростков, два или три туриста, любовная парочка: парень, засунувший руку в
задний карман джинсов своей девушки.
И ещё вот нищая старуха, просящая милостыню.

Это была довольно высокая грузная женщина, не то восточная, не то африканка,
сказать было трудно, потому что вся она была обмотана пёстрыми цветастыми
тряпками так, что даже лицо её было плохо видно.
Не обращая внимание на весело щебечущих о чём-то школьников, она двигалась
сейчас по платформе в сторону Варанга, двигалась медленным, но уверенным шагом,
заранее держа перед собой протянутую руку.

У Варанга был такой принцип: он давал деньги всем встречавшимся ему нищим.
А может быть, дело было не в принципе, может быть, просто он был достаточно
добрым и отзывчивым человеком "--- неясно.
Как бы то ни было, он давал деньги всем категориям встречавшимся по улице или в
метро нищим или бездомным: местным и иностранцам, женщинам и мужчинам,
жалующимся на болезни старикам или потерявшим работу молодым людям.
И даже тем молчаливым беженцам, кладущим перед тобой записочку с однообразным
содержанием:

\enquote{Я осталась (или остался) без крова, у меня 2 (3, 4 или 5) детей,
помогите мне, чтобы я смогла (смог) найти работу и прокормить свою семью.
Да хранит вас бог, вас и ваших ближних}.

Количество детей обычно вставлялось от руки согласно ситуации:
у людей постарше 3 или 4, у молодых 1 или 2, в остальном же это был один и тот
же печатный текст, размноженный с помощью ксерокса и не менявшийся в течении
многих лет.
Неясно, для чего эти люди ходили по вагонам метро, им почти никогда никто ничего
не давал.
Было также не совсем понятно, откуда именно брались эти люди и насколько вообще
им можно было помочь, но для Варанга не было исключений.
Он считал нужным давать немного денег каждому, действительно каждому, кто
попросит его помощи.

В правом кармане своих брюк он всегда носил для этой цели большое количество
мелочи.
Именно в правом, чтобы знать точно, где она лежит и не искать долго, когда
понадобится "--- потому что его очень смущало, если окружающие замечали, что он
старательно обшаривает свои карманы в поисках очередной монетки.
Он испытывал в такие моменты чувство, похожее на стыд, хотя казалось бы никаких
причин стыдиться у него не было.

Сейчас, как обычно, он стал доставать мелочь, и, как это нередко случалось,
правило правого кармана не очень сильно ему помогало.
Даже если мелочь и находилась, как ей это было положено, именно в этом правом
кармане, это вовсе не означало, что там не было ничего другого.
Сегодня, к примеру, там было немало бумажек, исписанных математическими
формулами, старых трамвайных билетиков, фантиков и обёрточек от кускового
сахара, который дают к кофе или чаю в кафе или в институтской столовой.
Так что он довольно долго рылся в содержимом своего кармана, прежде чем ему
удалось извлечь оттуда пару монеток.

Он собрался отдать их подходящей к нему старухе, но тут оказалось, что женщина
не протягивала руку за милостыней, а скорее указывала своей рукой в сторону
Варанга, и когда он это наконец понял и удивлённо поднял на неё глаза, она не то
закашлялась, не то засмеялась, из её гортани стали доносится странные глухие
звуки, и Варангу показалось, что сквозь этот смех или кашель она пробормотала:

"--* Вот идёт профессор математики, который в этом году прославится на весь мир,
ха-ха-ха-ха-ха!

\enquote{Трудно было представить, чтобы старая незнакомая ему женщина в метро
могла бы произнести эту странную фразу, просто невозможно себе это представить},
"--- пытался убедить себя Варанг.
И всё-таки ему было не по себе.
Он быстро оглянулся по сторонам "--- никто другой не обратил на эту загадочную
реплику никакого внимания.
Что, впрочем, было неудивительно, так как разобрать её было довольно трудно,
потому что говорила женщина с ужасным акцентом, а конец её фразы совсем потонул
в её глухом и прерывистом кашле.

К счастью, в этот момент подошёл поезд метро.
Варанг запрыгнул в него, стараясь больше не оборачиваться на оставшуюся стоять
на платформе нищенку, но даже в вагоне он всё ещё слышал у себе за спиной её
глухой гортанный смех.
И даже когда поезд уже отъехал от станции, в его ушах всё ещё звучали хриплые
слова:

"--* Который прославится в следующем году на весь мир, хо-хо-хо, ха-ха-ха,
хе-хе-хе!

Слова эти вызвали какое-то чувство в его груди, как будто там что-то сжималось и
было немного больно, но в то же время сладостно и приятно.
Он старался выкинуть их из головы, потому что они подтверждали одну важную и
хорошую новость, которую он знал уже некоторое время, но о которой ему ещё не
положено было знать.
А раз ещё не было положено знать, то он старался из суеверных соображений о ней
не думать, но фраза нищенки всплывала у него в сознании всё снова, и снова,
против его воли:

"--* Вот идёт профессор математики Варанг, который прославится в этом году на
весь мир!

Слова эти были, конечно, очень приятными, но Варанг не мог так же забыть и смех,
следующий за ними "--- или, может быть, это всё-таки был только кашель?

Смех этот что-то напоминал, он не сразу вспомнил что, но когда вспомнил, чувство
в его груди стало гораздо менее сладостным.

"--* Ах, да, Шекспир, "--- пробормотал Варанг себе под нос.
Кашель старухи напомнил ему о его заветной детской мечте: стать великим
писателем.
Настолько замечательным и талантливым, чтобы все вокруг могли воскликнуть: по
сравнению с произведениями Варанга даже Шекспир "--- это ерунда, сущая ерунда!

И если надежды Варанга прославиться своими математическими теоремами не были,
пожалуй, необоснованными, то до того, чтобы стать великим писателем ему пока что
было далеко, очень далеко\ldots

Уже после того, как поезд метро скрылся в туннеле, Раджафер подошёл к краю
платформы и подобрал клочок бумаги, который Варанг до этого уронил, ища в своём
кармане мелочь для милостыни.
Он всё это время был на станции и, оставаясь незамеченным Варангом, наблюдал
сцену, произошедшую между ним и старухой.
Впрочем, по его обычному задумчивому выражению лица трудно было сказать,
действительно ли он до этого за ними наблюдал, или думал о чём-то своём.
Так вот, когда поезд уехал, он подобрал скомканный тетрадный листочек,
оброненный Варангом, повертел его некоторое время в руках, сел в подошедший
очень скоро следующий поезд, и уже только в вагоне развернул и прочитал то, что
было написано на листке.
Зная Варанга, можно было ожидать, что на клочке бумаги будет какое"=нибудь
сложное вычисление.
Но нет, листок был расчерчен надвое неровной вертикальной линией, слева и справа
от которой находились имена или должности людей, а именно:

\begin{table}[h!]
\centering
\begin{tabular}{r|l}
Директор             & министр Пьерро \\
Профессор Кон~Фу~дзе & бывший премьер"=министр \\
библиотекарь         & аптекарь \\
??                   & Альсбета \\
Профессор Моррис     & ?? \\
\end{tabular}
\end{table}

И далее шёл ещё целый десяток имён, причём слева большинство имён принадлежало
людям, имеющим то или иное отношение к институту высших научных исследований, а
часть имён справа были довольно странными.

Раджафер открыл свой портфель, достал из него плохо отточенный карандаш и, сам не
зная для чего, заполнил недостающими именами встречающиеся в списке
вопросительные знаки.
После чего задумчиво покачал головой, сложил листок в несколько раз и спрятал
его вместе с карандашом обратно в свой портфель.
