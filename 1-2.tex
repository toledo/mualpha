\section{Греческий текст}

Набережная Кортевеговки была пустынной.
Может быть оттого, что только что прошёл дождь.
Он и сейчас не совсем ещё закончился, время от времени падали отдельные капли,
и небо было затянуто густыми серыми тучами.

Гвенаэль хотел сесть на скамейку около реки, но она была очень мокрой.
Он стал оглядываться по сторонам "--- нет ли где поблизости скамейки посуше, и
увидел идущую по набережной женщину.

\enquote{Альсбета!} "--- его мгновенно пронзило это странное чувство узнавания,
похожее на сильную радость и на резкую боль одновременно, но уже секундой позже
он понял, что ошибся "--- женщина была ниже ростом, и волосы у неё были более
тёмными, так что даже издалека было ясно, что это была не Альсбета.
Просто одета она была в осеннее пальто в крупную серую клетку "--- точно такое
же, как было у Альсбеты.

Вообще, с этими серыми клетчатыми пальто было сплошное мучение: этот серый
клетчатый драп был недорогим, но достаточно модным, и огромное количество
девушек ходили вот в таких серых демисезонных пальто, и почти каждый раз,
увидев его, Гвенаэль вздрагивал и тут же мучительно разочаровывался, если это
была не та, которую он так хотел увидеть.
Странно, что хотя ошибка эта повторялась довольно часто, пронзительное
ощущение, говорившее ему каждый раз \enquote{Альсбета}, ничуть от этого не
притуплялось.

Сухую скамейку ему так и не удалось найти, но зато около закрытой в это время
года лодочной станции он обнаружил навес, встал под него и начал разглядывать
немного мутную из-за дождя воду Кортевеговки.
Если бы вода была немного более прозрачной, то наверняка можно было бы
разглядеть живущих в ней разноцветных рыбок, но сегодня их не было видно.

Гвенаэль задумчиво пошарил у себя в кармане, вынул небольшой кубик кускового
сахара, развернул его и положил себе в рот.
И тут же понял, что сделал глупость: мучившая его с утра зубная боль, до этого
почти утихшая, тут же возобновилась с новой силой.
Он выплюнул сахар, но зуб продолжал неприятно ныть.
Эта была не самая сильная боль, её вполне можно было терпеть, но всё-таки
достаточно противная.

Чтобы отвлечься, он стал размышлять о разных вещах: о прочитанном им недавно
романе Омера, о своём незаконченном романе, о предстоящей сегодня встрече со
старшим Филипоном.
Ещё он думал о том, что вечером надо будет зайти к принцессе, что он не видел
её со вчерашнего утра, и что это ужасно долго.
Правда, думать об Альсбете сквозь зубную боль ему не нравилось, но он всё равно
думал: о том, как надо поскорее разделаться с Филипоном, вернуться домой,
написать пару глав своего нового романа, и после этого, наконец, можно будет к
ней пойти.

И тут его взгляд, задумчиво скользя по поверхности реки, наткнулся на
солнечного зайчика.
Яркое солнечное пятнышко сделало несколько прыжков по воде, выскочило на
набережную и стало метаться взад"=вперёд по мокрой мостовой между лодочной
станцией, около которой стоял Гвенаэль, и Большим Горбатым мостом.

\enquote{Странно, откуда он взялся? "--- подумал Гвенаэль.
"--- Ведь солнца совсем не видно!}

Он оглядел небо, оно по-прежнему было серым, солнце даже и не думало
проглядывать.
Он опустил глаза на мостовую: солнечный зайчик был по-прежнему тут как тут.
Потом, через несколько мгновений, ему надоело скакать на одном и том же месте,
и он стал удаляться от Гвенаэля, попрыгал некоторое время по ступенькам
лестницы, ведущей к аптеке, и потом куда-то скрылся.

\enquote{Аптека, "--- сказал сам себе Гвенаэль.
"--- Наверное, стоит в неё зайти и попросить какое-нибудь средство от зубной
боли}.

Он не торопясь пошёл вдоль реки, проделал тот путь, который только что совершил
до него солнечный зайчик, и стал подниматься по лестнице: две ступеньки, три
ступеньки побольше, поворот, ещё пять ступенек, ещё поворот и ещё семь мокрых
от дождя и очень скользких сегодня ступенек.
На крылечке, прямо в центре выложенного лазурным песчаником правильного
семнадцатиугольника, поблёскивала небольшая лужа.

Стараясь не наступить в неё и не промочить и без того уже довольно мокрые
ботинки, Гвенаэль толкнул обитую тёмным деревом дверь и вошёл в здание аптеки.

Ещё из прихожей он услышал, что он не единственный посетитель, и, войдя в
главный аптечный зал, увидел аптекаря, разговаривающего с какой-то невысокого
роста дамой, полненькой, почти что кругленькой, одетой в довольно смешное
ярко"=зелёное пальто.
На голове у дамы была зелёная шляпка, украшенная большими разноцветными перьями.

По всей видимости, разговаривали они уже довольно давно, перед посетительницей
уже стояла целая груда коробочек со всевозможными снадобьями, она совсем было
собралась уже за них заплатить и уйти, но тут вспомнила ещё об одном деле.

"--* Самое главное чуть не забыла, "--- сказала она аптекарю, и тот в ответ с
любопытством завращал своими большими немного выпуклыми глазами.
У него было такое свойство: он очень любил болтать, но по долгу службы ему
приходилось не только самому что-нибудь рассказывать, но и выслушивать своих
посетителей.
В те минуты, когда надо было слушать, и, сдерживаясь, самому ничего не
говорить, у него всегда от напряжения начинали вращаться глаза.
Впрочем, когда он что-либо рассказывал, глаза у него обычно вращались с ещё в
два раза большей скоростью.

"--* Самое главное чуть не забыла, "--- сказала дама в зелёной шляпке.
"--- Ещё мне обязательно нужно купить лекарство для моего старшего сына.
У него эта новая, непонятная болезнь, вы знаете, говорят, что не опасная, но
всё-таки я очень волнуюсь \ldots

\enquote{Начинается, "--- подумал про себя Гвенаэль.
"--- Так она вообще никогда отсюда не уйдёт.
А мне нужна всего только одна микстура от зубной боли!
Не факт, конечно, что она поможет, но всё-таки так хотелось бы, чтобы помогла}.

Пока дама продолжала описывать странную болезнь, которой, по её мнению, страдал
её старший сын, Гвенаэль от нечего делать рассматривал аптечные полки: почти
все стены, от пола до потолка, были заставлены большими фарфоровыми банками с
разноцветными надписями: шалфей, лаванда, корни подорожника.
Многие надписи были сделаны на латыни, и часть названий была совсем незнакома
Гвенаэлю.

"--* Знаете, "--- говорила тем временем посетительница, "--- он почти всё время
задумчивый и мечтательный.
Часто засыпает, и сразу ясно, что это необычный сон "--- бывает, даже
посередине дня, задумается над раскрытой книгой или чашкой недопитого чая, и
вдруг как будто бы начинает дремать, но вид у него при этом такой, как будто он
не просто дремлет, как будто бы он не здесь, а где-то совсем в другом месте,
далеко-далеко отсюда.

"--* Ясное дело, это сонная меланхолия, "--- не выдержав, аптекарь не слишком
учтиво перебил рассказ толстенькой дамы в зелёной шляпке.

"--* Да, я от соседок уже слышала это название, "--- подтвердила
посетительница.
"--- Говорят, что это не опасно, но я очень, очень беспокоюсь.
Вы уж посоветуйте что-нибудь по-настоящему помогающее.
Никакие слабые средства, вроде ромашки или корня подорожника, вы мне не
предлагайте, дайте мне сразу что"=нибудь действительно серьёзное, порошок из
толчёной мумии, или что вам там виднее будет \ldots\
Неважно, если лекарство будет дорогим, ради родного сына чего только не
сделаешь!
Так что ничего, если лекарство дорогое, в определённых пределах, я имею ввиду.

"--* Да вы не беспокойтесь, сонная меланхолия "--- это сущий пустяк! "---
сказал аптекарь, вращая глазами и одновременно шевеля своими большими
красноватыми ушами.
Он знал, что некоторые посетители, особенно если они видят его впервые, могут
побаиваться выражения его быстро вращающихся глаз.
С глазами он ничего не мог поделать, поэтому в таких случаях он старался
успокоительно шевелить ушами, зная, что это обычно помогает.

"--* Сонная меланхолия абсолютно не опасна, это вам совершенно верно рассказали
ваши соседки.
Её можно вообще не лечить.
Но если вам так спокойней будет, то дайте вашему сыну вот этот порошок.
Желательно его принимать два раза в день, утром и перед сном.

Аптекарь протянул даме небольшую коробочку, приговаривая:

"--* Видите, у меня этого порошка несколько коробочек уже заготовлено заранее,
сонной меланхолии последнее время очень много случаев появилось, особенно среди
молодёжи.
Но вы не переживайте, давайте вашему сыну по пол"=ложки порошка утром и
вечером.
И пусть его запивает хорошенько, желательно чем"=нибудь сладким, потому что
порошок горький и довольно противный на вкус \ldots

Дама, наконец, расплатилась за все купленные ею снадобья, и, покачивая
разноцветными перьями на своей зелёной шляпке, удалилась.

"--* Только бы помогло лекарство, "--- доносилось уже из прихожей её
обеспокоенное бормотание.
"--- А то заснёт, или просто задремлет, и сразу ясно, что он не здесь, а где-то
очень далеко \ldots

"--* Здравствуйте, давно вас не видел, "--- сказал аптекарь и энергично пожал
протянутую ему Гвенаэлем руку.
"--- Чем могу вам служить?

"--* Есть у вас что-нибудь от зубной боли? "--- спросил Гвенаэль.
"--- А то что-то один из зубов не даёт мне покоя, и вчера болел, и утром сегодня,
сейчас, правда, немного полегче стало \ldots

Аптекарь взял приставную деревянную лесенку, и полез доставать одну из
фарфоровых банок, стоящую на верхней полке, почти под самым потолком.

"--* Помощника моего нет на этой неделе, приходится самому всё делать, "---
сказал он Гвенаэлю, при этом белая аптекарская шапочка слетела с его головы,
обнажив большой и абсолютно лысый череп.
Он спустился, неловко держа под мышкой цветную фарфоровую банку, и уже только
внизу её открыл.

В банке оказались какие-то неровные маленькие светлые шарики, он отсыпал их в
приготовленную для Гвенаэля коробочку.
Кажется, он так и не заметил пропажу своей белой шапочки, и Гвенаэль постеснялся
указать ему на то, что она лежит на полу около приставной деревянной лестницы.

Улыбаясь, он сказал аптекарю: \enquote{До свидания}.

"--* До свидания, "--- ответил аптекарь.
"--- Не забывайте принимать лекарство каждый раз, когда зуб будет болеть.
Только заболит "--- сразу съедайте два, три или даже четыре шарика!
Можно сразу несколько съедать, они ведь совсем маленькие!

И он в знак прощания зашевелил своими большими красноватыми ушами.

До встречи с Филипоном у Гвенаэля оставалось ещё по крайней мере полтора часа.
Их встреча была назначена в харчевне \enquote{Жирная похлёбка}.
Это было на самой окраине города, довольно далеко от старой аптеки, но идти туда
было всё же ещё рано.
Он перешёл по Большому Горбатому мосту через Кортевеговку и стал бродить по
городу, чтобы как-то убить остававшееся время.

Для начала он зашёл в Кошачий переулок.
В путеводителях по городу было написано, что назван он Кошачьим потому, что был
таким узким, что в старые времена кошки любили прыгать с крыш домов на одной его
стороне на крыши домов на другой.
На самом деле так было не только в старые времена, но и ещё совсем недавно.
Только вот после прихода к власти императора R.~ле~Кина и его приказе об отлове
кошек они моментально отовсюду исчезли, даже здесь, в Кошачьем переулке, ни
одной теперь кошки не увидишь больше.

Не только сам переулок был узким, но и дома, на нём стоящие, тоже были узкими
шириной в два"--~три окна, не больше.
Зато некоторые были довольно высоким, в пять или шесть этажей.
Для этого города, где многие дома были двух или трёхэтажными, даже четыре или
пять этажей это не так уж мало.

Проходя мимо дома номер~53, Гвенаэль невольно остановился и поднял глаза.
Где-то здесь, под самой крышей этого дома, жил Омер "--- красавец, любимец
женщин, и, главное, самый читаемый из современных писателей.

Последнее обстоятельство не могло не вызывать зависти у Гвенаэля.
Вообще"=то, ему не на что было жаловаться.
То, что будучи таким молодым, он успел опубликовать уже несколько книжек, уже
само по себе было достаточно большим успехом.
И его читали, да, он был уверен, что его читали.
Более того, один небезызвестный литературный критик недавно крайне лестно
отозвался о его стиле.
Пожалуй, когда Омеру было столько же лет, сколько было сейчас Гвенаэлю, его
вообще никто ещё не знал!

Но всё это мало утешало молодого писателя.
Ведь Омера читают все, абсолютно все, и почти все им ужасно восхищаются!
Даже Альсбета!

Конечно, что касается Альсбеты, дело отчасти было в том, что Альсбету с Омером
связывала давняя и не совсем обычная дружба.
Когда она была ребёнком, Омер был назначен её опекуном.
Сейчас, когда принцесса стала взрослой, это не имело большого значения, да и
король, когда-то издавший указ об опекунстве принцессы, теперь, после прихода к
власти императора R.~ле~Кина, находился в далёком изгнании.
Говорили, что он живёт теперь на мысе Странных аттракторов, а может быть, даже и
ещё дальше, где-то на одном из островов опасного для проходящих через него
кораблей Новобермудского тропического треугольника \ldots

\enquote{Всё-таки раньше было веселей, когда кошки прыгали здесь над головой с одной
стороны переулка на другую}, "--- подумал Гвенаэль и спустился к площади, на
которой стояла городская ратуша.

\enquote{Ещё больше часа до встречи с Филипоном в \enquote{Жирной похлёбке}},
"--- сказал он сам себе, глядя на большие круглые часы на ратушной башне.
Надо ещё где-нибудь побродить немного.

Неподалёку от площади начиналась Безымянная улица, та самая, на которой жила
принцесса.
Обычно, стоило Гвенаэлю оказаться в этой части города, как ноги сами собой вели
его к её дому.

Так были и в этот раз.
Он уже свернул на Безымянную улицу, уже увидел на немного возвышающемся над
соседними улицами холме знакомый двухэтажный дом "--- тот самый дом, крыльцо
которого подпирает стоящий на плечах деревянный человечек.
Но тут, вопреки обыкновению, он не стал подходить к дому Альсбеты и свернул в
один из пересекающих Безымянную улицу переулков.

Его мучили угрызения совести.
Принцесса наверняка расстроилась бы, узнай она о его предстоящей встрече с
Филипоном в харчевне \enquote{Жирная похлёбка}.
Ей вообще не нравится общение Гвенаэля с людьми, имеющими отношение к новой
императорской власти.

Ему и самому больше нравился старый король, чем пришедший к власти путём
государственного переворота генерал R.~ле~Кин, называемый теперь императором
R.~ле~Кином.
Впрочем, самого императора он почти не знал, и ему любопытно было узнать его
получше "--- может быть, он и не был таким плохим, как многие это считали.

"--* Поскорее бы покончить с этой встречей в \enquote{Жирной похлёбке}, и
выбросить её из головы!
И, главное, не проболтаться потом принцессе о том, что я встречался со старшим
Филипоном!

И дело было не только в том, что Филипон имел какую-то, не совсем самому
Гвенаэлю понятную, связь с окружением нового императора.
С Филипоном вообще была отдельная история.
Ходили слухи, что и он, и его возничий Филипон Второй младший "--- людоеды.
Это, конечно, было совершенным вздором!
Правдой было то, что жили они на далёкой окраине города, в полуразрушенном
грязноватом пригороде.
Когда-то, когда принцесса была ещё совсем маленькой, её родители пропали без
вести.
Последний раз их видели перед тем, когда их карета свернула поздним вечером в
сторону улицы, на которой жили два Филипона.
Но, безусловно, абсолютно абсурдно было бы считать, что родители Альсбеты были
действительно ими съедены!

Во всё это Гвенаэль верил вполне искренне, но всё же совесть его была нечиста.

"--* Черт, я уже думал, что лечебные шарики помогли!
Но зуб снова стал болеть, даже ещё сильнее, чем раньше, "--- воскликнул он
сердито, и едва не споткнувшись о валявшееся на дороге колесо, свернул на
Большую Торговую улицу.
Он прошёл мимо ювелирного магазина, мимо фарфоровой лавки и магазина зеркал,
задержался минутку около витрины часовщика, рассматривая большой круглый
циферблат старинных часов, у которых минутная стрелка была отломана и оставалась
только одна часовая.

Потом он свернул на улицу, где находились магазины попроще: мебельные и посудные
лавки, магазины одежды и игрушек.
Затем вышел на длинную"=длинную и не очень прямую улицу, на которой почти все
дома уже были жилыми, только изредка попадалась булочная или зелёная лавка.
Когда эта длинная улица закончилась, Гвенаэль не совсем был уверен, куда ему
дальше надо было идти.
Филипон до этого достаточно подробно описал ему дорогу в \enquote{Жирную
похлёбку}, но переулки, по которым шёл Гвенаэль, были настолько похожи друг на
друга, что он всё больше и больше сомневался в правильности выбранного им пути.

Он внимательно смотрел по сторонам, старясь хотя бы запомнить, куда он идёт, но
повсюду были однообразные низкие дома, окружённые однообразными грязными
двориками.
Из одного такого двора выбежал и со стремительной скоростью бросился ему под
ноги небольшой поросёнок.

Гвенаэль вскрикнул и побледнел, его лицо, и без того довольно серое в последние
дни, стало ещё более серым и даже чуточку зелёным.

"--* Какой же я дурак, "--- сказал он сам себе, понемногу успокаиваясь.
"--- Это поросёнок, обычный розовый поросёнок.
Розовый, и вовсе не чёрный.
Маленький симпатичный розовый поросёночек.

То, что поросёнок был розовым, было, пожалуй, некоторым преувеличением "--- он
настолько был перемазан грязью, что его настоящий цвет было трудно определить.
О его симпатичности тоже можно было поспорить: и глазки, и свиное рыльце
казались довольно свирепыми.
Но что правда, то правда "--- это был вполне обычный поросёнок, и по всей
видимости он ничего общего не имел с чёрными свиньями, заполонившими последние
время все подходы к императорскому дворцу "--- ко дворцу, который раньше был
королевским, и который теперь, когда в нём поселился R.~ле~Кин, назывался
императорским.

Всем, не только Гвенаэлю, не давала покоя недавняя история про чёрную свиную
лихорадку.

Эта была очень тяжёлая болезнь, которой заразиться можно было только от чёрных
свиней.
Никто не знал, каким образом свиньи появились в городе, вернее, на окраине
города, там, где стоял императорский дворец.
Специально ли их привёл туда бывший генерал и нынешний император R.~ле~Кин?

Жили они главным образом на холме, на котором стоял дворец, и сам R.~ле~Кин, как
и вся его гвардия, был по всей видимости иммунен к чёрной свиной лихорадке.
Для всех же остальных войти во дворец стало теперь очень непросто.
Впрочем, во время общественных приёмов свиней прогоняли от главного входа, так
что приглашённые на императорские приёмы гости могли всё же войти во дворец, не
рискуя заразиться этой малоизученной тяжёлой болезнью.

В конце концов оказалось, что Гвенаэль не так уж сильно отклонился от
положенного пути.
Ему удалось в итоге найти харчевню \enquote{Жирная похлёбка}, о приближении к
которой он догадался, услышав оживлённые и возможно несколько пьяные голоса.
Улицы, по которым он шёл, были настолько тихими, что шум, доносящийся из
\enquote{Жирной похлёбки}, был слышен ещё за насколько кварталов до неё.

Он вошёл в довольно большое и плохо освещённое помещение харчевни и стал
оглядываться по сторонам, ища глазами Филипона.

"--* Довольно странное заведение, "--- пробормотал он себе под нос.
Он находился в большом зале, окружённом неровными деревянными стенами, с низко
нависшим деревянным потолком.
За большими деревянными столами, не покрытыми скатертями, сидели весьма
подозрительного вида плохо одетые люди, одни из них пили из больших глиняных
кружек, другие что-то ели из больших глиняных мисок.

"--* Хо-хо, вот и вы, наконец! "--- произнёс низкий и немного булькающий голос
Филипона, и, обернувшись, Гвенаэль увидел и самого толстяка, сидящего за круглым
деревянным столом и поедающего горячий, ещё дымящийся суп.

Гвенаэль поздоровался и присел за стол напротив Филипона.
Время было обеденным, но глядя на мутный странный суп в большущей миске, стоящей
на столе перед Филипоном, Гвенаэль совершенно утратил желание есть.
Он изо всех сил старался скрыть отвращение, возникающее у него при виде этой
тёмной горячей жидкости, которую Филипон уплетал с огромным аппетитом "--- даже
на бороде и усах у толстяка были капельки супа и небольшие кусочки плававшего в
супе мяса "--- он ел настолько быстро, что у него не было времени утереть свой
рот.

Гвенаэль извинился, сказал, что не голоден, и попросил подошедшего к ним хозяина
харчевни сделать ему бутерброд с сыром.

"--* Бутерброд? "--- несколько насмешливо ответил хозяин.
"--- Вообще-то, у нас здесь принято есть похлёбку, ей, собственно, и славится
наше харчевня.

Но всё же он вскорости принёс заказанный Гвенаэлем бутерброд "--- пару толстых
кусков козьего сыра, положенных в разломанную надвое серую и довольно чёрствую
булку.

Гвенаэль приступил к еде и одновременно стал слушать то, что должен был сообщить
ему Филипон.

"--* Дело моё касается материй, так сказать, литературных, хе-хе, "--- начал
свой рассказ бородатый толстяк.
"--- Это, конечно, неудивительно, так как вы являетесь автором преувлекательных,
позволю себе это выражение, сочинений \ldots

\enquote{Черт побери, к чему это он клонит? "--- подумал про себя Гвенаэль.
"--- Ни за что не поверю, что он читал хотя бы одну из моих книжек.
Вообще, сомневаюсь, что он в принципе что-то читает разве что газеты иногда
может пролистывать \ldots}

Филипон же тем временем стал переходить к сути своего рассказа.
Дело было в том, что по разным соображениям ему был неугоден Омер, и он надеялся
найти в Гвенаэле союзника для борьбы с этим заносчивым и чересчур популярным
писателем.
Должно быть, многие догадывались, что Гвенаэль очень завидует Омеру.
К тому же, как намекнул Филипон, ему из очень секретных и достоверных источников
было известно, что через несколько месяцев во дворце состоится церемония
посвящения в придворные писатели, и что Омер был единственным представляющим для
Гвенаэля опасность соперником.

Так, во всяком случае, считал Филипон.
Увлёкшись своим рассказом он даже забыл про недоеденный суп, который остывал на
дне стоящей у него перед носом большой глиняной миски.
Время от времени толстяк вопросительно посматривал на Гвенаэля, но тот молчал и
колебался.
Не то что бы он всерьёз рассматривал это предложение вступить в какой-то
сомнительный заговор против Омера.
Как бы он не недолюбливал его, как бы ему не завидовал, такого рода интриги были
совершенно против его правил.

Но всё же, с другой стороны, титул придворного писателя был огромной честью.
Это он только так назывался придворный писатель, само название мало о чём
говорило.
Это было звание, присуждаемое величайшим писателям, и церемонии посвящения
проходили не чаще, чем раз в десять или пятнадцать лет.
Неужели у Гвенаэля действительно был шанс получить этот почётный титул?

И, невольно краснея, молодой человек стал слушать дальше о подробностях
придуманного Филипоном плана действий.

Для того, чтобы навлечь немилость на Омера, Филипон рассчитывал использовать
пасквиль, якобы написанный Омером для осмеяния нового премьер"=министра Пьерро,
и в котором также фигурировал он сам, Филипон, под видом "--- кого бы вы думали
"--- людоеда!

Гвенаэль всё ещё колебался некоторое время, но, вникнув внимательнее в детали
рассказываемого Филипоном плана, сердито мотнул головой и уверенно сказал:

"--* Нет, Омер никак не может являться автором сочинения, о котором вы говорите!

"--* Почему же, "--- попытался возразить ему Филипон, теребя от волнения свою
густую рыжую бороду.
"--- У нас есть достаточно веские доказательства того, что \ldots

"--* Нет, этого не может быть! "--- снова перебил его Гвенаэль.
"--- Я знаю книгу, о которой вы говорите.
Она написана давным"=давно.
Её автор, Евгений Шварц, жил в совсем другой стране и умер за много лет до
рождения нашего премьер"=министра.
Так что ни к нему, ни к вам, ни к Омеру вся эта история никакого отношения иметь
не может!

Произнеся эту последнюю фразу, Гвенаэль попытался быстро дожевать свой бутерброд.
Ему хотелось поскорее закончить этот нелепый разговор, уйти прочь и никогда
больше не возвращаться в эту малоприветливую грязноватую харчевню под не слишком
заманчивым названием \enquote{Жирная похлёбка}.

Но, дожёвывая последний кусочек своего бутерброда, Гвенаэль едва не закричал от
боли: его больной зуб наткнулся в сыре на что-то очень твёрдое.
В их краях был такой обычай: в головки козьего сыра иногда клали ядрышко лесного
ореха "--- считалось, что на счастье.
Но то, что попытался разгрызть своим больным зубом Гвенаэль, было гораздо
твёрже, чем орех.
Он осторожно ощупал языком то, обо что он только что чуть не сломал свои зубы,
потом выплюнул этот твёрдый острый предмет и окончательно убедился в своей
догадке: это был не орех, а кусочек ореховой скорлупы.

Насколько мог припомнить Гвенаэль, в отличии от самих орехов, найденная в козьем
сыре скорлупа ничего особо хорошего не предвещала.
