\phantomsection
\chapter*{Часть третья. Вместо эпилога}
\addcontentsline{toc}{chapter}{Часть третья. Вместо эпилога}

Неподалёку от города был лес, обычный хвойный лес, меж сосен которого стояли
огромные камни.
Вроде бы, камни эти появились в лесу относительно недавно, столетия три или
четыре тому назад, во всяком случае более ранних упоминаний о них не сохранилось.

В позапрошлом веке секрет этого леса знали художники.
Камни были воротами в другой мир.
Неверно думать, что есть только один способ в него попасть: задуматься о
математике, и только один способ из него вернуться: заснуть.
Есть и другие способы попадать с одной стороны на другую, но все они более
редкие, чем математический.
Камни были одним из них.

Время от времени большие камни могли открываться, и, если особенно повезёт, из
них мог кто"=нибудь выскочить с той, другой стороны.
Это мог быть, например, крестьянин, козочка или большая светло"=золотистая
собака.

Художники собирались каждую неделю в лесу и ждали, как из камня кто"=нибудь
выскочит "--- и тут же его рисовали.
Вернее, рисовали они не всех, проникших в наш мир через эти своеобразные ворота,
а только тех, кто им больше всего понравится.
Особенно они любили рисовать крестьянок, козочек и овечек, а вот золотистых
собак, наоборот, почему-то никогда не рисовали.
Многие удивлялись: почему на их картинах самые обыденные вещи: овца
какая"=нибудь, или простая деревенская женщина "--- имеют такую непривычную
выразительность?
Художники в ответ довольно посмеивались и делали всё от них возможное, чтобы их
тайна никому не стала известна.

Но камни имеют свой возраст, со временем камни становятся твёрже.
Чем старше становились большие камни в сосновом лесу, тем труднее им было
открываться.
Происходило это всё реже и реже.
К началу прошлого века это стало столь редким событием, что художники полностью
забросили свои лесные встречи, а кроме них про секрет камней так до сих пор
почти никто и не знал.

Камни теперь открывались не только редко, но также всё менее широко, теперь
через отверстие никто уже не мог пролезть "--- ни козочки, не щенята "--- разве
может быть белые бабочки иногда ещё залетали.
Потом камни стали приоткрываться совсем чуть-чуть, даже мотыльку теперь было
через них не пролететь.
Потом вовсе перестали открываться.

Но в некоторых из камней остались трещины, и через них несколько десятков лет
назад начали прорастать деревья.
На вид они не слишком отличались от обычных, но если присмотреться
повнимательней, то всё же заметно, что деревья эти нездешние.

Профессор Мэйкпис Иеремия Тайт был одним из наиболее замечательных математиков
последнего века.
Он был известен самыми разнообразными своими достижениями, и одной из областей,
которыми он занимался, была теория деревьев.

Когда-то, каждый день, стоило ему только задуматься о математике, как он тут же
оказывался там, по ту сторону: на набережной Кортевеговки, на площади
Александера, в Кошачьем переулке или на улице Перевёрнутого Атланта, там, где
жили в то время родители принцессы Альсбеты.

Когда он думал об изобретённой им самим теории деревьев, он оказывался обычно
неподалёку от города, предместья которого "--- особенно с южной стороны,
смотрящей на Белую гору "--- были покрыты вечнозелёным лесом.

Но чем старше становился профессор Тайт, тем реже он попадал в тот другой,
математический мир.
В какой-то момент он обнаружил, что даже если он долго думает о математике,
непонятные перемещения больше не происходят.

Не то чтобы теоремы больше не доказываются, нет, он по"=прежнему занимается
математикой, (и во всяком случае одних воспоминаний о городе под Белой горой ему
должно хватить ещё на целую сотню теорем!).

Его результаты важны и интересны, зачастую они не менее остроумны, чем у его
более молодых коллег.
Но всё же это не то же самое.

Многие даже вряд ли заметили эту перемену в нем и в его работах, но сам-то он о
ней знает.
Он чувствует грусть, вспоминая тот, другой город, теперь ему недоступный.
И когда его печаль становится особенно сильной, он садится в пригородную
электричку и едет на ней до станции \enquote{Лесные глыбы}, бродит по хвойным
зарослям и вглядывается в деревья, проросшие через камни.
Он-то лучше всех "--- во всяком случае, лучше всех, находящихся по эту сторону
"--- знает, что это за деревья.

Мы сказали, что в прошлом веке камни уже по-настоящему не открывались.
Это не совсем верно.
Был один исключительный случай, о котором мало кто знает.
Чтобы рассказать о нем, надо сперва рассказать о том, кто такой Александр
Гротендик.

Это ещё один совершенно замечательный математик второй половины ушедшего
столетия.
Он доказал множество прекрасных теорем, но ещё больше, чем доказанных им теорем,
было количество придуманных им новых научных теорий.

Многие математики его очень любили, потом что теории эти были абсолютно
гениальными.
Но некоторые всё же его не любили, потому что характер у него был довольно
тяжёлый.
Если он решал что"=либо сделать, то никогда не менял потом своего решения.

Про него часто рассказывают, что он был высокого роста, что он брил наголо свою
голову, что родным языком его матери был немецкий, что его отец родился в
Белоруссии, но все эти детали его биографии не имеют, на наш взгляд, большого
значения.

Ещё говорят, что математикой он занимался с утра до вечера, что все, и
почитатели, и недруги его с нетерпением ждали, какую же новую теорию он вот-вот
придумает "--- потому что новые теории возникали в его голове не реже, чем раз в
месяц.

Всё это, несомненно, чистая правда, также как и является правдой то, что в один
прекрасный день Александр Гротендик подумал за завтраком:
не перестать ли мне заниматься математикой?

Он доел яйцо, сваренное в мешочек, дожевал свой утренний сухарик, и после этого
решение его созрело окончательно: математикой он больше никогда заниматься не
будет.

Но уже за чашечкой чая он понял, что решение это не слишком простое: он,
несомненно, будет очень сильно скучать по реке со странным названием
Кортевег~де~Фриз, по Старой Торговой улице, по Кошачьему переулку, по Белой горе
"--- короче по всему тому, что осталось с той, отныне закрытой для него стороны.

\enquote{Надо было хотя бы взять что"=нибудь оттуда себе на память},
"--- подумал Гротендик.

Он обвёл взглядом свою комнату.
Нет, ничего нет, он ни разу раньше не догадывался захватить с собой, засыпая,
что"=либо из того, другого мира.
Таким образом, для того, чтобы у него что"=нибудь осталось на память, необходимо
было снова туда хотя бы один раз попасть, а для этого, как известно, надо было
очень сильно задуматься о математике.
Но Александр Гротендик не был человек, который станет менять свои решения.
Раз уж он пообещал себе за завтраком, что никогда больше не будет заниматься
математикой, значит, так тому и быть.

\enquote{Раздобыть бы хотя бы какую"=нибудь мелочь: обрывок утренней газеты,
камешек с набережной Кортевеговки или на худой конец крылышко голубя, сидящего
на рогах скульптуры в центре площади Александера}, "--- сказал сам себе Гротендик.

Через минуту его глаза радостно сверкнули "--- не математической, а самой
обычной радостью "--- потому что он нашёл выход из сложившейся ситуации.

Он взял с собой большой холщовый мешок и направился в пригородный лес, в тот
самый, в котором раньше любили собираться художники.
В то время электрички туда не ходили, железнодорожной станции \enquote{Лесные
глыбы} ещё и в помине не было, так что ездить туда приходилось на автобусе и
дорога была довольно долгой.

Александр Гротендик был единственным пассажиром, вышедшим на окраине леса с
большими камнями.
По выходным дням в этом лесу довольно часто встречались гуляющие люди (хотя было
их гораздо меньше, чем сейчас, когда туда провели железную дорогу), но, к
счастью, день был будним, в лесу никого не было видно, и это было на руку
Гротендику.

Он подошёл к одному из больших камней, положил перед собой холщовый мешок и стал
ждать.
Не прошло и получаса, как камень открылся, и из него выскочил маленький белый
козлёнок.
Гротендик поймал козлёнка, засунул его в мешок и стал ждать дальше.
Ещё через несколько минут: раз, два, три, четыре, пять "--- и из камня друг за
дружкой выскочило ещё несколько козлят.
Гротендик всех их тоже посадил в мешок, убедился в том, что отверстие камня
плотно закрылось за последним козлёнком, и отправился в обратный путь.

По будням автобусы ходили редко, ему пришлось ждать почти что целый час, но
другого способа выбраться из леса не было.

Подошедший, наконец, автобус не был полным.
Но всё-таки в нем было человек десять пассажиров, и все они очень удивились,
увидев входящего высокого мужчину, с большим лбом и наголо бритой головой, за
спиной которого был огромный холщовый мешок, всё время шевелившийся и
вздрагивающий.
Нельзя было также обойти вниманием и тот факт, что из мешка довольно часто
доносилось нежное негромкое блеяние.

Но Гротендик, не обращая внимания на удивлённые взгляды попутчиков, благополучно
доехал до города и вышел на остановке около центрального вокзала.
Есть очевидцы, которые утверждают, что видели, как он покупал после этого в
вокзальной кассе билет на ближайшей поезд в направлении Южного моря.

Не удаётся доподлинно выяснить, что с ним происходило потом и что с ним
происходит сейчас.
Ходят слухи, что он живёт где-то на юге, окружённый огромным козьим стадом.

Говорят также, что в головках козьего сыра, продающегося на рынке одного из
приморских городов, обнаруживают теперь время от времени золотистые ядрышки
крупных лесных орехов.
