\section{Конференция по теории мембран}

В последние дни людей в институте было намного больше, чем обычно, и повсюду
царила изрядная суматоха.
Дело в том, что на этой неделе проходила конференция по теории мембран, и со
всего света в институт съехалось множество физиков, от крупнейших специалистов
до аспирантов и даже студентов.
Молодёжи было особенно много.

По поводу этого события были даже нарушены некоторые из основных традиций.
Так, например, раньше в институте было принято пить только чай (не считая,
конечно, обеда: в столовой после него каждый пил то, что хочет), но теперь в
перерывах между докладами в чайной комнате появлялись также термосы с кофе.

Большинство участников конференции были теоретическими физиками, многие
занимались абстрактными областями физики, довольно близкими к математике, и всё
же по виду их довольно часто можно было отличить от математиков.
Они собирались группами по несколько человек и разговаривали, редко оставаясь в
одинокой задумчивости, и вообще казались более весёлыми, но в то же время и
несколько более легкомысленными, чем математические сотрудники и визитёры.

И хотя физиков на этой неделе было намного больше, чем математиков, последние
тоже время от времени попадались на глаза.
Одни заходили послушать доклады по теории мембран, другие просто занимались
своими делами, а некоторые из них, возможно, даже не подозревали о проходящей
сейчас конференции.
У этих последних лица часто были задумчивыми и осунувшимися, взгляд вдохновенным
и блуждающим, движения худых рук не совсем понятными "--- Омер так и не узнал,
что именно предпринял в своё время профессор Кон~Фу~дзе, но баланс в институте
восстановился.
Упитанных и румяных визитёров стало не больше, чем обычно, и вроде бы все из них
были в целости и сохранности.
Среди приехавших физиков тоже был некоторый процент людей если не толстых, то во
всяком случае с довольно цветущей внешностью, но за них уж точно нечего было
беспокоиться "--- сразу видно, что если что, они сами за себя постоят.

Оба Филиппа, Моррис и Мэйли, были, видимо, очень раздосадованы срывом своих
людоедских планов и последнее время в институте вообще не появлялись.
Впрочем, отсутствовали не только они, Омера Раджафера почему"=то тоже не было
видно на этой неделе.
Его отсутствие явно не нравилось Гвенаэлю Варангу, который уже не один раз за
этот день подходил к его кабинету, стучал, убеждался в том, что дверь заперта, и
раздосадовано уходил.

Вообще вид у Гвенаэля был крайне мрачный, лицо бледное какой"=то темно"=зелёной
бледностью, глаза красные и под ними мешки, как это бывает после ночи,
проведённой без сна.
Но никто на всё это не обращал внимания.
Даже столкнувшийся с ним в коридоре директор, обычно окружавший Варанга
почтительной заботливостью, в этот день только мимолётно поздоровался с ним и
поспешил куда-то дальше "--- у него, наверное, как и многих других, мысли на
этой неделе были полностью заняты конференцией по теории мембран.

После вечернего чаепития Варанг в очередной раз подошёл к кабинету Раджафера.
Он не очень верил в то, что тот мог прийти, но считал нужным в этом
удостовериться из какой-то настойчивой упрямости.
Поэтому он удивился и даже немного отпрянул от двери, увидев, что она приоткрыта.
Потом собрался с духом, постучал и решительно вошёл в кабинет.

Раджафер стоял перед доской с мелом в руке и, должно быть, думал о какой"=то
математической задаче.
Но его задумчивая отстранённость не остановила Варанга.

"--* Послушайте, не буду скрывать, не то что бы я когда-нибудь к Вам очень
хорошо относился, но всё же я не думал, что Вы способны на такое, "--- выговорил
он негромко, но яростно и замолчал на мгновение, потому что и от волнения, и от
быстроты, с которой он говорил, у него перехватило дыхание.

Омер в этот время неспеша перевёл свой взгляд с какой-то формулы на доске на
Гвенаэля.
Казалось, он не очень хорошо понимал, о чём тот говорил, или, может быть, ему
просто не хотелось отрываться от своих математических мыслей.

"--* Я и раньше подозревал, что Вы в большой степени можете управлять событиями
и с той, и с этой стороны, но я не понимал, каким образом Вы этого добиваетесь,
"--- продолжил Варанг.
"--- Я и сейчас всё ещё не понимаю, но не думайте, что это Вам так сойдёт!
Как"=нибудь я разберусь и пойму, как Вам это удаётся.
Во что бы то ни стало пойму "--- я даже дал себе зарок, что справлюсь с этим до
ближайшего сочельника!

Тут он остановился, несколько сожалея, что наговорил лишнего.
Если он хотел что"=либо узнать у Раджафера, нелепо было на него накидываться с
такими ребяческими угрозами.
Впрочем, казалось, что на Омера его слова не произвели ни малейшего впечатления.

"--* Я до сих пор не могу поверить, что Вы могли сделать такое!
По отношению к ней \ldots\ "--- Гвенаэль запнулся, и вид у него был такой, как
если бы он изо всех сил сдерживался, чтобы не заплакать.

"--* Когда я вгляделся в фотографию \ldots\
Раньше, ещё месяц назад, она выглядела по"=другому, я точно это помню.
Не было этого выражения глаз, и поворот головы был немного другой.
Я не понимаю, это Вы подменили фотографию, или она сама изменилась.

Омер выслушал сбивчивые упрёки Гвенаэля совершенно невозмутимо, после чего
сказал, казалось бы, без всякой связи с предыдущим:

"--* Когда Вы взяли письмо из моей рукописи, Гвенаэль, Вы могли заметить, что
оно не было закончено.

"--* Да, "--- еле слышно сказал в ответ на это Гвенаэль.

"--* Вы хотите прочитать конец письма?

Гвенаэль кивнул головой.
По его виду можно было подумать, что он вот-вот упадёт в обморок.
Он отказался от предложенного Омером кресла, стоя подождал, пока тот достанет
письмо из ящика стола, и взяв в руку несколько листочков, исписанных
неразборчивым, но знакомым ему почерком, спросил:

"--* Я могу прочитать его у себя?

"--* Нет уж, читайте здесь, я и так не люблю никому давать свои черновики.
Хватит и того, что мне пришлось восстанавливать по памяти первую половину письма.

При этих словах Гвенаэль покраснел (хотя ещё несколько минут назад по цвету его
лица трудно было поверить, что он может покраснеть) и углубился в чтение.

\enquote{Когда я пытаюсь написать тебе или о тебе, Альсбета}, "--- так начиналось
письмо, и Гвенаэль быстро пробежал глазами первую, и так ему хорошо знакомую
часть написанного:

\ldots\ так же беспомощен перед чистым листом бумаги \ldots

\ldots\ твои белоснежные волосы и твои голубые глаза \ldots

\ldots\ можно написать, например, что у тебя вытянутое лицо, крупные губы и
довольно большие уши \ldots

\ldots\ ямочка на правой щеке \ldots\ то, как ты прикусываешь нижнюю губу
\ldots\ еле заметно вздрагивающие ноздри \ldots\ узнаю твою походку, каблучки
твоих туфелек, касающиеся неровной мостовой \ldots

Он дошёл до фразы \enquote{мне всегда хотелось объяснить тебе, как ты красива},
и после этого стал внимательно читать.

\medskip
\ldots\ Мне всегда хотелось объяснить тебе, Альсбета, как ты красива, потому что
мне кажется, что молодых девушек сознание собственной красоты обычно делает
счастливыми, а мне так хотелось всегда, чтобы ты была счастлива,
очень"=очень"=очень счастлива!
С тех пор как я стал твоим опекуном, это казалось мне одним из самых важных моих
дел, основной лежащей на мне ответственностью.

И дело это было не из простых.
Как было, например, объяснить тебе исчезновение твоих родителей?
Все в городе догадывались, что они были съедены людоедом Филипоном, но разве
можно было сказать такое совсем ещё маленькой девочке?
Ты спрашивала: где мои мама и папа?
когда они вернутся?
И я не знал, что ответить.

Чтобы отвлечь тебя от грустных мыслей, я стал придумывать сказки, помнишь, я
рассказывал тебе множество сказок пока ты была маленькой?
Я пытался изменить окружающий мир, неуютный и неприветливый, я пытался
что"=нибудь с ним сделать, чтобы тебе в нем было проще быть счастливой.
Особенно много сказок приходилось придумывать когда ты болела.
Помнишь, когда у тебя была скарлатина, ты с утра до вечера заставляла меня
читать тебе вслух книжки или что"=нибудь рассказывать, а есть лекарства упорно
отказывалась?
Вообще, ты всегда терпеть не могла лекарства, даже аптеку на набережной
Кортевеговки любила обходить стороной.
И тогда, во время твоей скарлатины, я придумал сказку, в которой здание аптеки
превратилось в библиотеку, баночки на её полках стали толстыми учёными книгами,
а лысый разговорчивый аптекарь "--- библиотекарем.
Сказка тебе понравилась, ты даже согласилась съесть несколько ложек горькой
микстуры, а мне пришлось рассказывать дальше.

Я старался изменять окружающие тебя предметы таким образом, чтобы они
становились более интересными, а известных тебе людей делать более добрыми и
симпатичными.
Даже самые страшные люди из реального мира становились в моей сказке менее
пугающими: так, например, из людоеда Филипона и его угрюмого возничего Филипона
Младшего я сделал двух толстых профессоров, не очень приятных, но и не злодеев.
А их приятеля, хозяина харчевни \enquote{Жирная похлёбка}, прошлое которого было
не менее мутным, чем подаваемый в этой харчевне суп, я превратил в весёлого
шеф"=повара институтской столовой и лучшего в мире специалиста по пирожкам с
рябиновым вареньем.

Но чтобы ни происходило в моих сказках, настоящий мир оставался таким, каким он
был.
Пока ты была ребёнком, я очень расстраивался видя, что тебе трудно найти себе
друзей.
Дети из богатых семей не хотели с тобой дружить потому что ты не была богатой, а
детей из бедных семей смущало то, что ты как никак принцесса.
Кстати, ни те, ни другие не были злыми или плохими детьми.
Просто с богатыми детьми было трудно познакомиться, потому что на улице они не
играли, а встречались на частых празднествах, устраиваемых своими богатыми
родителями.
А тебя, несмотря на хвое королевское происхождение, на эти праздники звали редко.
А если иногда и звали, ты, в своём единственном парадном (и всё же слишком
скромном) платьице слишком выделялась из толпы других детей, смущалась и с
нетерпением ждала конца торжества, когда, наконец, можно было идти обратно домой.

А с бедными детьми тоже было непросто подружиться, их смешила твоя литературная
речь (ты всегда много читала, и неудивительно, что некоторые произносимые тобой
слова другие дети просто не понимали), их смешила также твоя немного балетная
осанка "--- ты никогда не занималась танцами, но, видимо, осанка перешла к тебе
по наследству от твоей матери \ldots

Потом ты выросла и стала такой красивой и очаровательной девушкой, что мне стало
казаться, теперь никаких таких проблем больше не будет.
Появился Гвенаэль, и я глазом не успел моргнуть, как вы объявили, что
собираетесь пожениться.

Кое-что меня смущало в характере Гвенаэля: его честолюбие, его какая"=то мрачная
целеустремлённость "--- целеустремлённость не совсем понятно куда.
Но, с другой стороны, я знал, что он завидует моей литературной популярности, и
пытался себя убедить в том, что это единственная причина по которой у нас с ним
не сложились хорошие отношения, в том, что когда он сам напишет побольше книг,
это всё само собой пройдёт, в том, что в целом он молодой человек милый и
симпатичный, и, главное, что тебе он очень нравится "--- какое всё остальное
могло иметь значение?

Раз Гвенаэль был теперь твоим женихом, мне хотелось, чтобы он был как можно
лучше, и я тут же придумал сказку, в которой Гвенаэль был очень похож на
настоящего Гвенаэля, но при этом был ещё более талантливым и преуспевающим, чем
на самом деле.
Но тебе эту сказку я рассказывать не стал, ведь ты давно уже выросла, а сказки
полагается рассказывать только маленьких детям.
К тому же ты считала, что знаешь Гвенаэля гораздо лучше, чем я, и вряд ли
нуждалась в моих рассказах. Однажды ты сказала, что если когда"=нибудь будешь
счастлива, то только с ним.
Такие фразы люди очень часто произносят просто так, но я почувствовал, что в
этом случае именно так оно и есть.

Мне казалось, вот он, наконец, единственный шанс сделать тебя счастливой в этом
городе, всегда бывшем к тебе враждебным, я думал, любовь Гвенаэля всё исправит и
заставит тебя забыть твоё не слишком счастливое детство.
Но я ошибался \ldots\
Я давно уже догадывался, что ошибался, но не хотел себе в этом признаваться и
надеялся, что как"=нибудь ещё всё уладится.
Но после нашей встречи, после того, как выяснилось, что Гвенаэль украл тогда из
моей рукописи письмо, которое отправил тебе и которое так тебе понравилось,
после этого разговора я увидел в твоих глазах такое отчаяние, что впервые в
жизни по"=настоящему за тебя испугался.

Как мне сделать тебя счастливой в этом городе, узкие улицы которого ты всегда
немного недолюбливала, в городе, где у тебя не было друзей, где соседки без
конца судачили о твоей слишком скромной "--- для принцессы "--- одежде, о твоих
слишком больших (хотя на самом деле совершенно очаровательных!) ушах, иногда, с
притворной жалостью, о твоих пропавших родителях \ldots\
Стоило мне представить, с какой радостью и с каким энтузиазмом они примутся
сплетничать о твоей несостоявшейся свадьбе с Гвенаэлем, как все городские улицы
и переулки начинали казаться мне совершенно невыносимыми, во всяком случае, было
невозможно тебя здесь дальше оставлять!

Я вспомнил, что в детстве ты не слишком любила городскую жизнь.
Когда погода была ясной, мы ходили с тобой смотреть на возвышающуюся с юга от
города Белую гору, и ты без конца меня просила рассказывать сказки о белых
животных, живущих на горе: о белых золотистых собаках, о бело"=голубых орлах,
о пасущихся неподалёку от селений из светлого камня белых козочках.

Наверное, надо было поселить тебя там, в горах, пока ты была маленькой, но я не
знал, как это было организовать "--- в городе, по крайней мере, я мог за тобой
присматривать.
К тому же, мне казалось, что будучи принцессой ты обязана получить хорошее
городское образование: помнить годы правления всех королей последней династии,
уметь писать стихи триадическим анапестом и знать, что если у прямоугольного
треугольника катеты равны двадцати восьми и сорока пяти, то гипотенуза всегда
будет равна пятидесяти трём \ldots

В общем, в голове у меня тогда была сплошная ерунда, как я теперь понимаю, и
заплатить за эту ерунду пришлось слишком дорого. \enquote{Я никогда не смогу
быть счастлива без Гвенаэля}, "--- сказала ты когда"=то, так просто сказала, но
не оставляя не малейшей возможности для сомнений в истинности сказанного, и
потом эти слезы в твоих безнадёжных глазах, когда ты узнала о подделанном им
письме.

Ты права, как не горько мне это признавать, но в твоём обычном обличье ты не
можешь быть счастлива в нашем неприветливом городе.
Но я решился наконец и отпускаю тебя: беги прочь из него, беги быстро"=быстро и
не останавливайся, пока не достигнешь белых лугов и светлых рощ стоящей на юге
Белой горы.
Там ты ещё можешь быть счастливой, я знаю, ты уже отчасти забыла своё прошлое и
чувствуешь себя намного свободней и лучше.
Я представляю тебя бегущей по светлой горной тропе, и сердце моё наполняется
радостью.
Я вижу, как ты наклоняешься к ручью чтобы из него попить, и в прозрачной воде
отражаются твои голубые глаза и твоя белая чёлка, ты пьёшь, и напившись,
прикусываешь левей краешек твоей крупной нижней губы, довольно поводишь большими
белыми ушами, откидываешь за них твою густую светлую гриву и весело скачешь
прочь от ручья, в сторону душистого луга, и твои маленькие копытца стучат по
прибрежным камням.
Когда я слышу этот стук, этот ритм, который я ни с чем никогда не спутаю, мне
становится немного грустно.
Я думаю о том, что теперь мне уже не поболтать с тобой, не поспорить с тобой о
теореме Пифагора, о мелодике классического древне"=королевского эпоса "--- или о
чём мы только с тобой не любили спорить! "--- но это не очень важно, теперь я
знаю: и по звуку твоих лёгких шагов, когда ты перепрыгиваешь с камня на камень,
и по свету в твоих голубых глазах, и по еле заметной складочке на твоей светлой
правой скуле "--- знаю, что ты наконец счастлива \ldots

\medskip
Дочитав письмо, Гвенаэль положил его на стол, вышел, не глядя на Раджафера, из
его кабинета, вернувшись к себе бережно поднял с пола брошенный им до этого
портрет белой дикой лошади, повесил его обратно на стену над письменным столом,
и, впервые за последние дни, беззвучно заплакал.
