\section[Магазин экспериментальных продуктов]
{Магазин экспериментальных продуктов. Разговор с профессором Раджафером}

Константин проснулся рано утром и сонным взглядом обвёл малознакомую ему комнату.
Кажется, это здесь он поселился вчера вечером "--- вон даже около письменного
стола валяется его ещё нераспакованный рюкзак.
Он встал с постели, подошёл к окну и выглянул в окно.
Вокруг стояли маленькие избушки, крыши которых были покрыты тонким слоем снега.
Дальше за избушками виднелись хвойные деревья, уже виденные им вчера
полусосны"=полуёлки, они тоже были припорошены снегом.
А вот на земле снег уже совсем растаял, дорожки между избушками были, судя по их
виду, покрыты какой"=то тёмной слякотью.

Сами избушки были маленькими, одноэтажными "--- если не считать за второй этаж
крошечные помещения, которые, должно быть, находились под сводом их треугольных
крыш.
Константин по всей видимости находился в доме побольше, потому что его окна были
немножко выше соседних домов.
Можно было, наверное, открыть окно и, выглянув из него, посмотреть, сколько
этажей было в его доме, и на каком из них он сам находился, но молодой человек
поленился это сделать.
На улице явно было довольно холодно, а в его комнате тепло и уютно, поэтому
открывать окно, особенно не полностью ещё проснувшись, совершенно не хотелось.

Он стряхнул с себя последние остатки сна и после этого уже совершенно ясно
вспомнил, что находится он в Институте высших научных исследований, вернее, в
маленькой деревеньке неподалёку от института, в которой жили все его посетители,
а также некоторые из профессоров.
В комнату, в которой он ночевал, он поселился вчера уже поздно вечером, уставший
после длинного дня, включавшего перелёт на самолёте и разнообразные впечатления
от института.
Поэтому, придя сюда вчера вечером, он почти что сразу заснул, даже не успев её
оглядеть, но теперь, выспавшись, он с любопытством рассматривал квартиру, в
которой ему предстояло жить ближайший месяц.

Его комната была маленькой, но уютной и очень чистой, должно быть, совсем
недавно отремонтированной.
Через небольшой коридор из неё можно было пройти в ещё более маленькую кухню.
Было здорово, что в течение месяца он сможет пожить совсем один, в Корнелле он
снимал квартиру вместе с ещё двумя студентами, и ему это нравилось, но пожить
одному тоже было неплохо.
К тому же, хотя и в Итаке его соседи редко этому мешали, здесь будет совсем
хорошо заниматься дома математикой, никто и ничто не должно его здесь от этого
отвлекать.

Но пока что было время позавтракать, и он стал оглядывать то, что находилось в
его небольшой кухоньке.
Плита, посудомоечная машина, холодильник "--- он открыл его дверцу и убедился в
том, что холодильник было совершенно пуст.
Вчера, пообедав в институте, он непредусмотрительно забыл о покупке еды.
Даже спички он не подумал купить: в Корнелле он привык к электрической плите, а
если и видел когда"=либо газовые, то все они были с автоподжигом, здесь же плита
была газовая и несколько старого образца, для того, чтобы её зажечь, кажется,
требовались спички или зажигалка.
Он не курил, поэтому зажигалки у него не было, но у него было какое"=то
неопределённое воспоминание, что спички он где-то недавно видел.

Он вернулся в комнату и только сейчас обнаружил стоящий на столе поднос, на
котором находился большой коробок спичек, прозрачный целлофановый пакет
сухарей, несколько пакетиков чая и растворимого кофе, три маленьких
индивидуальных порции клубничного джема и несколько кубиков кускового сахара.
Вполне достаточный набор для незамысловатого завтрака, если в день приезда
посетитель института не успеет сходить в магазин или просто забудет это сделать,
как это забыл вчера Константин.

Но молодой человек поборол соблазн позавтракать сухариками с джемом и после
этого пойти в институт.
Всё-таки надо было наконец сходить в магазин, тем более что завтра в связи с
Новым годом наверняка все магазины будут закрыты.
К тому же после его странной болезни, похожей на Бери"=Бери, врач настоятельно
советовал ему хорошо питаться.
И, вернее, для него было более существенным то, что он проболтался тогда об
инструкциях врача своей маме, после чего она не оставляла его в покое, пока он
не поклялся самыми торжественными клятвами, что отныне так и будет это делать:
есть рыбу, мясо, овощи и фрукты, и не только не забывать обедать и ужинать, но и
завтракать всегда толком: свежим хлебом, не забывая про масло и свежие сливки
или молоко для кофе.
Не то что бы он всегда следовал этим данным своей маме обещаниям, но всё же
несколько следил теперь за тем, что он ел.
Поэтому вместо того, чтобы ограничиться сухариком с чашечкой растворимого кофе,
он решил пойти в булочную (кажется, по дороге из института домой он вчера
проходил мимо неё) и купить к завтраку хлеб или свежие булочки, и заодно
посмотреть, нет ли поблизости других магазинов.
Хотя ему очень неохота было это делать, но назавтра точно надо было запастись
едой, даже институтская столовая должна была завтра быть закрытой из-за Нового
года.
И даже всегдашнего пятичасового институтского чая завтра не предвиделось.

На лестнице Константин столкнулся с Младеном.
Это был хорватский постдок, с которым он познакомился вчера в институте.
Тот объяснил ему, что уже сегодня все обычные магазины закрыты, но есть
маленький магазинчик неподалёку в университетском городке, он должен быть
сегодня открыт.

"--* Только много еды там не купишь, "--- добавил Младен.
"--- Он маленький и немного странный к тому же.
Называется \enquote{Магазин экспериментальных продуктов}.

"--* Как это экспериментальных? "--- поинтересовался Константин.
"--- Мне хотя бы хлеба купить.
Но я так понял, ты сказал, что все булочные тоже уже закрыты?

"--* Как раз хлеб там скорее всего можно будет купить.
А экспериментальными продукты называются потому, что их изготавливают по новым,
недавно разработанным в университете технологиям.
Кстати, я подумал, что мне тоже неплохо кое-чего назавтра прикупить "--- хочешь,
пойдём вместе, я покажу тебе дорогу?

Университет и университетский городок находились примерно в сорока минутах
ходьбы от института.
Единственный способ туда добраться "--- идти пешком.
Если конечно, не считать за способ следующий вариант: доехать на метро до города
и пересесть над другую ветку, ведущую в университет.
Всё вместе это заняло бы не меньше двух часов, поэтому, понятное дело, обычно
никто так не поступал.

"--* А как, собственно, получилось, что университет находится на другой ветке
метро? "--- спросил через некоторое время Константин.
"--- Очень ведь неудобно.
Наверное, его построили когда метро уже было проложено, этот загородный кампус
ведь, кажется, относительно недавний?

"--* Ты что, не знаешь эту замечательную историю? "--- спросил его в ответ
Младен, ужасно довольный представившейся возможностью её кому-то рассказать.
Такое случалось нечасто, потому что даже новички, первый раз приезжающие в
институт, обычно где"=нибудь её уже слышали.

"--* Сразу же после постройки загородного университетского кампуса было решено
провести туда линию метро, чтобы студенты могли спокойно ездить на лекции,
"--- начал свой рассказ Младен, и глаза у него при этом почему"=то хитровато
поблёскивали.
"--- Было бы логично, чтобы эта новая линия метро прошла бы и мимо Института
высших научных исследований, до которого до этого времени можно было добраться
из города только на машине или автобусе, что, конечно, очень способствовало его
научной уединённой атмосфере, но при этом всё же было крайне неудобно.
Но мэр городка, на окраине которого построили университетский кампус и мэр
деревушки, около которого находился институт, были членами двух враждующих
политических партий.
Поэтому мэр городка, желая навредить своему недругу, вступил в сговор с ректором
университета, который в этот момент находился в ссоре с директором института, и
совместными усилиями им удалось добиться того, что линию метро провели на
максимально возможном расстоянии от института высших научных исследований.
Узнав об этом решении, директор института расстроился, но решил не сдаваться.
Прошло не меньше года до тех пор, пока ему не удалось мобилизовать имевшиеся у
него в политических кругах связи, и к этому времени строительство идущей в
университет ветки Юг--1 было почти завершено, её направление нельзя уже было
изменить.
Но директор и тут не стал сдаваться, и у него хватило энтузиазма и влияния для
того, чтобы добиться строительства ещё одной, 25-ой ветки метро, ведущей в
институт!
В конечном счёте для города это оказалось не так уж и плохо: по двадцать третьей
ветке метро в направлении Юг--1 можно было доехать до всех южных и
юго"=восточных пригородов, а двадцать пятая ветка метро (в направлении Юг--2),
проходя институт высших научных исследований, заворачивала немного в западном
направлении и доходила до многих юго"=западных деревушек.
Но вот для университета и института ситуация осталось довольно неудобной: до сих
пор, институтским профессорам, желающим зайти на какую"=нибудь лекцию в
университет, или, наоборот, университетским студентам и профессорам, желающим
посетить один из институтских семинаров, приходится ходить по этой длинной
лесной тропинке, по которой мы с тобой сейчас идём, "--- закончил свой рассказ
Младен.

Через некоторое время их тропа пересекла небольшой лесной ручей.
Неподалёку от деревянного моста, по которому они его перешли, бултыхались в воде
довольно крупные водяные крысы.

Деревья за ручьём довольно скоро из хвойных превратились в лиственные, и минут
через пятнадцать ходьбы они оказались на окраине университетского кампуса.

"--* Обычно здесь очень людно, "--- объяснил Младен.
"--- Но сейчас, перед Новым годом, все занятия давно уже закончились, и студенты
разъехались на каникулы\ldots

Магазин экспериментальных продуктов оказался совсем крошечным.
Его полки были заставлены в основном консервными банками, сделанными из
какого-то нового типа жести.

"--* Осторожно, не нажимай на вот эту красную точку на крышке, когда их
рассматриваешь, "--- предупредил Константина Младен.
"--- Они сделаны по новой самооткрывающейся технологии, стоит нажать на эту
красную точку, как они тут же откроются\ldots

Константин хотел купить несколько стаканчиков супа из китайской лапши, но Младен
его от этого отговорил.

"--* Китайскую лапшу лучше покупать в китайских магазинах или хотя бы в
супермаркете около университета, "--- сказал он своему приятелю.
"--- Подожди, через два дня после Нового года он должен уже открыться, а здесь
вся лапша производится находящейся в университете фирмой
\enquote{\foreignlanguage{english}{Brownian Loop}},
и суп из неё довольно своеобразный получается.

У Константина, напротив, вызвали опасения продававшиеся в этом магазине булочки.
Они были довольно неказистыми на вид, неровной формы: отчасти пухленькие,
отчасти плоские, но тут, наоборот, Младен за них вступился:

"--* Это как раз ужасно вкусный хлеб, очень специальная технология: его делают
на дрожжах промежуточного роста, поэтому он получается такой странной формы.
Но на вкус он очень интересный: нечто среднее между песочным и дрожжевым тестом,
так словами не объяснишь.
Я тебе советую купить их побольше, уверен, что тебе они должны понравиться.

Сделав необходимые на ближайшие дни покупки, друзья отправились в обратный путь:
по той же лесной дорожке, через ручей, в сторону институтского городка.
Купавшихся в ручье крыс теперь уже не было видно, зато они встретили двух
ярко"=рыжих белок, а на вершине одной из высоченных ёлко"=сосен, стоящей у
самого выхода из леса, сидела большая чёрная ворона.

Прощаясь с Младеном, Константин спросил у своего нового друга:

"--* Мне тут пришёл в голову такой вопрос: крысы, которых мы видели около ручья,
это нутрии или просто обычные крысы?

"--* Не знаю, "--- ответил Младен.
"--- У нутрий должен быть красивый мех, так что вроде не нутрии.
С другой стороны, они очень любят воду и почти всегда купаются в этом ручье.
Даже сейчас зимой, когда вода в нём, должно быть, ледяная\ldots\
Кстати о ручье.
Знаешь, если честно, про метро есть более реалистичная версия.
Этот ручей, хотя на вид небольшой, но почва под ним влажная до самой глубины,
возможно именно из-за этого под ним не удобно было прокладывать линию метро.
Но все всегда любят рассказывать про ректора, директора и двух враждующих
мэров\ldots

Константин занёс свои покупки домой, позавтракал дрожжевыми булочками
промежуточного роста, и пошёл в институт, чтобы продолжить чтение препринта
Раджафера.
Входя в здание института он столкнулся с самим Раджафером.
У Константина был некоторый вопрос, связанный с его недавно доказанной теоремой,
про который тот мог кое-что знать, возможно, даже не хуже Варанга.
Вчера за обедом, во время общего разговора, он не нашёл удобного момента, чтобы
о нём заговорить.
От Младена Константин знал, что у Раджафера бывает два разных состояния "---
когда он здоровается со знакомыми и когда он не здоровается, вероятно их не
замечая.
Сегодня он, должно быть, был в первом из двух состояний "--- даже на приветствие
мало знакомого ему Константина он вполне отчётливо ответил, и молодой человек
решил, что это хороший знак и что можно прямо сейчас задать ему свой вопрос.

Выслушав Константина, Раджафер задумался, потом глаза его на мгновение
вспыхнули, после этого он задумался ещё глубже, и глаза его стали
непроницаемыми.
У Константина возникло странное ощущение, как будто бы тот находился где-то не
здесь: вот тело его тут, но тёмная непроницаемая стена в его глазах как будто бы
закрывала и глаза, и его самого.
Из-за этого странного ощущения он не мог сказать, как долго размышлял Раджафер:
минуту, несколько минут или дольше.
Казалось, что время для Раджафера остановилось, и в какой"=то момент Константин
засомневался, думает ли тот действительно над заданным вопросом, или, может
быть, вообще про него забыл.

Но Раджафер закрыл на мгновение глаза, и когда он их открыл, они снова были
ясными, закрывающая их непроницаемая преграда куда-то исчезла.

"--* Вы некорректно поставили ваш вопрос, "--- сказал он Константину.
"--- Надо предполагать, что $\mathrm{K}(A)$ не имеет нетривиальных морфизмов.
Но если это предположить, то можно перейти к производной категории, после чего
всё становится понятным\ldots

Константин с восхищением слушал объяснения Раджафера.
Он предполагал, что вопрос его не был очень сложным, но всё-таки до этого
несколько дней о нём думал, а Раджаферу оказалось достаточно всего нескольких
минут, чтобы во всём с ним связанным совершенно детально разобраться.

Закончив своё объяснение, Раджафер удалился по коридору в сторону флигеля, в
котором находился его кабинет.
Только после этого молодой человек спохватился и вспомнил, что забыл у него
спросить про найденный вчера в библиотеке препринт о нулях дзета"=функций.
Ему было неловко теперь идти за Раджафером и снова его о чём-то спрашивать.
Правда, ему хотелось узнать, каким образом тому удалось добиться того, что
греческие буквы в формулах складываются в связный текст.
Казалось бы, это должно было быть очень трудно, специально ли Раджафер так
подбирал обозначения и порядок следования формул, или может быть всё получалось
само собой?
Впрочем, история с греческими буквами, чем больше он о ней думал, тем менее
правдоподобной она ему казалась.
Но по крайней мере само математическое содержание препринта его очень
интересовало.

\enquote{Надо получше понять то, что написано в этой статье, прежде чем
спрашивать о ней у Раджафера}, "--- решил Константин и пошёл в библиотеку.

Но на том месте, где он вчера его оставил, препринта сегодня не оказалось.
Должно быть, кто"=нибудь взял его почитать.

"--* Подожду пару дней, и если препринт не вернётся на своё место, спрошу о нём
после праздников у библиотекаря, "--- сказал сам себе молодой человек.
"--- А сейчас Новый год уже совсем на носу, и есть другие вещи, помимо нулей
дзета"=функций, которыми пора уже в связи с этим заняться!
